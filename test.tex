\documentclass{article}
\pagestyle{empty}

%\usepackage{color}
\usepackage{tabularx}
\usepackage{geometry}
\usepackage{comment}
\usepackage{longtable}
\usepackage{titlesec}
\usepackage{chngpage}
\usepackage{calc}
\usepackage{url}
\usepackage[utf8x]{inputenc}

% Fontsetup
\usepackage[T1]{fontenc}

\DeclareUnicodeCharacter {135}{{\textascii ?}}
\DeclareUnicodeCharacter {129}{{\textascii ?}}
\DeclareUnicodeCharacter {128}{{\textascii ?}}

\usepackage{colortbl}

% must come last
\usepackage{hyperref}
\definecolor{linkblue}{rgb}{0.11,0.56,1}
\definecolor{inactive}{rgb}{0.56,0.56,0.56}
\definecolor{gvm_debug}{rgb}{0.78,0.78,0.78}
\definecolor{gvm_false_positive}{rgb}{0.2275,0.2275,0.2275}
\definecolor{gvm_log}{rgb}{0.2275,0.2275,0.2275}
\definecolor{gvm_hole}{rgb}{0.7960,0.1137,0.0902}
\definecolor{gvm_note}{rgb}{0.3255,0.6157,0.7961}
\definecolor{gvm_report}{rgb}{0.68,0.74,0.88}
\definecolor{gvm_user_note}{rgb}{1.0,1.0,0.5625}
\definecolor{gvm_user_override}{rgb}{1.0,1.0,0.5625}
\definecolor{gvm_warning}{rgb}{0.9764,0.6235,0.1922}
\definecolor{chunk}{rgb}{0.9412,0.8275,1}
\definecolor{line_new}{rgb}{0.89,1,0.89}
\definecolor{line_gone}{rgb}{1.0,0.89,0.89}
\hypersetup{colorlinks=true,linkcolor=linkblue,urlcolor=blue,bookmarks=true,bookmarksopen=true}
\usepackage[all]{hypcap}

%\geometry{verbose,a4paper,tmargin=24mm,bottom=24mm}
\geometry{verbose,a4paper}
\setlength{\parskip}{\smallskipamount}
\setlength{\parindent}{0pt}
\title{Scan Report}
\pagestyle{headings}
\pagenumbering{arabic}

\begin{document}

\maketitle

\renewcommand{\abstractname}{Summary}
\begin{abstract}
This document reports on the results of an automatic security scan.
All dates are displayed using the timezone ``Coordinated Universal Time'', which is abbreviated ``UTC''.
The task was ``Unnamed''.  The scan started at Fri Apr 22 12:21:15 2022 UTC and ended at Fri Apr 22 13:15:09 2022 UTC.  The
report first summarises the results found.  Then, for each host,
the report describes every issue found.  Please consider the
advice given in each description, in order to rectify the issue.
\end{abstract}
\tableofcontents
\newpage
\section{Result Overview}

\begin{longtable}{|l|l|l|l|l|l|}
\hline
\rowcolor{gvm_report}Host&High&Medium&Low&Log&False Positive\\
\hline
\endfirsthead
\multicolumn{6}{l}{\hfill\ldots (continued) \ldots}\\
\hline
\rowcolor{gvm_report}Host&High&Medium&Low&Log&False Positive\\
\hline
\endhead
\hline
\multicolumn{6}{l}{\ldots (continues) \ldots}\\
\endfoot
\hline
\endlastfoot
\hline
\hyperref[host:192.168.178.2]{192.168.178.2}&0&0&0&18&0\\
\hyperref[host:192.168.178.2]{gvm.fritz.box}&&&&&\\
\hline
\hyperref[host:192.168.178.37]{192.168.178.37}&0&0&0&22&0\\
\hyperref[host:192.168.178.37]{dreambox.fritz.box}&&&&&\\
\hline
\hyperref[host:192.168.178.20]{192.168.178.20}&0&0&0&13&0\\
\hyperref[host:192.168.178.20]{homeprinter.fritz.box}&&&&&\\
\hline
\hyperref[host:192.168.178.22]{192.168.178.22}&0&0&0&13&0\\
\hyperref[host:192.168.178.22]{homeprinter.fritz.box}&&&&&\\
\hline
\hyperref[host:192.168.178.1]{192.168.178.1}&0&0&0&22&0\\
\hyperref[host:192.168.178.1]{fritz.nas}&&&&&\\
\hline
\hyperref[host:192.168.178.24]{192.168.178.24}&0&0&0&2&0\\
\hline
\hyperref[host:192.168.178.23]{192.168.178.23}&0&0&0&3&0\\
\hyperref[host:192.168.178.23]{amazon-f0c31363c.fritz.box}&&&&&\\
\hline
\hyperref[host:192.168.178.29]{192.168.178.29}&0&0&0&4&0\\
\hyperref[host:192.168.178.29]{wohn-tl-sg108e.fritz.box}&&&&&\\
\hline
\hyperref[host:192.168.178.27]{192.168.178.27}&0&0&0&2&0\\
\hline
\hyperref[host:192.168.178.25]{192.168.178.25}&0&0&0&2&0\\
\hline
\hyperref[host:192.168.178.36]{192.168.178.36}&0&0&0&2&0\\
\hyperref[host:192.168.178.36]{iphone-fj.fritz.box}&&&&&\\
\hline
\hyperref[host:192.168.178.28]{192.168.178.28}&0&0&0&4&0\\
\hyperref[host:192.168.178.28]{s20-fe-von-carina.fritz.box}&&&&&\\
\hline
\hyperref[host:192.168.178.42]{192.168.178.42}&0&0&0&2&0\\
\hyperref[host:192.168.178.42]{nasale.fritz.box}&&&&&\\
\hline
\hyperref[host:192.168.178.41]{192.168.178.41}&0&0&0&2&0\\
\hyperref[host:192.168.178.41]{nasale.fritz.box}&&&&&\\
\hline
\hline
Total: 14&0&0&0&111&0\\
\hline
\end{longtable}
Vendor security updates are not trusted.\\
Overrides are off.  Even when a result has an override, this report uses the actual threat of the result.\\
Information on overrides is included in the report.\\
Notes are included in the report.\\
This report might not show details of all issues that were found.\\
Issues with the threat level ``Debug'' are not shown.\\
\\
This report contains all 111 results selected by the filtering described above.  Before filtering there were 111 results.\subsection{Host Authentications}
\begin{longtable}{|l|l|l|l|}
\hline
\rowcolor{gvm_report}Host&Protocol&Result&Port/User\\
\hline
\endfirsthead
\multicolumn{4}{l}{\hfill\ldots (continued) \ldots}\\
\hline
\rowcolor{gvm_report}Host&Protocol&Result&Port/User\\
\hline
\endhead
\hline
\multicolumn{4}{l}{\ldots (continues) \ldots}\\
\endfoot
\hline
\endlastfoot
\hline
192.168.178.37 - dreambox.fritz.box & SMB & Success & Protocol SMB, Port 445, User \\ \hline192.168.178.20 - homeprinter.fritz.box & SMB & Success & Protocol SMB, Port 445, User \\ \hline192.168.178.22 - homeprinter.fritz.box & SMB & Success & Protocol SMB, Port 445, User \\ \hline192.168.178.22 - homeprinter.fritz.box & SMB & Success & Protocol SMB, Port 445, User \\ \hline\hline
\end{longtable}
\section{Results per Host}

\subsection{192.168.178.2}
\label{host:192.168.178.2}

\begin{tabular}{ll}
Host scan start&Fri Apr 22 12:21:26 2022 UTC\\
Host scan end&Fri Apr 22 12:37:51 2022 UTC\\
\end{tabular}

\begin{longtable}{|l|l|}
\hline
\rowcolor{gvm_report}Service (Port)&Threat Level\\
\hline
\endfirsthead
\multicolumn{2}{l}{\hfill\ldots (continued) \ldots}\\
\hline
\rowcolor{gvm_report}Service (Port)&Threat Level\\
\hline
\endhead
\hline
\multicolumn{2}{l}{\ldots (continues) \ldots}\\
\endfoot
\hline
\endlastfoot
\hline
\hyperref[port:192.168.178.2 5432/tcp Log]{5432/tcp}&Log\\
\hline
\hyperref[port:192.168.178.2 25/tcp Log]{25/tcp}&Log\\
\hline
\hyperref[port:192.168.178.2 8081/tcp Log]{8081/tcp}&Log\\
\hline
\hyperref[port:192.168.178.2 9390/tcp Log]{9390/tcp}&Log\\
\hline
\hyperref[port:192.168.178.2 general/tcp Log]{general/tcp}&Log\\
\hline
\hyperref[port:192.168.178.2 22/tcp Log]{22/tcp}&Log\\
\hline
\end{longtable}


%\subsection*{Security Issues and Fixes -- 192.168.178.2}

\subsubsection{Log 5432/tcp}
\label{port:192.168.178.2 5432/tcp Log}

\begin{longtable}{|p{\textwidth * 1}|}
\hline
\rowcolor{gvm_log}{\color{white}{Log (CVSS: 0.0) }}\\
\rowcolor{gvm_log}{\color{white}{NVT: Services}}\\
\hline
\endfirsthead
\hfill\ldots continued from previous page \ldots \\
\hline
\endhead
\hline
\ldots continues on next page \ldots \\
\endfoot
\hline
\endlastfoot
\\
\textbf{Summary}\\
This routine attempts to guess which service is running on the
  remote ports. For instance, it searches for a web server which could listen on another port than
  80 or 443 and makes this information available for other check routines.\\

        \hline
        \\
\textbf{Vulnerability Detection Result}\\
\rowcolor{white}{\verb=An unknown service is running on this port.=}\\
\rowcolor{white}{\verb=It is usually reserved for Postgres=}\\

          \hline
          \\
\textbf{Solution:}\\
\\


        \hline
        \\
\textbf{Log Method}\\
Details:
\rowcolor{white}{\verb=Services=}\\
OID:1.3.6.1.4.1.25623.1.0.10330\\
Version used:
\rowcolor{white}{\verb=2021-03-15T10:42:03Z=}\\
\end{longtable}

\begin{longtable}{|p{\textwidth * 1}|}
\hline
\rowcolor{gvm_log}{\color{white}{Log (CVSS: 0.0) }}\\
\rowcolor{gvm_log}{\color{white}{NVT: PostgreSQL Detection}}\\
\hline
\endfirsthead
\hfill\ldots continued from previous page \ldots \\
\hline
\endhead
\hline
\ldots continues on next page \ldots \\
\endfoot
\hline
\endlastfoot
\\
\textbf{Summary}\\
Detection of PostgreSQL, an open source object-relational
  database system.\\
  The script sends a connection request to the server (user:postgres, DB:postgres)
  and attempts to extract the version number from the reply.\\

        \hline
        \\
\textbf{Vulnerability Detection Result}\\
\rowcolor{white}{\verb=Detected PostgreSQL=}\\
\rowcolor{white}{\verb=Version:       unknown=}\\
\rowcolor{white}{\verb=Location:      5432/tcp=}\\
\rowcolor{white}{\verb=CPE:           cpe:/a:postgresql:postgresql=}\\
\rowcolor{white}{\verb=Concluded from version/product identification result:=}\\
\rowcolor{white}{\verb=0x00:  52 00 00 00 0C 00 00 00 05 61 F9 DA 4B             R........a..K=}\\

          \hline
          \\
\textbf{Solution:}\\
\\


        \hline
        \\
\textbf{Log Method}\\
Details:
\rowcolor{white}{\verb=PostgreSQL Detection=}\\
OID:1.3.6.1.4.1.25623.1.0.100151\\
Version used:
\rowcolor{white}{\verb=2020-11-12T10:09:08Z=}\\

      \hline
      \\
\textbf{References}\\
\rowcolor{white}{\verb=url: https://www.postgresql.org/=}\\
\end{longtable}

\begin{footnotesize}\hyperref[host:192.168.178.2]{[ return to 192.168.178.2 ]}\end{footnotesize}
\subsubsection{Log 25/tcp}
\label{port:192.168.178.2 25/tcp Log}

\begin{longtable}{|p{\textwidth * 1}|}
\hline
\rowcolor{gvm_log}{\color{white}{Log (CVSS: 0.0) }}\\
\rowcolor{gvm_log}{\color{white}{NVT: SMTP Server type and version}}\\
\hline
\endfirsthead
\hfill\ldots continued from previous page \ldots \\
\hline
\endhead
\hline
\ldots continues on next page \ldots \\
\endfoot
\hline
\endlastfoot
\\
\textbf{Summary}\\
This detects the SMTP Server's type and version by connecting to
  the server and processing the buffer received.\\

        \hline
        \\
\textbf{Vulnerability Detection Result}\\
\rowcolor{white}{\verb=Remote SMTP server banner:=}\\
\rowcolor{white}{\verb=220 localhost.fritz.box ESMTP Postfix (Debian/GNU)=}\\
\rowcolor{white}{\verb=The remote SMTP server is announcing the following available ESMTP commands (EHL=}\\
\rowcolor{white}{$\hookrightarrow$\verb=O response) via an unencrypted connection:=}\\
\rowcolor{white}{\verb=8BITMIME, CHUNKING, DSN, ENHANCEDSTATUSCODES, ETRN, PIPELINING, SIZE 10240000, S=}\\
\rowcolor{white}{$\hookrightarrow$\verb=MTPUTF8, STARTTLS, VRFY=}\\

          \hline
          \\
\textbf{Solution:}\\
\\


        \hline
        \\
\textbf{Log Method}\\
Details:
\rowcolor{white}{\verb=SMTP Server type and version=}\\
OID:1.3.6.1.4.1.25623.1.0.10263\\
Version used:
\rowcolor{white}{\verb=2022-02-01T10:00:18Z=}\\
\end{longtable}

\begin{longtable}{|p{\textwidth * 1}|}
\hline
\rowcolor{gvm_log}{\color{white}{Log (CVSS: 0.0) }}\\
\rowcolor{gvm_log}{\color{white}{NVT: Services}}\\
\hline
\endfirsthead
\hfill\ldots continued from previous page \ldots \\
\hline
\endhead
\hline
\ldots continues on next page \ldots \\
\endfoot
\hline
\endlastfoot
\\
\textbf{Summary}\\
This routine attempts to guess which service is running on the
  remote ports. For instance, it searches for a web server which could listen on another port than
  80 or 443 and makes this information available for other check routines.\\

        \hline
        \\
\textbf{Vulnerability Detection Result}\\
\rowcolor{white}{\verb=An SMTP server is running on this port=}\\
\rowcolor{white}{\verb=Here is its banner : =}\\
\rowcolor{white}{\verb=220 localhost.fritz.box ESMTP Postfix (Debian/GNU)=}\\

          \hline
          \\
\textbf{Solution:}\\
\\


        \hline
        \\
\textbf{Log Method}\\
Details:
\rowcolor{white}{\verb=Services=}\\
OID:1.3.6.1.4.1.25623.1.0.10330\\
Version used:
\rowcolor{white}{\verb=2021-03-15T10:42:03Z=}\\
\end{longtable}

\begin{longtable}{|p{\textwidth * 1}|}
\hline
\rowcolor{gvm_log}{\color{white}{Log (CVSS: 0.0) }}\\
\rowcolor{gvm_log}{\color{white}{NVT: SSL/TLS: SMTP 'STARTTLS' Command Detection}}\\
\hline
\endfirsthead
\hfill\ldots continued from previous page \ldots \\
\hline
\endhead
\hline
\ldots continues on next page \ldots \\
\endfoot
\hline
\endlastfoot
\\
\textbf{Summary}\\
Checks if the remote SMTP server supports SSL/TLS with the
  'STARTTLS' command.\\

        \hline
        \\
\textbf{Vulnerability Detection Result}\\
\rowcolor{white}{\verb=The remote SMTP server supports SSL/TLS with the 'STARTTLS' command.=}\\
\rowcolor{white}{\verb=The remote SMTP server is announcing the following available ESMTP commands (EHL=}\\
\rowcolor{white}{$\hookrightarrow$\verb=O response) before sending the 'STARTTLS' command:=}\\
\rowcolor{white}{\verb=8BITMIME, CHUNKING, DSN, ENHANCEDSTATUSCODES, ETRN, PIPELINING, SIZE 10240000, S=}\\
\rowcolor{white}{$\hookrightarrow$\verb=MTPUTF8, STARTTLS, VRFY=}\\
\rowcolor{white}{\verb=The remote SMTP server is announcing the following available ESMTP commands (EHL=}\\
\rowcolor{white}{$\hookrightarrow$\verb=O response) after sending the 'STARTTLS' command:=}\\
\rowcolor{white}{\verb=8BITMIME, CHUNKING, DSN, ENHANCEDSTATUSCODES, ETRN, PIPELINING, SIZE 10240000, S=}\\
\rowcolor{white}{$\hookrightarrow$\verb=MTPUTF8, VRFY=}\\

          \hline
          \\
\textbf{Solution:}\\
\\


        \hline
        \\
\textbf{Log Method}\\
Details:
\rowcolor{white}{\verb=SSL/TLS: SMTP 'STARTTLS' Command Detection=}\\
OID:1.3.6.1.4.1.25623.1.0.103118\\
Version used:
\rowcolor{white}{\verb=2021-11-12T09:42:39Z=}\\

      \hline
      \\
\textbf{References}\\
\rowcolor{white}{\verb=url: https://tools.ietf.org/html/rfc3207=}\\
\end{longtable}

\begin{longtable}{|p{\textwidth * 1}|}
\hline
\rowcolor{gvm_log}{\color{white}{Log (CVSS: 0.0) }}\\
\rowcolor{gvm_log}{\color{white}{NVT: SSL/TLS: Version Detection}}\\
\hline
\endfirsthead
\hfill\ldots continued from previous page \ldots \\
\hline
\endhead
\hline
\ldots continues on next page \ldots \\
\endfoot
\hline
\endlastfoot
\\
\textbf{Summary}\\
Enumeration and reporting of SSL/TLS protocol versions supported
  by a remote service.\\

        \hline
        \\
\textbf{Vulnerability Detection Result}\\
\rowcolor{white}{\verb=The remote SSL/TLS service supports the following SSL/TLS protocol version(s):=}\\
\rowcolor{white}{\verb=TLSv1.0=}\\
\rowcolor{white}{\verb=TLSv1.1=}\\
\rowcolor{white}{\verb=TLSv1.2=}\\
\rowcolor{white}{\verb=TLSv1.3=}\\

          \hline
          \\
\textbf{Solution:}\\
\\


        \hline
        \\
\textbf{Log Method}\\
Sends multiple connection requests to the remote service and
  attempts to determine the SSL/TLS protocol versions supported by the service from the replies.\\
  Note: The supported SSL/TLS protocol versions included in the report of this VT are reported
  independently from the allowed / supported SSL/TLS ciphers.\\
Details:
\rowcolor{white}{\verb=SSL/TLS: Version Detection=}\\
OID:1.3.6.1.4.1.25623.1.0.105782\\
Version used:
\rowcolor{white}{\verb=2021-12-06T15:42:24Z=}\\
\end{longtable}

\begin{longtable}{|p{\textwidth * 1}|}
\hline
\rowcolor{gvm_log}{\color{white}{Log (CVSS: 0.0) }}\\
\rowcolor{gvm_log}{\color{white}{NVT: SSL/TLS: Collect and Report Certificate Details}}\\
\hline
\endfirsthead
\hfill\ldots continued from previous page \ldots \\
\hline
\endhead
\hline
\ldots continues on next page \ldots \\
\endfoot
\hline
\endlastfoot
\\
\textbf{Summary}\\
This script collects and reports the details of all SSL/TLS
  certificates.\\
  This data will be used by other tests to verify server certificates.\\

        \hline
        \\
\textbf{Vulnerability Detection Result}\\
\rowcolor{white}{\verb=The following certificate details of the remote service were collected.=}\\
\rowcolor{white}{\verb=Certificate details:=}\\
\rowcolor{white}{\verb=fingerprint (SHA-1)             | DA9C7CB3E91B933B84FD4B8A51B0CF8C35763F02=}\\
\rowcolor{white}{\verb=fingerprint (SHA-256)           | 61FF73A8F85051B08467B262445A409CB9877252E66D0B=}\\
\rowcolor{white}{$\hookrightarrow$\verb=88C9EBDB01108D824E=}\\
\rowcolor{white}{\verb=issued by                       | C=\verb-=-\verb=DE,L=\verb-=-\verb=Osnabrueck,O=\verb-=-\verb=GVM Users,OU=\verb-=-\verb=Certificate A=}\\
\rowcolor{white}{$\hookrightarrow$\verb=uthority for 48f7b451fe6e=}\\
\rowcolor{white}{\verb=public key algorithm            | RSA=}\\
\rowcolor{white}{\verb=public key size (bits)          | 3072=}\\
\rowcolor{white}{\verb=serial                          | 2E9E3F1D5036BD870DBD2CCEF0748CC40AA6EA47=}\\
\rowcolor{white}{\verb=signature algorithm             | sha256WithRSAEncryption=}\\
\rowcolor{white}{\verb=subject                         | C=\verb-=-\verb=DE,L=\verb-=-\verb=Osnabrueck,O=\verb-=-\verb=GVM Users,CN=\verb-=-\verb=48f7b451fe6e=}\\
\rowcolor{white}{\verb=subject alternative names (SAN) | None=}\\
\rowcolor{white}{\verb=valid from                      | 2022-03-22 09:55:11 UTC=}\\
\rowcolor{white}{\verb=valid until                     | 2024-03-21 09:55:11 UTC=}\\

          \hline
          \\
\textbf{Solution:}\\
\\


        \hline
        \\
\textbf{Log Method}\\
Details:
\rowcolor{white}{\verb=SSL/TLS: Collect and Report Certificate Details=}\\
OID:1.3.6.1.4.1.25623.1.0.103692\\
Version used:
\rowcolor{white}{\verb=2021-12-10T12:48:00Z=}\\
\end{longtable}

\begin{footnotesize}\hyperref[host:192.168.178.2]{[ return to 192.168.178.2 ]}\end{footnotesize}
\subsubsection{Log 8081/tcp}
\label{port:192.168.178.2 8081/tcp Log}

\begin{longtable}{|p{\textwidth * 1}|}
\hline
\rowcolor{gvm_log}{\color{white}{Log (CVSS: 0.0) }}\\
\rowcolor{gvm_log}{\color{white}{NVT: Services}}\\
\hline
\endfirsthead
\hfill\ldots continued from previous page \ldots \\
\hline
\endhead
\hline
\ldots continues on next page \ldots \\
\endfoot
\hline
\endlastfoot
\\
\textbf{Summary}\\
This routine attempts to guess which service is running on the
  remote ports. For instance, it searches for a web server which could listen on another port than
  80 or 443 and makes this information available for other check routines.\\

        \hline
        \\
\textbf{Vulnerability Detection Result}\\
\rowcolor{white}{\verb=A web server is running on this port=}\\

          \hline
          \\
\textbf{Solution:}\\
\\


        \hline
        \\
\textbf{Log Method}\\
Details:
\rowcolor{white}{\verb=Services=}\\
OID:1.3.6.1.4.1.25623.1.0.10330\\
Version used:
\rowcolor{white}{\verb=2021-03-15T10:42:03Z=}\\
\end{longtable}

\begin{footnotesize}\hyperref[host:192.168.178.2]{[ return to 192.168.178.2 ]}\end{footnotesize}
\subsubsection{Log 9390/tcp}
\label{port:192.168.178.2 9390/tcp Log}

\begin{longtable}{|p{\textwidth * 1}|}
\hline
\rowcolor{gvm_log}{\color{white}{Log (CVSS: 0.0) }}\\
\rowcolor{gvm_log}{\color{white}{NVT: Services}}\\
\hline
\endfirsthead
\hfill\ldots continued from previous page \ldots \\
\hline
\endhead
\hline
\ldots continues on next page \ldots \\
\endfoot
\hline
\endlastfoot
\\
\textbf{Summary}\\
This routine attempts to guess which service is running on the
  remote ports. For instance, it searches for a web server which could listen on another port than
  80 or 443 and makes this information available for other check routines.\\

        \hline
        \\
\textbf{Vulnerability Detection Result}\\
\rowcolor{white}{\verb=A TLScustom server answered on this port=}\\

          \hline
          \\
\textbf{Solution:}\\
\\


        \hline
        \\
\textbf{Log Method}\\
Details:
\rowcolor{white}{\verb=Services=}\\
OID:1.3.6.1.4.1.25623.1.0.10330\\
Version used:
\rowcolor{white}{\verb=2021-03-15T10:42:03Z=}\\
\end{longtable}

\begin{longtable}{|p{\textwidth * 1}|}
\hline
\rowcolor{gvm_log}{\color{white}{Log (CVSS: 0.0) }}\\
\rowcolor{gvm_log}{\color{white}{NVT: Service Detection with '<xml/>' Request}}\\
\hline
\endfirsthead
\hfill\ldots continued from previous page \ldots \\
\hline
\endhead
\hline
\ldots continues on next page \ldots \\
\endfoot
\hline
\endlastfoot
\\
\textbf{Summary}\\
This plugin performs service detection.\\
  This plugin is a complement of find\_service.nasl. It sends a '<xml/>'
  request to the remaining unknown services and tries to identify them.\\

        \hline
        \\
\textbf{Vulnerability Detection Result}\\
\rowcolor{white}{\verb=A OpenVAS / Greenbone Vulnerability Manager supporting the OMP/GMP protocol seem=}\\
\rowcolor{white}{$\hookrightarrow$\verb=s to be running on this port.=}\\

          \hline
          \\
\textbf{Solution:}\\
\\


        \hline
        \\
\textbf{Log Method}\\
Details:
\rowcolor{white}{\verb=Service Detection with '<xml/>' Request=}\\
OID:1.3.6.1.4.1.25623.1.0.108198\\
Version used:
\rowcolor{white}{\verb=2022-02-01T12:51:06Z=}\\
\end{longtable}

\begin{longtable}{|p{\textwidth * 1}|}
\hline
\rowcolor{gvm_log}{\color{white}{Log (CVSS: 0.0) }}\\
\rowcolor{gvm_log}{\color{white}{NVT: OpenVAS / Greenbone Vulnerability Manager Detection (OMP/GMP)}}\\
\hline
\endfirsthead
\hfill\ldots continued from previous page \ldots \\
\hline
\endhead
\hline
\ldots continues on next page \ldots \\
\endfoot
\hline
\endlastfoot
\\
\textbf{Summary}\\
OpenVAS Management Protocol (OMP) / Greenbone Management
  Protocol (GMP) based detection of an OpenVAS Manager (openvasmd) or Greebone Vulnerability Manager
  (gmvd).\\

        \hline
        \\
\textbf{Vulnerability Detection Result}\\
\rowcolor{white}{\verb=Detected Greenbone Vulnerability Manager=}\\
\rowcolor{white}{\verb=Version:       21.4=}\\
\rowcolor{white}{\verb=Location:      9390/tcp=}\\
\rowcolor{white}{\verb=CPE:           cpe:/a:greenbone:greenbone_vulnerability_manager:21.4=}\\
\rowcolor{white}{\verb=Concluded from version/product identification result:=}\\
\rowcolor{white}{\verb= - GMP protocol version request:  <get_version/>=}\\
\rowcolor{white}{\verb= - GMP protocol version response: <get_version_response status=\verb-=-\verb="200" status_text=}\\
\rowcolor{white}{$\hookrightarrow$\verb==\verb-=-\verb="OK"><version>21.4</version>=}\\

          \hline
          \\
\textbf{Solution:}\\
\\


        \hline
        \\
\textbf{Log Method}\\
Details:
\rowcolor{white}{\verb=OpenVAS / Greenbone Vulnerability Manager Detection (OMP/GMP)=}\\
OID:1.3.6.1.4.1.25623.1.0.103825\\
Version used:
\rowcolor{white}{\verb=2021-05-10T09:35:39Z=}\\
\end{longtable}

\begin{longtable}{|p{\textwidth * 1}|}
\hline
\rowcolor{gvm_log}{\color{white}{Log (CVSS: 0.0) }}\\
\rowcolor{gvm_log}{\color{white}{NVT: SSL/TLS: Version Detection}}\\
\hline
\endfirsthead
\hfill\ldots continued from previous page \ldots \\
\hline
\endhead
\hline
\ldots continues on next page \ldots \\
\endfoot
\hline
\endlastfoot
\\
\textbf{Summary}\\
Enumeration and reporting of SSL/TLS protocol versions supported
  by a remote service.\\

        \hline
        \\
\textbf{Vulnerability Detection Result}\\
\rowcolor{white}{\verb=The remote SSL/TLS service supports the following SSL/TLS protocol version(s):=}\\
\rowcolor{white}{\verb=TLSv1.2=}\\
\rowcolor{white}{\verb=TLSv1.3=}\\

          \hline
          \\
\textbf{Solution:}\\
\\


        \hline
        \\
\textbf{Log Method}\\
Sends multiple connection requests to the remote service and
  attempts to determine the SSL/TLS protocol versions supported by the service from the replies.\\
  Note: The supported SSL/TLS protocol versions included in the report of this VT are reported
  independently from the allowed / supported SSL/TLS ciphers.\\
Details:
\rowcolor{white}{\verb=SSL/TLS: Version Detection=}\\
OID:1.3.6.1.4.1.25623.1.0.105782\\
Version used:
\rowcolor{white}{\verb=2021-12-06T15:42:24Z=}\\
\end{longtable}

\begin{longtable}{|p{\textwidth * 1}|}
\hline
\rowcolor{gvm_log}{\color{white}{Log (CVSS: 0.0) }}\\
\rowcolor{gvm_log}{\color{white}{NVT: SSL/TLS: Collect and Report Certificate Details}}\\
\hline
\endfirsthead
\hfill\ldots continued from previous page \ldots \\
\hline
\endhead
\hline
\ldots continues on next page \ldots \\
\endfoot
\hline
\endlastfoot
\\
\textbf{Summary}\\
This script collects and reports the details of all SSL/TLS
  certificates.\\
  This data will be used by other tests to verify server certificates.\\

        \hline
        \\
\textbf{Vulnerability Detection Result}\\
\rowcolor{white}{\verb=The following certificate details of the remote service were collected.=}\\
\rowcolor{white}{\verb=Certificate details:=}\\
\rowcolor{white}{\verb=fingerprint (SHA-1)             | DA9C7CB3E91B933B84FD4B8A51B0CF8C35763F02=}\\
\rowcolor{white}{\verb=fingerprint (SHA-256)           | 61FF73A8F85051B08467B262445A409CB9877252E66D0B=}\\
\rowcolor{white}{$\hookrightarrow$\verb=88C9EBDB01108D824E=}\\
\rowcolor{white}{\verb=issued by                       | C=\verb-=-\verb=DE,L=\verb-=-\verb=Osnabrueck,O=\verb-=-\verb=GVM Users,OU=\verb-=-\verb=Certificate A=}\\
\rowcolor{white}{$\hookrightarrow$\verb=uthority for 48f7b451fe6e=}\\
\rowcolor{white}{\verb=public key algorithm            | RSA=}\\
\rowcolor{white}{\verb=public key size (bits)          | 3072=}\\
\rowcolor{white}{\verb=serial                          | 2E9E3F1D5036BD870DBD2CCEF0748CC40AA6EA47=}\\
\rowcolor{white}{\verb=signature algorithm             | sha256WithRSAEncryption=}\\
\rowcolor{white}{\verb=subject                         | C=\verb-=-\verb=DE,L=\verb-=-\verb=Osnabrueck,O=\verb-=-\verb=GVM Users,CN=\verb-=-\verb=48f7b451fe6e=}\\
\rowcolor{white}{\verb=subject alternative names (SAN) | None=}\\
\rowcolor{white}{\verb=valid from                      | 2022-03-22 09:55:11 UTC=}\\
\rowcolor{white}{\verb=valid until                     | 2024-03-21 09:55:11 UTC=}\\

          \hline
          \\
\textbf{Solution:}\\
\\


        \hline
        \\
\textbf{Log Method}\\
Details:
\rowcolor{white}{\verb=SSL/TLS: Collect and Report Certificate Details=}\\
OID:1.3.6.1.4.1.25623.1.0.103692\\
Version used:
\rowcolor{white}{\verb=2021-12-10T12:48:00Z=}\\
\end{longtable}

\begin{footnotesize}\hyperref[host:192.168.178.2]{[ return to 192.168.178.2 ]}\end{footnotesize}
\subsubsection{Log general/tcp}
\label{port:192.168.178.2 general/tcp Log}

\begin{longtable}{|p{\textwidth * 1}|}
\hline
\rowcolor{gvm_log}{\color{white}{Log (CVSS: 0.0) }}\\
\rowcolor{gvm_log}{\color{white}{NVT: SSL/TLS: Hostname discovery from server certificate}}\\
\hline
\endfirsthead
\hfill\ldots continued from previous page \ldots \\
\hline
\endhead
\hline
\ldots continues on next page \ldots \\
\endfoot
\hline
\endlastfoot
\\
\textbf{Summary}\\
It was possible to discover an additional hostname
  of this server from its certificate Common or Subject Alt Name.\\

        \hline
        \\
\textbf{Vulnerability Detection Result}\\
\rowcolor{white}{\verb=The following additional but not resolvable hostnames were detected:=}\\
\rowcolor{white}{\verb=48f7b451fe6e=}\\

          \hline
          \\
\textbf{Solution:}\\
\\


        \hline
        \\
\textbf{Log Method}\\
Details:
\rowcolor{white}{\verb=SSL/TLS: Hostname discovery from server certificate=}\\
OID:1.3.6.1.4.1.25623.1.0.111010\\
Version used:
\rowcolor{white}{\verb=2021-11-22T15:32:39Z=}\\
\end{longtable}

\begin{longtable}{|p{\textwidth * 1}|}
\hline
\rowcolor{gvm_log}{\color{white}{Log (CVSS: 0.0) }}\\
\rowcolor{gvm_log}{\color{white}{NVT: OS Detection Consolidation and Reporting}}\\
\hline
\endfirsthead
\hfill\ldots continued from previous page \ldots \\
\hline
\endhead
\hline
\ldots continues on next page \ldots \\
\endfoot
\hline
\endlastfoot
\\
\textbf{Summary}\\
This script consolidates the OS information detected by several
  VTs and tries to find the best matching OS.\\
  Furthermore it reports all previously collected information leading to this best matching OS. It
  also reports possible additional information which might help to improve the OS detection.\\
  If any of this information is wrong or could be improved please consider to report these to the
  referenced community portal.\\

        \hline
        \\
\textbf{Vulnerability Detection Result}\\
\rowcolor{white}{\verb=Best matching OS:=}\\
\rowcolor{white}{\verb=OS:           Debian GNU/Linux=}\\
\rowcolor{white}{\verb=CPE:          cpe:/o:debian:debian_linux=}\\
\rowcolor{white}{\verb=Found by NVT: 1.3.6.1.4.1.25623.1.0.105586 (Operating System (OS) Detection (SSH=}\\
\rowcolor{white}{$\hookrightarrow$\verb=))=}\\
\rowcolor{white}{\verb=Concluded from SSH banner on port 22/tcp: SSH-2.0-OpenSSH_8.4p1 Debian-5=}\\
\rowcolor{white}{\verb=Setting key "Host/runs_unixoide" based on this information=}\\
\rowcolor{white}{\verb=Other OS detections (in order of reliability):=}\\
\rowcolor{white}{\verb=OS:           Debian GNU/Linux=}\\
\rowcolor{white}{\verb=CPE:          cpe:/o:debian:debian_linux=}\\
\rowcolor{white}{\verb=Found by NVT: 1.3.6.1.4.1.25623.1.0.111068 (Operating System (OS) Detection (SMT=}\\
\rowcolor{white}{$\hookrightarrow$\verb=P/POP3/IMAP))=}\\
\rowcolor{white}{\verb=Concluded from SMTP banner on port 25/tcp: 220 localhost.fritz.box ESMTP Postfix=}\\
\rowcolor{white}{$\hookrightarrow$\verb= (Debian/GNU)=}\\
\rowcolor{white}{\verb=OS:           Linux/Unix=}\\
\rowcolor{white}{\verb=CPE:          cpe:/o:linux:kernel=}\\
\rowcolor{white}{\verb=Found by NVT: 1.3.6.1.4.1.25623.1.0.103825 (OpenVAS / Greenbone Vulnerability Ma=}\\
\rowcolor{white}{$\hookrightarrow$\verb=nager Detection (OMP/GMP))=}\\

          \hline
          \\
\textbf{Solution:}\\
\\


        \hline
        \\
\textbf{Log Method}\\
Details:
\rowcolor{white}{\verb=OS Detection Consolidation and Reporting=}\\
OID:1.3.6.1.4.1.25623.1.0.105937\\
Version used:
\rowcolor{white}{\verb=2022-04-05T09:27:51Z=}\\

      \hline
      \\
\textbf{References}\\
\rowcolor{white}{\verb=url: https://community.greenbone.net/c/vulnerability-tests=}\\
\end{longtable}

\begin{longtable}{|p{\textwidth * 1}|}
\hline
\rowcolor{gvm_log}{\color{white}{Log (CVSS: 0.0) }}\\
\rowcolor{gvm_log}{\color{white}{NVT: Hostname Determination Reporting}}\\
\hline
\endfirsthead
\hfill\ldots continued from previous page \ldots \\
\hline
\endhead
\hline
\ldots continues on next page \ldots \\
\endfoot
\hline
\endlastfoot
\\
\textbf{Summary}\\
The script reports information on how the hostname
  of the target was determined.\\

        \hline
        \\
\textbf{Vulnerability Detection Result}\\
\rowcolor{white}{\verb=Hostname determination for IP 192.168.178.2:=}\\
\rowcolor{white}{\verb=Hostname|Source=}\\
\rowcolor{white}{\verb=gvm.fritz.box|Reverse-DNS=}\\

          \hline
          \\
\textbf{Solution:}\\
\\


        \hline
        \\
\textbf{Log Method}\\
Details:
\rowcolor{white}{\verb=Hostname Determination Reporting=}\\
OID:1.3.6.1.4.1.25623.1.0.108449\\
Version used:
\rowcolor{white}{\verb=2018-11-19T11:11:31Z=}\\
\end{longtable}

\begin{footnotesize}\hyperref[host:192.168.178.2]{[ return to 192.168.178.2 ]}\end{footnotesize}
\subsubsection{Log 22/tcp}
\label{port:192.168.178.2 22/tcp Log}

\begin{longtable}{|p{\textwidth * 1}|}
\hline
\rowcolor{gvm_log}{\color{white}{Log (CVSS: 0.0) }}\\
\rowcolor{gvm_log}{\color{white}{NVT: Services}}\\
\hline
\endfirsthead
\hfill\ldots continued from previous page \ldots \\
\hline
\endhead
\hline
\ldots continues on next page \ldots \\
\endfoot
\hline
\endlastfoot
\\
\textbf{Summary}\\
This routine attempts to guess which service is running on the
  remote ports. For instance, it searches for a web server which could listen on another port than
  80 or 443 and makes this information available for other check routines.\\

        \hline
        \\
\textbf{Vulnerability Detection Result}\\
\rowcolor{white}{\verb=An ssh server is running on this port=}\\

          \hline
          \\
\textbf{Solution:}\\
\\


        \hline
        \\
\textbf{Log Method}\\
Details:
\rowcolor{white}{\verb=Services=}\\
OID:1.3.6.1.4.1.25623.1.0.10330\\
Version used:
\rowcolor{white}{\verb=2021-03-15T10:42:03Z=}\\
\end{longtable}

\begin{longtable}{|p{\textwidth * 1}|}
\hline
\rowcolor{gvm_log}{\color{white}{Log (CVSS: 0.0) }}\\
\rowcolor{gvm_log}{\color{white}{NVT: SSH Server type and version}}\\
\hline
\endfirsthead
\hfill\ldots continued from previous page \ldots \\
\hline
\endhead
\hline
\ldots continues on next page \ldots \\
\endfoot
\hline
\endlastfoot
\\
\textbf{Summary}\\
This detects the SSH Server's type and version by connecting to the server
  and processing the buffer received.\\
  This information gives potential attackers additional information about the system they are attacking.
  Versions and Types should be omitted where possible.\\

        \hline
        \\
\textbf{Vulnerability Detection Result}\\
\rowcolor{white}{\verb=Remote SSH server banner: SSH-2.0-OpenSSH_8.4p1 Debian-5=}\\
\rowcolor{white}{\verb=Remote SSH supported authentication: none,password,publickey,hostbased,keyboard-=}\\
\rowcolor{white}{$\hookrightarrow$\verb=interactive=}\\
\rowcolor{white}{\verb=Remote SSH text/login banner: (not available)=}\\
\rowcolor{white}{\verb=This is probably:=}\\
\rowcolor{white}{\verb=- OpenSSH=}\\
\rowcolor{white}{\verb=Concluded from remote connection attempt with credentials:=}\\
\rowcolor{white}{\verb=Login:    OpenVASVT=}\\
\rowcolor{white}{\verb=Password: OpenVASVT=}\\

          \hline
          \\
\textbf{Solution:}\\
\\


        \hline
        \\
\textbf{Log Method}\\
Details:
\rowcolor{white}{\verb=SSH Server type and version=}\\
OID:1.3.6.1.4.1.25623.1.0.10267\\
Version used:
\rowcolor{white}{\verb=2021-09-28T06:32:28Z=}\\
\end{longtable}

\begin{footnotesize}\hyperref[host:192.168.178.2]{[ return to 192.168.178.2 ]}\end{footnotesize}
\subsection{192.168.178.37}
\label{host:192.168.178.37}

\begin{tabular}{ll}
Host scan start&Fri Apr 22 12:21:26 2022 UTC\\
Host scan end&Fri Apr 22 12:41:39 2022 UTC\\
\end{tabular}

\begin{longtable}{|l|l|}
\hline
\rowcolor{gvm_report}Service (Port)&Threat Level\\
\hline
\endfirsthead
\multicolumn{2}{l}{\hfill\ldots (continued) \ldots}\\
\hline
\rowcolor{gvm_report}Service (Port)&Threat Level\\
\hline
\endhead
\hline
\multicolumn{2}{l}{\ldots (continues) \ldots}\\
\endfoot
\hline
\endlastfoot
\hline
\hyperref[port:192.168.178.37 139/tcp Log]{139/tcp}&Log\\
\hline
\hyperref[port:192.168.178.37 80/tcp Log]{80/tcp}&Log\\
\hline
\hyperref[port:192.168.178.37 445/tcp Log]{445/tcp}&Log\\
\hline
\hyperref[port:192.168.178.37 8001/tcp Log]{8001/tcp}&Log\\
\hline
\hyperref[port:192.168.178.37 23/tcp Log]{23/tcp}&Log\\
\hline
\hyperref[port:192.168.178.37 general/tcp Log]{general/tcp}&Log\\
\hline
\hyperref[port:192.168.178.37 443/tcp Log]{443/tcp}&Log\\
\hline
\hyperref[port:192.168.178.37 22/tcp Log]{22/tcp}&Log\\
\hline
\hyperref[port:192.168.178.37 21/tcp Log]{21/tcp}&Log\\
\hline
\end{longtable}


%\subsection*{Security Issues and Fixes -- 192.168.178.37}

\subsubsection{Log 139/tcp}
\label{port:192.168.178.37 139/tcp Log}

\begin{longtable}{|p{\textwidth * 1}|}
\hline
\rowcolor{gvm_log}{\color{white}{Log (CVSS: 0.0) }}\\
\rowcolor{gvm_log}{\color{white}{NVT: SMB/CIFS Server Detection}}\\
\hline
\endfirsthead
\hfill\ldots continued from previous page \ldots \\
\hline
\endhead
\hline
\ldots continues on next page \ldots \\
\endfoot
\hline
\endlastfoot
\\
\textbf{Summary}\\
This script detects whether port 445 and 139 are open and
  if they are running a CIFS/SMB server.\\

        \hline
        \\
\textbf{Vulnerability Detection Result}\\
\rowcolor{white}{\verb=A SMB server is running on this port=}\\

          \hline
          \\
\textbf{Solution:}\\
\\


        \hline
        \\
\textbf{Log Method}\\
Details:
\rowcolor{white}{\verb=SMB/CIFS Server Detection=}\\
OID:1.3.6.1.4.1.25623.1.0.11011\\
Version used:
\rowcolor{white}{\verb=2020-11-10T15:30:28Z=}\\
\end{longtable}

\begin{footnotesize}\hyperref[host:192.168.178.37]{[ return to 192.168.178.37 ]}\end{footnotesize}
\subsubsection{Log 80/tcp}
\label{port:192.168.178.37 80/tcp Log}

\begin{longtable}{|p{\textwidth * 1}|}
\hline
\rowcolor{gvm_log}{\color{white}{Log (CVSS: 0.0) }}\\
\rowcolor{gvm_log}{\color{white}{NVT: Services}}\\
\hline
\endfirsthead
\hfill\ldots continued from previous page \ldots \\
\hline
\endhead
\hline
\ldots continues on next page \ldots \\
\endfoot
\hline
\endlastfoot
\\
\textbf{Summary}\\
This routine attempts to guess which service is running on the
  remote ports. For instance, it searches for a web server which could listen on another port than
  80 or 443 and makes this information available for other check routines.\\

        \hline
        \\
\textbf{Vulnerability Detection Result}\\
\rowcolor{white}{\verb=A web server is running on this port=}\\

          \hline
          \\
\textbf{Solution:}\\
\\


        \hline
        \\
\textbf{Log Method}\\
Details:
\rowcolor{white}{\verb=Services=}\\
OID:1.3.6.1.4.1.25623.1.0.10330\\
Version used:
\rowcolor{white}{\verb=2021-03-15T10:42:03Z=}\\
\end{longtable}

\begin{footnotesize}\hyperref[host:192.168.178.37]{[ return to 192.168.178.37 ]}\end{footnotesize}
\subsubsection{Log 445/tcp}
\label{port:192.168.178.37 445/tcp Log}

\begin{longtable}{|p{\textwidth * 1}|}
\hline
\rowcolor{gvm_log}{\color{white}{Log (CVSS: 0.0) }}\\
\rowcolor{gvm_log}{\color{white}{NVT: SMB/CIFS Server Detection}}\\
\hline
\endfirsthead
\hfill\ldots continued from previous page \ldots \\
\hline
\endhead
\hline
\ldots continues on next page \ldots \\
\endfoot
\hline
\endlastfoot
\\
\textbf{Summary}\\
This script detects whether port 445 and 139 are open and
  if they are running a CIFS/SMB server.\\

        \hline
        \\
\textbf{Vulnerability Detection Result}\\
\rowcolor{white}{\verb=A CIFS server is running on this port=}\\

          \hline
          \\
\textbf{Solution:}\\
\\


        \hline
        \\
\textbf{Log Method}\\
Details:
\rowcolor{white}{\verb=SMB/CIFS Server Detection=}\\
OID:1.3.6.1.4.1.25623.1.0.11011\\
Version used:
\rowcolor{white}{\verb=2020-11-10T15:30:28Z=}\\
\end{longtable}

\begin{longtable}{|p{\textwidth * 1}|}
\hline
\rowcolor{gvm_log}{\color{white}{Log (CVSS: 0.0) }}\\
\rowcolor{gvm_log}{\color{white}{NVT: SMB NativeLanMan}}\\
\hline
\endfirsthead
\hfill\ldots continued from previous page \ldots \\
\hline
\endhead
\hline
\ldots continues on next page \ldots \\
\endfoot
\hline
\endlastfoot
\\
\textbf{Summary}\\
It is possible to extract OS, domain and SMB server information
  from the Session Setup AndX Response packet which is generated during NTLM authentication.\\

        \hline
        \\
\textbf{Vulnerability Detection Result}\\
\rowcolor{white}{\verb=Detected Samba=}\\
\rowcolor{white}{\verb=Version:       4.6.2=}\\
\rowcolor{white}{\verb=Location:      445/tcp=}\\
\rowcolor{white}{\verb=CPE:           cpe:/a:samba:samba:4.6.2=}\\
\rowcolor{white}{\verb=Concluded from version/product identification result:=}\\
\rowcolor{white}{\verb=Samba 4.6.2=}\\
\rowcolor{white}{\verb=Extra information:=}\\
\rowcolor{white}{\verb=Detected SMB workgroup: DREAMBOX=}\\
\rowcolor{white}{\verb=Detected SMB server: Samba 4.6.2=}\\

          \hline
          \\
\textbf{Solution:}\\
\\


        \hline
        \\
\textbf{Log Method}\\
Details:
\rowcolor{white}{\verb=SMB NativeLanMan=}\\
OID:1.3.6.1.4.1.25623.1.0.102011\\
Version used:
\rowcolor{white}{\verb=2021-09-06T06:22:50Z=}\\
\end{longtable}

\begin{longtable}{|p{\textwidth * 1}|}
\hline
\rowcolor{gvm_log}{\color{white}{Log (CVSS: 0.0) }}\\
\rowcolor{gvm_log}{\color{white}{NVT: SMB NativeLanMan}}\\
\hline
\endfirsthead
\hfill\ldots continued from previous page \ldots \\
\hline
\endhead
\hline
\ldots continues on next page \ldots \\
\endfoot
\hline
\endlastfoot
\\
\textbf{Summary}\\
It is possible to extract OS, domain and SMB server information
  from the Session Setup AndX Response packet which is generated during NTLM authentication.\\

        \hline
        \\
\textbf{Vulnerability Detection Result}\\
\rowcolor{white}{\verb=Detected SMB workgroup: DREAMBOX=}\\
\rowcolor{white}{\verb=Detected SMB server: Samba 4.6.2=}\\
\rowcolor{white}{\verb=Detected OS: Linux/Unix=}\\

          \hline
          \\
\textbf{Solution:}\\
\\


        \hline
        \\
\textbf{Log Method}\\
Details:
\rowcolor{white}{\verb=SMB NativeLanMan=}\\
OID:1.3.6.1.4.1.25623.1.0.102011\\
Version used:
\rowcolor{white}{\verb=2021-09-06T06:22:50Z=}\\
\end{longtable}

\begin{longtable}{|p{\textwidth * 1}|}
\hline
\rowcolor{gvm_log}{\color{white}{Log (CVSS: 0.0) }}\\
\rowcolor{gvm_log}{\color{white}{NVT: SMB log in}}\\
\hline
\endfirsthead
\hfill\ldots continued from previous page \ldots \\
\hline
\endhead
\hline
\ldots continues on next page \ldots \\
\endfoot
\hline
\endlastfoot
\\
\textbf{Summary}\\
This script attempts to logon into the remote host using
  login/password credentials.\\

        \hline
        \\
\textbf{Vulnerability Detection Result}\\
\rowcolor{white}{\verb=It was possible to log into the remote host using the SMB protocol.=}\\

          \hline
          \\
\textbf{Solution:}\\
\\


        \hline
        \\
\textbf{Log Method}\\
Details:
\rowcolor{white}{\verb=SMB log in=}\\
OID:1.3.6.1.4.1.25623.1.0.10394\\
Version used:
\rowcolor{white}{\verb=2021-08-11T09:39:10Z=}\\
\end{longtable}

\begin{footnotesize}\hyperref[host:192.168.178.37]{[ return to 192.168.178.37 ]}\end{footnotesize}
\subsubsection{Log 8001/tcp}
\label{port:192.168.178.37 8001/tcp Log}

\begin{longtable}{|p{\textwidth * 1}|}
\hline
\rowcolor{gvm_log}{\color{white}{Log (CVSS: 0.0) }}\\
\rowcolor{gvm_log}{\color{white}{NVT: Services}}\\
\hline
\endfirsthead
\hfill\ldots continued from previous page \ldots \\
\hline
\endhead
\hline
\ldots continues on next page \ldots \\
\endfoot
\hline
\endlastfoot
\\
\textbf{Summary}\\
This routine attempts to guess which service is running on the
  remote ports. For instance, it searches for a web server which could listen on another port than
  80 or 443 and makes this information available for other check routines.\\

        \hline
        \\
\textbf{Vulnerability Detection Result}\\
\rowcolor{white}{\verb=A web server is running on this port=}\\

          \hline
          \\
\textbf{Solution:}\\
\\


        \hline
        \\
\textbf{Log Method}\\
Details:
\rowcolor{white}{\verb=Services=}\\
OID:1.3.6.1.4.1.25623.1.0.10330\\
Version used:
\rowcolor{white}{\verb=2021-03-15T10:42:03Z=}\\
\end{longtable}

\begin{footnotesize}\hyperref[host:192.168.178.37]{[ return to 192.168.178.37 ]}\end{footnotesize}
\subsubsection{Log 23/tcp}
\label{port:192.168.178.37 23/tcp Log}

\begin{longtable}{|p{\textwidth * 1}|}
\hline
\rowcolor{gvm_log}{\color{white}{Log (CVSS: 0.0) }}\\
\rowcolor{gvm_log}{\color{white}{NVT: Services}}\\
\hline
\endfirsthead
\hfill\ldots continued from previous page \ldots \\
\hline
\endhead
\hline
\ldots continues on next page \ldots \\
\endfoot
\hline
\endlastfoot
\\
\textbf{Summary}\\
This routine attempts to guess which service is running on the
  remote ports. For instance, it searches for a web server which could listen on another port than
  80 or 443 and makes this information available for other check routines.\\

        \hline
        \\
\textbf{Vulnerability Detection Result}\\
\rowcolor{white}{\verb=A telnet server seems to be running on this port=}\\

          \hline
          \\
\textbf{Solution:}\\
\\


        \hline
        \\
\textbf{Log Method}\\
Details:
\rowcolor{white}{\verb=Services=}\\
OID:1.3.6.1.4.1.25623.1.0.10330\\
Version used:
\rowcolor{white}{\verb=2021-03-15T10:42:03Z=}\\
\end{longtable}

\begin{longtable}{|p{\textwidth * 1}|}
\hline
\rowcolor{gvm_log}{\color{white}{Log (CVSS: 0.0) }}\\
\rowcolor{gvm_log}{\color{white}{NVT: Telnet Service Detection}}\\
\hline
\endfirsthead
\hfill\ldots continued from previous page \ldots \\
\hline
\endhead
\hline
\ldots continues on next page \ldots \\
\endfoot
\hline
\endlastfoot
\\
\textbf{Summary}\\
This scripts tries to detect a Telnet service running
  at the remote host.\\

        \hline
        \\
\textbf{Vulnerability Detection Result}\\
\rowcolor{white}{\verb=A Telnet server seems to be running on this port=}\\

          \hline
          \\
\textbf{Solution:}\\
\\


        \hline
        \\
\textbf{Log Method}\\
Details:
\rowcolor{white}{\verb=Telnet Service Detection=}\\
OID:1.3.6.1.4.1.25623.1.0.100074\\
Version used:
\rowcolor{white}{\verb=2020-11-10T15:30:28Z=}\\

      \hline
      \\
\textbf{References}\\
\rowcolor{white}{\verb=url: https://tools.ietf.org/html/rfc854=}\\
\end{longtable}

\begin{longtable}{|p{\textwidth * 1}|}
\hline
\rowcolor{gvm_log}{\color{white}{Log (CVSS: 0.0) }}\\
\rowcolor{gvm_log}{\color{white}{NVT: Telnet Banner Reporting}}\\
\hline
\endfirsthead
\hfill\ldots continued from previous page \ldots \\
\hline
\endhead
\hline
\ldots continues on next page \ldots \\
\endfoot
\hline
\endlastfoot
\\
\textbf{Summary}\\
This scripts reports the received banner of a Telnet service.\\

        \hline
        \\
\textbf{Vulnerability Detection Result}\\
\rowcolor{white}{\verb=Remote Telnet banner:=}\\
\rowcolor{white}{\verb=opendreambox 2.6.0 dreambox=}\\
\rowcolor{white}{\verb=dreambox login: =}\\

          \hline
          \\
\textbf{Solution:}\\
\\


        \hline
        \\
\textbf{Log Method}\\
Details:
\rowcolor{white}{\verb=Telnet Banner Reporting=}\\
OID:1.3.6.1.4.1.25623.1.0.10281\\
Version used:
\rowcolor{white}{\verb=2022-02-11T08:39:43Z=}\\
\end{longtable}

\begin{footnotesize}\hyperref[host:192.168.178.37]{[ return to 192.168.178.37 ]}\end{footnotesize}
\subsubsection{Log general/tcp}
\label{port:192.168.178.37 general/tcp Log}

\begin{longtable}{|p{\textwidth * 1}|}
\hline
\rowcolor{gvm_log}{\color{white}{Log (CVSS: 0.0) }}\\
\rowcolor{gvm_log}{\color{white}{NVT: Dropbear Detection Consolidation}}\\
\hline
\endfirsthead
\hfill\ldots continued from previous page \ldots \\
\hline
\endhead
\hline
\ldots continues on next page \ldots \\
\endfoot
\hline
\endlastfoot
\\
\textbf{Summary}\\
Consolidation of Dropbear detections.\\

        \hline
        \\
\textbf{Vulnerability Detection Result}\\
\rowcolor{white}{\verb=Detected Dropbear SSH=}\\
\rowcolor{white}{\verb=Version:       2016.74=}\\
\rowcolor{white}{\verb=Location:      22/tcp=}\\
\rowcolor{white}{\verb=CPE:           cpe:/a:dropbear_ssh_project:dropbear_ssh:2016.74=}\\
\rowcolor{white}{\verb=Concluded from version/product identification result:=}\\
\rowcolor{white}{\verb=SSH-2.0-dropbear_2016.74=}\\

          \hline
          \\
\textbf{Solution:}\\
\\


        \hline
        \\
\textbf{Log Method}\\
Details:
\rowcolor{white}{\verb=Dropbear Detection Consolidation=}\\
OID:1.3.6.1.4.1.25623.1.0.112869\\
Version used:
\rowcolor{white}{\verb=2021-11-10T06:44:11Z=}\\

      \hline
      \\
\textbf{References}\\
\rowcolor{white}{\verb=url: https://matt.ucc.asn.au/dropbear/dropbear.html=}\\
\end{longtable}

\begin{longtable}{|p{\textwidth * 1}|}
\hline
\rowcolor{gvm_log}{\color{white}{Log (CVSS: 0.0) }}\\
\rowcolor{gvm_log}{\color{white}{NVT: SSL/TLS: Hostname discovery from server certificate}}\\
\hline
\endfirsthead
\hfill\ldots continued from previous page \ldots \\
\hline
\endhead
\hline
\ldots continues on next page \ldots \\
\endfoot
\hline
\endlastfoot
\\
\textbf{Summary}\\
It was possible to discover an additional hostname
  of this server from its certificate Common or Subject Alt Name.\\

        \hline
        \\
\textbf{Vulnerability Detection Result}\\
\rowcolor{white}{\verb=The following additional but not resolvable hostnames were detected:=}\\
\rowcolor{white}{\verb=dreamone=}\\

          \hline
          \\
\textbf{Solution:}\\
\\


        \hline
        \\
\textbf{Log Method}\\
Details:
\rowcolor{white}{\verb=SSL/TLS: Hostname discovery from server certificate=}\\
OID:1.3.6.1.4.1.25623.1.0.111010\\
Version used:
\rowcolor{white}{\verb=2021-11-22T15:32:39Z=}\\
\end{longtable}

\begin{longtable}{|p{\textwidth * 1}|}
\hline
\rowcolor{gvm_log}{\color{white}{Log (CVSS: 0.0) }}\\
\rowcolor{gvm_log}{\color{white}{NVT: OS Detection Consolidation and Reporting}}\\
\hline
\endfirsthead
\hfill\ldots continued from previous page \ldots \\
\hline
\endhead
\hline
\ldots continues on next page \ldots \\
\endfoot
\hline
\endlastfoot
\\
\textbf{Summary}\\
This script consolidates the OS information detected by several
  VTs and tries to find the best matching OS.\\
  Furthermore it reports all previously collected information leading to this best matching OS. It
  also reports possible additional information which might help to improve the OS detection.\\
  If any of this information is wrong or could be improved please consider to report these to the
  referenced community portal.\\

        \hline
        \\
\textbf{Vulnerability Detection Result}\\
\rowcolor{white}{\verb=Best matching OS:=}\\
\rowcolor{white}{\verb=OS:           Linux/Unix=}\\
\rowcolor{white}{\verb=CPE:          cpe:/o:linux:kernel=}\\
\rowcolor{white}{\verb=Found by NVT: 1.3.6.1.4.1.25623.1.0.102011 (SMB NativeLanMan)=}\\
\rowcolor{white}{\verb=Concluded from SMB/Samba banner on port 445/tcp: =}\\
\rowcolor{white}{\verb=OS String:  Windows 6.1=}\\
\rowcolor{white}{\verb=SMB String: Samba 4.6.2=}\\
\rowcolor{white}{\verb=Note: The service is running on a Linux/Unix based OS but reporting itself with =}\\
\rowcolor{white}{$\hookrightarrow$\verb=an Windows related OS string.=}\\
\rowcolor{white}{\verb=Setting key "Host/runs_unixoide" based on this information=}\\
\rowcolor{white}{\verb=Other OS detections (in order of reliability):=}\\
\rowcolor{white}{\verb=OS:           Linux/Unix=}\\
\rowcolor{white}{\verb=CPE:          cpe:/o:linux:kernel=}\\
\rowcolor{white}{\verb=Found by NVT: 1.3.6.1.4.1.25623.1.0.112869 (Dropbear Detection Consolidation)=}\\

          \hline
          \\
\textbf{Solution:}\\
\\


        \hline
        \\
\textbf{Log Method}\\
Details:
\rowcolor{white}{\verb=OS Detection Consolidation and Reporting=}\\
OID:1.3.6.1.4.1.25623.1.0.105937\\
Version used:
\rowcolor{white}{\verb=2022-04-05T09:27:51Z=}\\

      \hline
      \\
\textbf{References}\\
\rowcolor{white}{\verb=url: https://community.greenbone.net/c/vulnerability-tests=}\\
\end{longtable}

\begin{longtable}{|p{\textwidth * 1}|}
\hline
\rowcolor{gvm_log}{\color{white}{Log (CVSS: 0.0) }}\\
\rowcolor{gvm_log}{\color{white}{NVT: Hostname Determination Reporting}}\\
\hline
\endfirsthead
\hfill\ldots continued from previous page \ldots \\
\hline
\endhead
\hline
\ldots continues on next page \ldots \\
\endfoot
\hline
\endlastfoot
\\
\textbf{Summary}\\
The script reports information on how the hostname
  of the target was determined.\\

        \hline
        \\
\textbf{Vulnerability Detection Result}\\
\rowcolor{white}{\verb=Hostname determination for IP 192.168.178.37:=}\\
\rowcolor{white}{\verb=Hostname|Source=}\\
\rowcolor{white}{\verb=dreambox.fritz.box|Reverse-DNS=}\\

          \hline
          \\
\textbf{Solution:}\\
\\


        \hline
        \\
\textbf{Log Method}\\
Details:
\rowcolor{white}{\verb=Hostname Determination Reporting=}\\
OID:1.3.6.1.4.1.25623.1.0.108449\\
Version used:
\rowcolor{white}{\verb=2018-11-19T11:11:31Z=}\\
\end{longtable}

\begin{footnotesize}\hyperref[host:192.168.178.37]{[ return to 192.168.178.37 ]}\end{footnotesize}
\subsubsection{Log 443/tcp}
\label{port:192.168.178.37 443/tcp Log}

\begin{longtable}{|p{\textwidth * 1}|}
\hline
\rowcolor{gvm_log}{\color{white}{Log (CVSS: 0.0) }}\\
\rowcolor{gvm_log}{\color{white}{NVT: Services}}\\
\hline
\endfirsthead
\hfill\ldots continued from previous page \ldots \\
\hline
\endhead
\hline
\ldots continues on next page \ldots \\
\endfoot
\hline
\endlastfoot
\\
\textbf{Summary}\\
This routine attempts to guess which service is running on the
  remote ports. For instance, it searches for a web server which could listen on another port than
  80 or 443 and makes this information available for other check routines.\\

        \hline
        \\
\textbf{Vulnerability Detection Result}\\
\rowcolor{white}{\verb=A TLScustom server answered on this port=}\\

          \hline
          \\
\textbf{Solution:}\\
\\


        \hline
        \\
\textbf{Log Method}\\
Details:
\rowcolor{white}{\verb=Services=}\\
OID:1.3.6.1.4.1.25623.1.0.10330\\
Version used:
\rowcolor{white}{\verb=2021-03-15T10:42:03Z=}\\
\end{longtable}

\begin{longtable}{|p{\textwidth * 1}|}
\hline
\rowcolor{gvm_log}{\color{white}{Log (CVSS: 0.0) }}\\
\rowcolor{gvm_log}{\color{white}{NVT: Services}}\\
\hline
\endfirsthead
\hfill\ldots continued from previous page \ldots \\
\hline
\endhead
\hline
\ldots continues on next page \ldots \\
\endfoot
\hline
\endlastfoot
\\
\textbf{Summary}\\
This routine attempts to guess which service is running on the
  remote ports. For instance, it searches for a web server which could listen on another port than
  80 or 443 and makes this information available for other check routines.\\

        \hline
        \\
\textbf{Vulnerability Detection Result}\\
\rowcolor{white}{\verb=A web server is running on this port through SSL=}\\

          \hline
          \\
\textbf{Solution:}\\
\\


        \hline
        \\
\textbf{Log Method}\\
Details:
\rowcolor{white}{\verb=Services=}\\
OID:1.3.6.1.4.1.25623.1.0.10330\\
Version used:
\rowcolor{white}{\verb=2021-03-15T10:42:03Z=}\\
\end{longtable}

\begin{longtable}{|p{\textwidth * 1}|}
\hline
\rowcolor{gvm_log}{\color{white}{Log (CVSS: 0.0) }}\\
\rowcolor{gvm_log}{\color{white}{NVT: SSL/TLS: Version Detection}}\\
\hline
\endfirsthead
\hfill\ldots continued from previous page \ldots \\
\hline
\endhead
\hline
\ldots continues on next page \ldots \\
\endfoot
\hline
\endlastfoot
\\
\textbf{Summary}\\
Enumeration and reporting of SSL/TLS protocol versions supported
  by a remote service.\\

        \hline
        \\
\textbf{Vulnerability Detection Result}\\
\rowcolor{white}{\verb=The remote SSL/TLS service supports the following SSL/TLS protocol version(s):=}\\
\rowcolor{white}{\verb=TLSv1.0=}\\
\rowcolor{white}{\verb=TLSv1.1=}\\
\rowcolor{white}{\verb=TLSv1.2=}\\

          \hline
          \\
\textbf{Solution:}\\
\\


        \hline
        \\
\textbf{Log Method}\\
Sends multiple connection requests to the remote service and
  attempts to determine the SSL/TLS protocol versions supported by the service from the replies.\\
  Note: The supported SSL/TLS protocol versions included in the report of this VT are reported
  independently from the allowed / supported SSL/TLS ciphers.\\
Details:
\rowcolor{white}{\verb=SSL/TLS: Version Detection=}\\
OID:1.3.6.1.4.1.25623.1.0.105782\\
Version used:
\rowcolor{white}{\verb=2021-12-06T15:42:24Z=}\\
\end{longtable}

\begin{longtable}{|p{\textwidth * 1}|}
\hline
\rowcolor{gvm_log}{\color{white}{Log (CVSS: 0.0) }}\\
\rowcolor{gvm_log}{\color{white}{NVT: SSL/TLS: Collect and Report Certificate Details}}\\
\hline
\endfirsthead
\hfill\ldots continued from previous page \ldots \\
\hline
\endhead
\hline
\ldots continues on next page \ldots \\
\endfoot
\hline
\endlastfoot
\\
\textbf{Summary}\\
This script collects and reports the details of all SSL/TLS
  certificates.\\
  This data will be used by other tests to verify server certificates.\\

        \hline
        \\
\textbf{Vulnerability Detection Result}\\
\rowcolor{white}{\verb=The following certificate details of the remote service were collected.=}\\
\rowcolor{white}{\verb=Certificate details:=}\\
\rowcolor{white}{\verb=fingerprint (SHA-1)             | 3226C9FA80F051BF085C95D6CD2E9ABA43678A13=}\\
\rowcolor{white}{\verb=fingerprint (SHA-256)           | 5D729D2F60BF2E7DDFF4017044F790DD8D059211B05C9D=}\\
\rowcolor{white}{$\hookrightarrow$\verb=9D30CF6E6BEC791DB3=}\\
\rowcolor{white}{\verb=issued by                       | CN=\verb-=-\verb=dreamone,OU=\verb-=-\verb=STB,O=\verb-=-\verb=Dreambox,L=\verb-=-\verb=Home,ST=\verb-=-\verb=Home,C=}\\
\rowcolor{white}{$\hookrightarrow$\verb==\verb-=-\verb=DE=}\\
\rowcolor{white}{\verb=public key algorithm            | RSA=}\\
\rowcolor{white}{\verb=public key size (bits)          | 2048=}\\
\rowcolor{white}{\verb=serial                          | 2B95AD25=}\\
\rowcolor{white}{\verb=signature algorithm             | sha256WithRSAEncryption=}\\
\rowcolor{white}{\verb=subject                         | CN=\verb-=-\verb=dreamone,OU=\verb-=-\verb=STB,O=\verb-=-\verb=Dreambox,L=\verb-=-\verb=Home,ST=\verb-=-\verb=Home,C=}\\
\rowcolor{white}{$\hookrightarrow$\verb==\verb-=-\verb=DE=}\\
\rowcolor{white}{\verb=subject alternative names (SAN) | None=}\\
\rowcolor{white}{\verb=valid from                      | 2012-01-01 00:00:00 UTC=}\\
\rowcolor{white}{\verb=valid until                     | 2030-12-31 23:59:00 UTC=}\\

          \hline
          \\
\textbf{Solution:}\\
\\


        \hline
        \\
\textbf{Log Method}\\
Details:
\rowcolor{white}{\verb=SSL/TLS: Collect and Report Certificate Details=}\\
OID:1.3.6.1.4.1.25623.1.0.103692\\
Version used:
\rowcolor{white}{\verb=2021-12-10T12:48:00Z=}\\
\end{longtable}

\begin{footnotesize}\hyperref[host:192.168.178.37]{[ return to 192.168.178.37 ]}\end{footnotesize}
\subsubsection{Log 22/tcp}
\label{port:192.168.178.37 22/tcp Log}

\begin{longtable}{|p{\textwidth * 1}|}
\hline
\rowcolor{gvm_log}{\color{white}{Log (CVSS: 0.0) }}\\
\rowcolor{gvm_log}{\color{white}{NVT: Services}}\\
\hline
\endfirsthead
\hfill\ldots continued from previous page \ldots \\
\hline
\endhead
\hline
\ldots continues on next page \ldots \\
\endfoot
\hline
\endlastfoot
\\
\textbf{Summary}\\
This routine attempts to guess which service is running on the
  remote ports. For instance, it searches for a web server which could listen on another port than
  80 or 443 and makes this information available for other check routines.\\

        \hline
        \\
\textbf{Vulnerability Detection Result}\\
\rowcolor{white}{\verb=An ssh server is running on this port=}\\

          \hline
          \\
\textbf{Solution:}\\
\\


        \hline
        \\
\textbf{Log Method}\\
Details:
\rowcolor{white}{\verb=Services=}\\
OID:1.3.6.1.4.1.25623.1.0.10330\\
Version used:
\rowcolor{white}{\verb=2021-03-15T10:42:03Z=}\\
\end{longtable}

\begin{longtable}{|p{\textwidth * 1}|}
\hline
\rowcolor{gvm_log}{\color{white}{Log (CVSS: 0.0) }}\\
\rowcolor{gvm_log}{\color{white}{NVT: SSH Server type and version}}\\
\hline
\endfirsthead
\hfill\ldots continued from previous page \ldots \\
\hline
\endhead
\hline
\ldots continues on next page \ldots \\
\endfoot
\hline
\endlastfoot
\\
\textbf{Summary}\\
This detects the SSH Server's type and version by connecting to the server
  and processing the buffer received.\\
  This information gives potential attackers additional information about the system they are attacking.
  Versions and Types should be omitted where possible.\\

        \hline
        \\
\textbf{Vulnerability Detection Result}\\
\rowcolor{white}{\verb=Remote SSH server banner: SSH-2.0-dropbear_2016.74=}\\
\rowcolor{white}{\verb=Remote SSH supported authentication: password,publickey=}\\
\rowcolor{white}{\verb=Remote SSH text/login banner: (not available)=}\\
\rowcolor{white}{\verb=This is probably:=}\\
\rowcolor{white}{\verb=- Dropbear SSH=}\\
\rowcolor{white}{\verb=Concluded from remote connection attempt with credentials:=}\\
\rowcolor{white}{\verb=Login:    OpenVASVT=}\\
\rowcolor{white}{\verb=Password: OpenVASVT=}\\

          \hline
          \\
\textbf{Solution:}\\
\\


        \hline
        \\
\textbf{Log Method}\\
Details:
\rowcolor{white}{\verb=SSH Server type and version=}\\
OID:1.3.6.1.4.1.25623.1.0.10267\\
Version used:
\rowcolor{white}{\verb=2021-09-28T06:32:28Z=}\\
\end{longtable}

\begin{footnotesize}\hyperref[host:192.168.178.37]{[ return to 192.168.178.37 ]}\end{footnotesize}
\subsubsection{Log 21/tcp}
\label{port:192.168.178.37 21/tcp Log}

\begin{longtable}{|p{\textwidth * 1}|}
\hline
\rowcolor{gvm_log}{\color{white}{Log (CVSS: 0.0) }}\\
\rowcolor{gvm_log}{\color{white}{NVT: Services}}\\
\hline
\endfirsthead
\hfill\ldots continued from previous page \ldots \\
\hline
\endhead
\hline
\ldots continues on next page \ldots \\
\endfoot
\hline
\endlastfoot
\\
\textbf{Summary}\\
This routine attempts to guess which service is running on the
  remote ports. For instance, it searches for a web server which could listen on another port than
  80 or 443 and makes this information available for other check routines.\\

        \hline
        \\
\textbf{Vulnerability Detection Result}\\
\rowcolor{white}{\verb=An FTP server is running on this port.=}\\
\rowcolor{white}{\verb=Here is its banner : =}\\
\rowcolor{white}{\verb=220 Welcome to the opendreambox FTP service!=}\\

          \hline
          \\
\textbf{Solution:}\\
\\


        \hline
        \\
\textbf{Log Method}\\
Details:
\rowcolor{white}{\verb=Services=}\\
OID:1.3.6.1.4.1.25623.1.0.10330\\
Version used:
\rowcolor{white}{\verb=2021-03-15T10:42:03Z=}\\
\end{longtable}

\begin{longtable}{|p{\textwidth * 1}|}
\hline
\rowcolor{gvm_log}{\color{white}{Log (CVSS: 0.0) }}\\
\rowcolor{gvm_log}{\color{white}{NVT: FTP Banner Detection}}\\
\hline
\endfirsthead
\hfill\ldots continued from previous page \ldots \\
\hline
\endhead
\hline
\ldots continues on next page \ldots \\
\endfoot
\hline
\endlastfoot
\\
\textbf{Summary}\\
This Plugin detects and reports a FTP Server Banner.\\

        \hline
        \\
\textbf{Vulnerability Detection Result}\\
\rowcolor{white}{\verb=Remote FTP server banner:=}\\
\rowcolor{white}{\verb=220 Welcome to the opendreambox FTP service!=}\\

          \hline
          \\
\textbf{Solution:}\\
\\


        \hline
        \\
\textbf{Log Method}\\
Details:
\rowcolor{white}{\verb=FTP Banner Detection=}\\
OID:1.3.6.1.4.1.25623.1.0.10092\\
Version used:
\rowcolor{white}{\verb=2022-02-16T13:39:14Z=}\\
\end{longtable}

\begin{footnotesize}\hyperref[host:192.168.178.37]{[ return to 192.168.178.37 ]}\end{footnotesize}
\subsection{192.168.178.20}
\label{host:192.168.178.20}

\begin{tabular}{ll}
Host scan start&Fri Apr 22 12:21:26 2022 UTC\\
Host scan end&Fri Apr 22 12:27:11 2022 UTC\\
\end{tabular}

\begin{longtable}{|l|l|}
\hline
\rowcolor{gvm_report}Service (Port)&Threat Level\\
\hline
\endfirsthead
\multicolumn{2}{l}{\hfill\ldots (continued) \ldots}\\
\hline
\rowcolor{gvm_report}Service (Port)&Threat Level\\
\hline
\endhead
\hline
\multicolumn{2}{l}{\ldots (continues) \ldots}\\
\endfoot
\hline
\endlastfoot
\hline
\hyperref[port:192.168.178.20 22/tcp Log]{22/tcp}&Log\\
\hline
\hyperref[port:192.168.178.20 general/tcp Log]{general/tcp}&Log\\
\hline
\hyperref[port:192.168.178.20 139/tcp Log]{139/tcp}&Log\\
\hline
\hyperref[port:192.168.178.20 631/tcp Log]{631/tcp}&Log\\
\hline
\hyperref[port:192.168.178.20 53/tcp Log]{53/tcp}&Log\\
\hline
\hyperref[port:192.168.178.20 445/tcp Log]{445/tcp}&Log\\
\hline
\end{longtable}


%\subsection*{Security Issues and Fixes -- 192.168.178.20}

\subsubsection{Log 22/tcp}
\label{port:192.168.178.20 22/tcp Log}

\begin{longtable}{|p{\textwidth * 1}|}
\hline
\rowcolor{gvm_log}{\color{white}{Log (CVSS: 0.0) }}\\
\rowcolor{gvm_log}{\color{white}{NVT: SSH Server type and version}}\\
\hline
\endfirsthead
\hfill\ldots continued from previous page \ldots \\
\hline
\endhead
\hline
\ldots continues on next page \ldots \\
\endfoot
\hline
\endlastfoot
\\
\textbf{Summary}\\
This detects the SSH Server's type and version by connecting to the server
  and processing the buffer received.\\
  This information gives potential attackers additional information about the system they are attacking.
  Versions and Types should be omitted where possible.\\

        \hline
        \\
\textbf{Vulnerability Detection Result}\\
\rowcolor{white}{\verb=Remote SSH server banner: SSH-2.0-OpenSSH_8.2p1 Ubuntu-4ubuntu0.5=}\\
\rowcolor{white}{\verb=Remote SSH supported authentication: password,publickey=}\\
\rowcolor{white}{\verb=Remote SSH text/login banner: (not available)=}\\
\rowcolor{white}{\verb=This is probably:=}\\
\rowcolor{white}{\verb=- OpenSSH=}\\
\rowcolor{white}{\verb=Concluded from remote connection attempt with credentials:=}\\
\rowcolor{white}{\verb=Login:    OpenVASVT=}\\
\rowcolor{white}{\verb=Password: OpenVASVT=}\\

          \hline
          \\
\textbf{Solution:}\\
\\


        \hline
        \\
\textbf{Log Method}\\
Details:
\rowcolor{white}{\verb=SSH Server type and version=}\\
OID:1.3.6.1.4.1.25623.1.0.10267\\
Version used:
\rowcolor{white}{\verb=2021-09-28T06:32:28Z=}\\
\end{longtable}

\begin{longtable}{|p{\textwidth * 1}|}
\hline
\rowcolor{gvm_log}{\color{white}{Log (CVSS: 0.0) }}\\
\rowcolor{gvm_log}{\color{white}{NVT: Services}}\\
\hline
\endfirsthead
\hfill\ldots continued from previous page \ldots \\
\hline
\endhead
\hline
\ldots continues on next page \ldots \\
\endfoot
\hline
\endlastfoot
\\
\textbf{Summary}\\
This routine attempts to guess which service is running on the
  remote ports. For instance, it searches for a web server which could listen on another port than
  80 or 443 and makes this information available for other check routines.\\

        \hline
        \\
\textbf{Vulnerability Detection Result}\\
\rowcolor{white}{\verb=An ssh server is running on this port=}\\

          \hline
          \\
\textbf{Solution:}\\
\\


        \hline
        \\
\textbf{Log Method}\\
Details:
\rowcolor{white}{\verb=Services=}\\
OID:1.3.6.1.4.1.25623.1.0.10330\\
Version used:
\rowcolor{white}{\verb=2021-03-15T10:42:03Z=}\\
\end{longtable}

\begin{footnotesize}\hyperref[host:192.168.178.20]{[ return to 192.168.178.20 ]}\end{footnotesize}
\subsubsection{Log general/tcp}
\label{port:192.168.178.20 general/tcp Log}

\begin{longtable}{|p{\textwidth * 1}|}
\hline
\rowcolor{gvm_log}{\color{white}{Log (CVSS: 0.0) }}\\
\rowcolor{gvm_log}{\color{white}{NVT: OS Detection Consolidation and Reporting}}\\
\hline
\endfirsthead
\hfill\ldots continued from previous page \ldots \\
\hline
\endhead
\hline
\ldots continues on next page \ldots \\
\endfoot
\hline
\endlastfoot
\\
\textbf{Summary}\\
This script consolidates the OS information detected by several
  VTs and tries to find the best matching OS.\\
  Furthermore it reports all previously collected information leading to this best matching OS. It
  also reports possible additional information which might help to improve the OS detection.\\
  If any of this information is wrong or could be improved please consider to report these to the
  referenced community portal.\\

        \hline
        \\
\textbf{Vulnerability Detection Result}\\
\rowcolor{white}{\verb=Best matching OS:=}\\
\rowcolor{white}{\verb=OS:           Ubuntu 20.04=}\\
\rowcolor{white}{\verb=Version:      20.04=}\\
\rowcolor{white}{\verb=CPE:          cpe:/o:canonical:ubuntu_linux:20.04=}\\
\rowcolor{white}{\verb=Found by NVT: 1.3.6.1.4.1.25623.1.0.105586 (Operating System (OS) Detection (SSH=}\\
\rowcolor{white}{$\hookrightarrow$\verb=))=}\\
\rowcolor{white}{\verb=Concluded from SSH banner on port 22/tcp: SSH-2.0-OpenSSH_8.2p1 Ubuntu-4ubuntu0.=}\\
\rowcolor{white}{$\hookrightarrow$\verb=5=}\\
\rowcolor{white}{\verb=Setting key "Host/runs_unixoide" based on this information=}\\
\rowcolor{white}{\verb=Other OS detections (in order of reliability):=}\\
\rowcolor{white}{\verb=OS:           Linux/Unix=}\\
\rowcolor{white}{\verb=CPE:          cpe:/o:linux:kernel=}\\
\rowcolor{white}{\verb=Found by NVT: 1.3.6.1.4.1.25623.1.0.111067 (Operating System (OS) Detection (HTT=}\\
\rowcolor{white}{$\hookrightarrow$\verb=P))=}\\
\rowcolor{white}{\verb=Concluded from HTTP Server banner on port 631/tcp: Server: CUPS/2.3 IPP/2.1=}\\

          \hline
          \\
\textbf{Solution:}\\
\\


        \hline
        \\
\textbf{Log Method}\\
Details:
\rowcolor{white}{\verb=OS Detection Consolidation and Reporting=}\\
OID:1.3.6.1.4.1.25623.1.0.105937\\
Version used:
\rowcolor{white}{\verb=2022-04-05T09:27:51Z=}\\

      \hline
      \\
\textbf{References}\\
\rowcolor{white}{\verb=url: https://community.greenbone.net/c/vulnerability-tests=}\\
\end{longtable}

\begin{longtable}{|p{\textwidth * 1}|}
\hline
\rowcolor{gvm_log}{\color{white}{Log (CVSS: 0.0) }}\\
\rowcolor{gvm_log}{\color{white}{NVT: SSL/TLS: Hostname discovery from server certificate}}\\
\hline
\endfirsthead
\hfill\ldots continued from previous page \ldots \\
\hline
\endhead
\hline
\ldots continues on next page \ldots \\
\endfoot
\hline
\endlastfoot
\\
\textbf{Summary}\\
It was possible to discover an additional hostname
  of this server from its certificate Common or Subject Alt Name.\\

        \hline
        \\
\textbf{Vulnerability Detection Result}\\
\rowcolor{white}{\verb=The following additional and resolvable hostnames pointing to a different host i=}\\
\rowcolor{white}{$\hookrightarrow$\verb=p were detected:=}\\
\rowcolor{white}{\verb=homeprinter=}\\
\rowcolor{white}{\verb=The following additional but not resolvable hostnames were detected:=}\\
\rowcolor{white}{\verb=homeprinter.local=}\\

          \hline
          \\
\textbf{Solution:}\\
\\


        \hline
        \\
\textbf{Log Method}\\
Details:
\rowcolor{white}{\verb=SSL/TLS: Hostname discovery from server certificate=}\\
OID:1.3.6.1.4.1.25623.1.0.111010\\
Version used:
\rowcolor{white}{\verb=2021-11-22T15:32:39Z=}\\
\end{longtable}

\begin{longtable}{|p{\textwidth * 1}|}
\hline
\rowcolor{gvm_log}{\color{white}{Log (CVSS: 0.0) }}\\
\rowcolor{gvm_log}{\color{white}{NVT: Hostname Determination Reporting}}\\
\hline
\endfirsthead
\hfill\ldots continued from previous page \ldots \\
\hline
\endhead
\hline
\ldots continues on next page \ldots \\
\endfoot
\hline
\endlastfoot
\\
\textbf{Summary}\\
The script reports information on how the hostname
  of the target was determined.\\

        \hline
        \\
\textbf{Vulnerability Detection Result}\\
\rowcolor{white}{\verb=Hostname determination for IP 192.168.178.20:=}\\
\rowcolor{white}{\verb=Hostname|Source=}\\
\rowcolor{white}{\verb=homeprinter.fritz.box|Reverse-DNS=}\\

          \hline
          \\
\textbf{Solution:}\\
\\


        \hline
        \\
\textbf{Log Method}\\
Details:
\rowcolor{white}{\verb=Hostname Determination Reporting=}\\
OID:1.3.6.1.4.1.25623.1.0.108449\\
Version used:
\rowcolor{white}{\verb=2018-11-19T11:11:31Z=}\\
\end{longtable}

\begin{footnotesize}\hyperref[host:192.168.178.20]{[ return to 192.168.178.20 ]}\end{footnotesize}
\subsubsection{Log 139/tcp}
\label{port:192.168.178.20 139/tcp Log}

\begin{longtable}{|p{\textwidth * 1}|}
\hline
\rowcolor{gvm_log}{\color{white}{Log (CVSS: 0.0) }}\\
\rowcolor{gvm_log}{\color{white}{NVT: SMB/CIFS Server Detection}}\\
\hline
\endfirsthead
\hfill\ldots continued from previous page \ldots \\
\hline
\endhead
\hline
\ldots continues on next page \ldots \\
\endfoot
\hline
\endlastfoot
\\
\textbf{Summary}\\
This script detects whether port 445 and 139 are open and
  if they are running a CIFS/SMB server.\\

        \hline
        \\
\textbf{Vulnerability Detection Result}\\
\rowcolor{white}{\verb=A SMB server is running on this port=}\\

          \hline
          \\
\textbf{Solution:}\\
\\


        \hline
        \\
\textbf{Log Method}\\
Details:
\rowcolor{white}{\verb=SMB/CIFS Server Detection=}\\
OID:1.3.6.1.4.1.25623.1.0.11011\\
Version used:
\rowcolor{white}{\verb=2020-11-10T15:30:28Z=}\\
\end{longtable}

\begin{footnotesize}\hyperref[host:192.168.178.20]{[ return to 192.168.178.20 ]}\end{footnotesize}
\subsubsection{Log 631/tcp}
\label{port:192.168.178.20 631/tcp Log}

\begin{longtable}{|p{\textwidth * 1}|}
\hline
\rowcolor{gvm_log}{\color{white}{Log (CVSS: 0.0) }}\\
\rowcolor{gvm_log}{\color{white}{NVT: Services}}\\
\hline
\endfirsthead
\hfill\ldots continued from previous page \ldots \\
\hline
\endhead
\hline
\ldots continues on next page \ldots \\
\endfoot
\hline
\endlastfoot
\\
\textbf{Summary}\\
This routine attempts to guess which service is running on the
  remote ports. For instance, it searches for a web server which could listen on another port than
  80 or 443 and makes this information available for other check routines.\\

        \hline
        \\
\textbf{Vulnerability Detection Result}\\
\rowcolor{white}{\verb=A TLScustom server answered on this port=}\\

          \hline
          \\
\textbf{Solution:}\\
\\


        \hline
        \\
\textbf{Log Method}\\
Details:
\rowcolor{white}{\verb=Services=}\\
OID:1.3.6.1.4.1.25623.1.0.10330\\
Version used:
\rowcolor{white}{\verb=2021-03-15T10:42:03Z=}\\
\end{longtable}

\begin{longtable}{|p{\textwidth * 1}|}
\hline
\rowcolor{gvm_log}{\color{white}{Log (CVSS: 0.0) }}\\
\rowcolor{gvm_log}{\color{white}{NVT: Services}}\\
\hline
\endfirsthead
\hfill\ldots continued from previous page \ldots \\
\hline
\endhead
\hline
\ldots continues on next page \ldots \\
\endfoot
\hline
\endlastfoot
\\
\textbf{Summary}\\
This routine attempts to guess which service is running on the
  remote ports. For instance, it searches for a web server which could listen on another port than
  80 or 443 and makes this information available for other check routines.\\

        \hline
        \\
\textbf{Vulnerability Detection Result}\\
\rowcolor{white}{\verb=A web server is running on this port through SSL=}\\

          \hline
          \\
\textbf{Solution:}\\
\\


        \hline
        \\
\textbf{Log Method}\\
Details:
\rowcolor{white}{\verb=Services=}\\
OID:1.3.6.1.4.1.25623.1.0.10330\\
Version used:
\rowcolor{white}{\verb=2021-03-15T10:42:03Z=}\\
\end{longtable}

\begin{longtable}{|p{\textwidth * 1}|}
\hline
\rowcolor{gvm_log}{\color{white}{Log (CVSS: 0.0) }}\\
\rowcolor{gvm_log}{\color{white}{NVT: SSL/TLS: Version Detection}}\\
\hline
\endfirsthead
\hfill\ldots continued from previous page \ldots \\
\hline
\endhead
\hline
\ldots continues on next page \ldots \\
\endfoot
\hline
\endlastfoot
\\
\textbf{Summary}\\
Enumeration and reporting of SSL/TLS protocol versions supported
  by a remote service.\\

        \hline
        \\
\textbf{Vulnerability Detection Result}\\
\rowcolor{white}{\verb=The remote SSL/TLS service supports the following SSL/TLS protocol version(s):=}\\
\rowcolor{white}{\verb=TLSv1.0=}\\
\rowcolor{white}{\verb=TLSv1.1=}\\
\rowcolor{white}{\verb=TLSv1.2=}\\
\rowcolor{white}{\verb=TLSv1.3=}\\

          \hline
          \\
\textbf{Solution:}\\
\\


        \hline
        \\
\textbf{Log Method}\\
Sends multiple connection requests to the remote service and
  attempts to determine the SSL/TLS protocol versions supported by the service from the replies.\\
  Note: The supported SSL/TLS protocol versions included in the report of this VT are reported
  independently from the allowed / supported SSL/TLS ciphers.\\
Details:
\rowcolor{white}{\verb=SSL/TLS: Version Detection=}\\
OID:1.3.6.1.4.1.25623.1.0.105782\\
Version used:
\rowcolor{white}{\verb=2021-12-06T15:42:24Z=}\\
\end{longtable}

\begin{longtable}{|p{\textwidth * 1}|}
\hline
\rowcolor{gvm_log}{\color{white}{Log (CVSS: 0.0) }}\\
\rowcolor{gvm_log}{\color{white}{NVT: SSL/TLS: Collect and Report Certificate Details}}\\
\hline
\endfirsthead
\hfill\ldots continued from previous page \ldots \\
\hline
\endhead
\hline
\ldots continues on next page \ldots \\
\endfoot
\hline
\endlastfoot
\\
\textbf{Summary}\\
This script collects and reports the details of all SSL/TLS
  certificates.\\
  This data will be used by other tests to verify server certificates.\\

        \hline
        \\
\textbf{Vulnerability Detection Result}\\
\rowcolor{white}{\verb=The following certificate details of the remote service were collected.=}\\
\rowcolor{white}{\verb=Certificate details:=}\\
\rowcolor{white}{\verb=fingerprint (SHA-1)             | 69429852EE28F42AC047C8CF69BC994B46C15D0F=}\\
\rowcolor{white}{\verb=fingerprint (SHA-256)           | 03C38A23289F58F7BDB0626EEF393A1CC50E5C14986692=}\\
\rowcolor{white}{$\hookrightarrow$\verb=2AE072521C4F5B7A0E=}\\
\rowcolor{white}{\verb=issued by                       | L=\verb-=-\verb=Unknown,ST=\verb-=-\verb=Unknown,OU=\verb-=-\verb=Unknown,O=\verb-=-\verb=homeprinter,=}\\
\rowcolor{white}{$\hookrightarrow$\verb=CN=\verb-=-\verb=homeprinter,C=\verb-=-\verb=DE=}\\
\rowcolor{white}{\verb=public key algorithm            | RSA=}\\
\rowcolor{white}{\verb=public key size (bits)          | 2048=}\\
\rowcolor{white}{\verb=serial                          | 61DDAF12=}\\
\rowcolor{white}{\verb=signature algorithm             | sha256WithRSAEncryption=}\\
\rowcolor{white}{\verb=subject                         | L=\verb-=-\verb=Unknown,ST=\verb-=-\verb=Unknown,OU=\verb-=-\verb=Unknown,O=\verb-=-\verb=homeprinter,=}\\
\rowcolor{white}{$\hookrightarrow$\verb=CN=\verb-=-\verb=homeprinter,C=\verb-=-\verb=DE=}\\
\rowcolor{white}{\verb=subject alternative names (SAN) | homeprinter, homeprinter.local, localhost=}\\
\rowcolor{white}{\verb=valid from                      | 2022-01-11 16:23:46 UTC=}\\
\rowcolor{white}{\verb=valid until                     | 2032-01-09 16:23:46 UTC=}\\

          \hline
          \\
\textbf{Solution:}\\
\\


        \hline
        \\
\textbf{Log Method}\\
Details:
\rowcolor{white}{\verb=SSL/TLS: Collect and Report Certificate Details=}\\
OID:1.3.6.1.4.1.25623.1.0.103692\\
Version used:
\rowcolor{white}{\verb=2021-12-10T12:48:00Z=}\\
\end{longtable}

\begin{footnotesize}\hyperref[host:192.168.178.20]{[ return to 192.168.178.20 ]}\end{footnotesize}
\subsubsection{Log 53/tcp}
\label{port:192.168.178.20 53/tcp Log}

\begin{longtable}{|p{\textwidth * 1}|}
\hline
\rowcolor{gvm_log}{\color{white}{Log (CVSS: 0.0) }}\\
\rowcolor{gvm_log}{\color{white}{NVT: DNS Server Detection (TCP)}}\\
\hline
\endfirsthead
\hfill\ldots continued from previous page \ldots \\
\hline
\endhead
\hline
\ldots continues on next page \ldots \\
\endfoot
\hline
\endlastfoot
\\
\textbf{Summary}\\
TCP based detection of a DNS server.\\

        \hline
        \\
\textbf{Vulnerability Detection Result}\\
Vulnerability was detected according to the Vulnerability Detection Method.\\

          \hline
          \\
\textbf{Solution:}\\
\\


        \hline
        \\
\textbf{Log Method}\\
Details:
\rowcolor{white}{\verb=DNS Server Detection (TCP)=}\\
OID:1.3.6.1.4.1.25623.1.0.108018\\
Version used:
\rowcolor{white}{\verb=2021-11-30T08:05:58Z=}\\
\end{longtable}

\begin{footnotesize}\hyperref[host:192.168.178.20]{[ return to 192.168.178.20 ]}\end{footnotesize}
\subsubsection{Log 445/tcp}
\label{port:192.168.178.20 445/tcp Log}

\begin{longtable}{|p{\textwidth * 1}|}
\hline
\rowcolor{gvm_log}{\color{white}{Log (CVSS: 0.0) }}\\
\rowcolor{gvm_log}{\color{white}{NVT: SMB/CIFS Server Detection}}\\
\hline
\endfirsthead
\hfill\ldots continued from previous page \ldots \\
\hline
\endhead
\hline
\ldots continues on next page \ldots \\
\endfoot
\hline
\endlastfoot
\\
\textbf{Summary}\\
This script detects whether port 445 and 139 are open and
  if they are running a CIFS/SMB server.\\

        \hline
        \\
\textbf{Vulnerability Detection Result}\\
\rowcolor{white}{\verb=A CIFS server is running on this port=}\\

          \hline
          \\
\textbf{Solution:}\\
\\


        \hline
        \\
\textbf{Log Method}\\
Details:
\rowcolor{white}{\verb=SMB/CIFS Server Detection=}\\
OID:1.3.6.1.4.1.25623.1.0.11011\\
Version used:
\rowcolor{white}{\verb=2020-11-10T15:30:28Z=}\\
\end{longtable}

\begin{longtable}{|p{\textwidth * 1}|}
\hline
\rowcolor{gvm_log}{\color{white}{Log (CVSS: 0.0) }}\\
\rowcolor{gvm_log}{\color{white}{NVT: SMB log in}}\\
\hline
\endfirsthead
\hfill\ldots continued from previous page \ldots \\
\hline
\endhead
\hline
\ldots continues on next page \ldots \\
\endfoot
\hline
\endlastfoot
\\
\textbf{Summary}\\
This script attempts to logon into the remote host using
  login/password credentials.\\

        \hline
        \\
\textbf{Vulnerability Detection Result}\\
\rowcolor{white}{\verb=It was possible to log into the remote host using the SMB protocol.=}\\

          \hline
          \\
\textbf{Solution:}\\
\\


        \hline
        \\
\textbf{Log Method}\\
Details:
\rowcolor{white}{\verb=SMB log in=}\\
OID:1.3.6.1.4.1.25623.1.0.10394\\
Version used:
\rowcolor{white}{\verb=2021-08-11T09:39:10Z=}\\
\end{longtable}

\begin{footnotesize}\hyperref[host:192.168.178.20]{[ return to 192.168.178.20 ]}\end{footnotesize}
\subsection{192.168.178.22}
\label{host:192.168.178.22}

\begin{tabular}{ll}
Host scan start&Fri Apr 22 12:21:26 2022 UTC\\
Host scan end&Fri Apr 22 12:31:04 2022 UTC\\
\end{tabular}

\begin{longtable}{|l|l|}
\hline
\rowcolor{gvm_report}Service (Port)&Threat Level\\
\hline
\endfirsthead
\multicolumn{2}{l}{\hfill\ldots (continued) \ldots}\\
\hline
\rowcolor{gvm_report}Service (Port)&Threat Level\\
\hline
\endhead
\hline
\multicolumn{2}{l}{\ldots (continues) \ldots}\\
\endfoot
\hline
\endlastfoot
\hline
\hyperref[port:192.168.178.22 22/tcp Log]{22/tcp}&Log\\
\hline
\hyperref[port:192.168.178.22 139/tcp Log]{139/tcp}&Log\\
\hline
\hyperref[port:192.168.178.22 general/tcp Log]{general/tcp}&Log\\
\hline
\hyperref[port:192.168.178.22 53/tcp Log]{53/tcp}&Log\\
\hline
\hyperref[port:192.168.178.22 631/tcp Log]{631/tcp}&Log\\
\hline
\hyperref[port:192.168.178.22 445/tcp Log]{445/tcp}&Log\\
\hline
\end{longtable}


%\subsection*{Security Issues and Fixes -- 192.168.178.22}

\subsubsection{Log 22/tcp}
\label{port:192.168.178.22 22/tcp Log}

\begin{longtable}{|p{\textwidth * 1}|}
\hline
\rowcolor{gvm_log}{\color{white}{Log (CVSS: 0.0) }}\\
\rowcolor{gvm_log}{\color{white}{NVT: Services}}\\
\hline
\endfirsthead
\hfill\ldots continued from previous page \ldots \\
\hline
\endhead
\hline
\ldots continues on next page \ldots \\
\endfoot
\hline
\endlastfoot
\\
\textbf{Summary}\\
This routine attempts to guess which service is running on the
  remote ports. For instance, it searches for a web server which could listen on another port than
  80 or 443 and makes this information available for other check routines.\\

        \hline
        \\
\textbf{Vulnerability Detection Result}\\
\rowcolor{white}{\verb=An ssh server is running on this port=}\\

          \hline
          \\
\textbf{Solution:}\\
\\


        \hline
        \\
\textbf{Log Method}\\
Details:
\rowcolor{white}{\verb=Services=}\\
OID:1.3.6.1.4.1.25623.1.0.10330\\
Version used:
\rowcolor{white}{\verb=2021-03-15T10:42:03Z=}\\
\end{longtable}

\begin{longtable}{|p{\textwidth * 1}|}
\hline
\rowcolor{gvm_log}{\color{white}{Log (CVSS: 0.0) }}\\
\rowcolor{gvm_log}{\color{white}{NVT: SSH Server type and version}}\\
\hline
\endfirsthead
\hfill\ldots continued from previous page \ldots \\
\hline
\endhead
\hline
\ldots continues on next page \ldots \\
\endfoot
\hline
\endlastfoot
\\
\textbf{Summary}\\
This detects the SSH Server's type and version by connecting to the server
  and processing the buffer received.\\
  This information gives potential attackers additional information about the system they are attacking.
  Versions and Types should be omitted where possible.\\

        \hline
        \\
\textbf{Vulnerability Detection Result}\\
\rowcolor{white}{\verb=Remote SSH server banner: SSH-2.0-OpenSSH_8.2p1 Ubuntu-4ubuntu0.5=}\\
\rowcolor{white}{\verb=Remote SSH supported authentication: password,publickey=}\\
\rowcolor{white}{\verb=Remote SSH text/login banner: (not available)=}\\
\rowcolor{white}{\verb=This is probably:=}\\
\rowcolor{white}{\verb=- OpenSSH=}\\
\rowcolor{white}{\verb=Concluded from remote connection attempt with credentials:=}\\
\rowcolor{white}{\verb=Login:    OpenVASVT=}\\
\rowcolor{white}{\verb=Password: OpenVASVT=}\\

          \hline
          \\
\textbf{Solution:}\\
\\


        \hline
        \\
\textbf{Log Method}\\
Details:
\rowcolor{white}{\verb=SSH Server type and version=}\\
OID:1.3.6.1.4.1.25623.1.0.10267\\
Version used:
\rowcolor{white}{\verb=2021-09-28T06:32:28Z=}\\
\end{longtable}

\begin{footnotesize}\hyperref[host:192.168.178.22]{[ return to 192.168.178.22 ]}\end{footnotesize}
\subsubsection{Log 139/tcp}
\label{port:192.168.178.22 139/tcp Log}

\begin{longtable}{|p{\textwidth * 1}|}
\hline
\rowcolor{gvm_log}{\color{white}{Log (CVSS: 0.0) }}\\
\rowcolor{gvm_log}{\color{white}{NVT: SMB/CIFS Server Detection}}\\
\hline
\endfirsthead
\hfill\ldots continued from previous page \ldots \\
\hline
\endhead
\hline
\ldots continues on next page \ldots \\
\endfoot
\hline
\endlastfoot
\\
\textbf{Summary}\\
This script detects whether port 445 and 139 are open and
  if they are running a CIFS/SMB server.\\

        \hline
        \\
\textbf{Vulnerability Detection Result}\\
\rowcolor{white}{\verb=A SMB server is running on this port=}\\

          \hline
          \\
\textbf{Solution:}\\
\\


        \hline
        \\
\textbf{Log Method}\\
Details:
\rowcolor{white}{\verb=SMB/CIFS Server Detection=}\\
OID:1.3.6.1.4.1.25623.1.0.11011\\
Version used:
\rowcolor{white}{\verb=2020-11-10T15:30:28Z=}\\
\end{longtable}

\begin{footnotesize}\hyperref[host:192.168.178.22]{[ return to 192.168.178.22 ]}\end{footnotesize}
\subsubsection{Log general/tcp}
\label{port:192.168.178.22 general/tcp Log}

\begin{longtable}{|p{\textwidth * 1}|}
\hline
\rowcolor{gvm_log}{\color{white}{Log (CVSS: 0.0) }}\\
\rowcolor{gvm_log}{\color{white}{NVT: SSL/TLS: Hostname discovery from server certificate}}\\
\hline
\endfirsthead
\hfill\ldots continued from previous page \ldots \\
\hline
\endhead
\hline
\ldots continues on next page \ldots \\
\endfoot
\hline
\endlastfoot
\\
\textbf{Summary}\\
It was possible to discover an additional hostname
  of this server from its certificate Common or Subject Alt Name.\\

        \hline
        \\
\textbf{Vulnerability Detection Result}\\
\rowcolor{white}{\verb=The following additional and resolvable hostnames were detected:=}\\
\rowcolor{white}{\verb=homeprinter=}\\
\rowcolor{white}{\verb=The following additional but not resolvable hostnames were detected:=}\\
\rowcolor{white}{\verb=homeprinter.local=}\\

          \hline
          \\
\textbf{Solution:}\\
\\


        \hline
        \\
\textbf{Log Method}\\
Details:
\rowcolor{white}{\verb=SSL/TLS: Hostname discovery from server certificate=}\\
OID:1.3.6.1.4.1.25623.1.0.111010\\
Version used:
\rowcolor{white}{\verb=2021-11-22T15:32:39Z=}\\
\end{longtable}

\begin{longtable}{|p{\textwidth * 1}|}
\hline
\rowcolor{gvm_log}{\color{white}{Log (CVSS: 0.0) }}\\
\rowcolor{gvm_log}{\color{white}{NVT: OS Detection Consolidation and Reporting}}\\
\hline
\endfirsthead
\hfill\ldots continued from previous page \ldots \\
\hline
\endhead
\hline
\ldots continues on next page \ldots \\
\endfoot
\hline
\endlastfoot
\\
\textbf{Summary}\\
This script consolidates the OS information detected by several
  VTs and tries to find the best matching OS.\\
  Furthermore it reports all previously collected information leading to this best matching OS. It
  also reports possible additional information which might help to improve the OS detection.\\
  If any of this information is wrong or could be improved please consider to report these to the
  referenced community portal.\\

        \hline
        \\
\textbf{Vulnerability Detection Result}\\
\rowcolor{white}{\verb=Best matching OS:=}\\
\rowcolor{white}{\verb=OS:           Ubuntu 20.04=}\\
\rowcolor{white}{\verb=Version:      20.04=}\\
\rowcolor{white}{\verb=CPE:          cpe:/o:canonical:ubuntu_linux:20.04=}\\
\rowcolor{white}{\verb=Found by NVT: 1.3.6.1.4.1.25623.1.0.105586 (Operating System (OS) Detection (SSH=}\\
\rowcolor{white}{$\hookrightarrow$\verb=))=}\\
\rowcolor{white}{\verb=Concluded from SSH banner on port 22/tcp: SSH-2.0-OpenSSH_8.2p1 Ubuntu-4ubuntu0.=}\\
\rowcolor{white}{$\hookrightarrow$\verb=5=}\\
\rowcolor{white}{\verb=Setting key "Host/runs_unixoide" based on this information=}\\
\rowcolor{white}{\verb=Other OS detections (in order of reliability):=}\\
\rowcolor{white}{\verb=OS:           Linux/Unix=}\\
\rowcolor{white}{\verb=CPE:          cpe:/o:linux:kernel=}\\
\rowcolor{white}{\verb=Found by NVT: 1.3.6.1.4.1.25623.1.0.111067 (Operating System (OS) Detection (HTT=}\\
\rowcolor{white}{$\hookrightarrow$\verb=P))=}\\
\rowcolor{white}{\verb=Concluded from HTTP Server banner on port 631/tcp: Server: CUPS/2.3 IPP/2.1=}\\
\rowcolor{white}{\verb=OS:           Linux/Unix=}\\
\rowcolor{white}{\verb=CPE:          cpe:/o:linux:kernel=}\\
\rowcolor{white}{\verb=Found by NVT: 1.3.6.1.4.1.25623.1.0.111067 (Operating System (OS) Detection (HTT=}\\
\rowcolor{white}{$\hookrightarrow$\verb=P))=}\\
\rowcolor{white}{\verb=Concluded from HTTP Server default page on port 631/tcp: <title>Home - CUPS 2.3.=}\\
\rowcolor{white}{$\hookrightarrow$\verb=1</title>=}\\

          \hline
          \\
\textbf{Solution:}\\
\\


        \hline
        \\
\textbf{Log Method}\\
Details:
\rowcolor{white}{\verb=OS Detection Consolidation and Reporting=}\\
OID:1.3.6.1.4.1.25623.1.0.105937\\
Version used:
\rowcolor{white}{\verb=2022-04-05T09:27:51Z=}\\

      \hline
      \\
\textbf{References}\\
\rowcolor{white}{\verb=url: https://community.greenbone.net/c/vulnerability-tests=}\\
\end{longtable}

\begin{longtable}{|p{\textwidth * 1}|}
\hline
\rowcolor{gvm_log}{\color{white}{Log (CVSS: 0.0) }}\\
\rowcolor{gvm_log}{\color{white}{NVT: Hostname Determination Reporting}}\\
\hline
\endfirsthead
\hfill\ldots continued from previous page \ldots \\
\hline
\endhead
\hline
\ldots continues on next page \ldots \\
\endfoot
\hline
\endlastfoot
\\
\textbf{Summary}\\
The script reports information on how the hostname
  of the target was determined.\\

        \hline
        \\
\textbf{Vulnerability Detection Result}\\
\rowcolor{white}{\verb=Hostname determination for IP 192.168.178.22:=}\\
\rowcolor{white}{\verb=Hostname|Source=}\\
\rowcolor{white}{\verb=homeprinter|SSL/TLS server certificate=}\\
\rowcolor{white}{\verb=homeprinter.fritz.box|Reverse-DNS=}\\

          \hline
          \\
\textbf{Solution:}\\
\\


        \hline
        \\
\textbf{Log Method}\\
Details:
\rowcolor{white}{\verb=Hostname Determination Reporting=}\\
OID:1.3.6.1.4.1.25623.1.0.108449\\
Version used:
\rowcolor{white}{\verb=2018-11-19T11:11:31Z=}\\
\end{longtable}

\begin{footnotesize}\hyperref[host:192.168.178.22]{[ return to 192.168.178.22 ]}\end{footnotesize}
\subsubsection{Log 53/tcp}
\label{port:192.168.178.22 53/tcp Log}

\begin{longtable}{|p{\textwidth * 1}|}
\hline
\rowcolor{gvm_log}{\color{white}{Log (CVSS: 0.0) }}\\
\rowcolor{gvm_log}{\color{white}{NVT: DNS Server Detection (TCP)}}\\
\hline
\endfirsthead
\hfill\ldots continued from previous page \ldots \\
\hline
\endhead
\hline
\ldots continues on next page \ldots \\
\endfoot
\hline
\endlastfoot
\\
\textbf{Summary}\\
TCP based detection of a DNS server.\\

        \hline
        \\
\textbf{Vulnerability Detection Result}\\
\rowcolor{white}{\verb=The remote DNS server banner is:=}\\
\rowcolor{white}{\verb=unbound 1.15.0=}\\

          \hline
          \\
\textbf{Solution:}\\
\\


        \hline
        \\
\textbf{Log Method}\\
Details:
\rowcolor{white}{\verb=DNS Server Detection (TCP)=}\\
OID:1.3.6.1.4.1.25623.1.0.108018\\
Version used:
\rowcolor{white}{\verb=2021-11-30T08:05:58Z=}\\
\end{longtable}

\begin{footnotesize}\hyperref[host:192.168.178.22]{[ return to 192.168.178.22 ]}\end{footnotesize}
\subsubsection{Log 631/tcp}
\label{port:192.168.178.22 631/tcp Log}

\begin{longtable}{|p{\textwidth * 1}|}
\hline
\rowcolor{gvm_log}{\color{white}{Log (CVSS: 0.0) }}\\
\rowcolor{gvm_log}{\color{white}{NVT: Services}}\\
\hline
\endfirsthead
\hfill\ldots continued from previous page \ldots \\
\hline
\endhead
\hline
\ldots continues on next page \ldots \\
\endfoot
\hline
\endlastfoot
\\
\textbf{Summary}\\
This routine attempts to guess which service is running on the
  remote ports. For instance, it searches for a web server which could listen on another port than
  80 or 443 and makes this information available for other check routines.\\

        \hline
        \\
\textbf{Vulnerability Detection Result}\\
\rowcolor{white}{\verb=A TLScustom server answered on this port=}\\

          \hline
          \\
\textbf{Solution:}\\
\\


        \hline
        \\
\textbf{Log Method}\\
Details:
\rowcolor{white}{\verb=Services=}\\
OID:1.3.6.1.4.1.25623.1.0.10330\\
Version used:
\rowcolor{white}{\verb=2021-03-15T10:42:03Z=}\\
\end{longtable}

\begin{longtable}{|p{\textwidth * 1}|}
\hline
\rowcolor{gvm_log}{\color{white}{Log (CVSS: 0.0) }}\\
\rowcolor{gvm_log}{\color{white}{NVT: Services}}\\
\hline
\endfirsthead
\hfill\ldots continued from previous page \ldots \\
\hline
\endhead
\hline
\ldots continues on next page \ldots \\
\endfoot
\hline
\endlastfoot
\\
\textbf{Summary}\\
This routine attempts to guess which service is running on the
  remote ports. For instance, it searches for a web server which could listen on another port than
  80 or 443 and makes this information available for other check routines.\\

        \hline
        \\
\textbf{Vulnerability Detection Result}\\
\rowcolor{white}{\verb=A web server is running on this port through SSL=}\\

          \hline
          \\
\textbf{Solution:}\\
\\


        \hline
        \\
\textbf{Log Method}\\
Details:
\rowcolor{white}{\verb=Services=}\\
OID:1.3.6.1.4.1.25623.1.0.10330\\
Version used:
\rowcolor{white}{\verb=2021-03-15T10:42:03Z=}\\
\end{longtable}

\begin{longtable}{|p{\textwidth * 1}|}
\hline
\rowcolor{gvm_log}{\color{white}{Log (CVSS: 0.0) }}\\
\rowcolor{gvm_log}{\color{white}{NVT: SSL/TLS: Version Detection}}\\
\hline
\endfirsthead
\hfill\ldots continued from previous page \ldots \\
\hline
\endhead
\hline
\ldots continues on next page \ldots \\
\endfoot
\hline
\endlastfoot
\\
\textbf{Summary}\\
Enumeration and reporting of SSL/TLS protocol versions supported
  by a remote service.\\

        \hline
        \\
\textbf{Vulnerability Detection Result}\\
\rowcolor{white}{\verb=The remote SSL/TLS service supports the following SSL/TLS protocol version(s):=}\\
\rowcolor{white}{\verb=TLSv1.0=}\\
\rowcolor{white}{\verb=TLSv1.1=}\\
\rowcolor{white}{\verb=TLSv1.2=}\\
\rowcolor{white}{\verb=TLSv1.3=}\\

          \hline
          \\
\textbf{Solution:}\\
\\


        \hline
        \\
\textbf{Log Method}\\
Sends multiple connection requests to the remote service and
  attempts to determine the SSL/TLS protocol versions supported by the service from the replies.\\
  Note: The supported SSL/TLS protocol versions included in the report of this VT are reported
  independently from the allowed / supported SSL/TLS ciphers.\\
Details:
\rowcolor{white}{\verb=SSL/TLS: Version Detection=}\\
OID:1.3.6.1.4.1.25623.1.0.105782\\
Version used:
\rowcolor{white}{\verb=2021-12-06T15:42:24Z=}\\
\end{longtable}

\begin{longtable}{|p{\textwidth * 1}|}
\hline
\rowcolor{gvm_log}{\color{white}{Log (CVSS: 0.0) }}\\
\rowcolor{gvm_log}{\color{white}{NVT: SSL/TLS: Collect and Report Certificate Details}}\\
\hline
\endfirsthead
\hfill\ldots continued from previous page \ldots \\
\hline
\endhead
\hline
\ldots continues on next page \ldots \\
\endfoot
\hline
\endlastfoot
\\
\textbf{Summary}\\
This script collects and reports the details of all SSL/TLS
  certificates.\\
  This data will be used by other tests to verify server certificates.\\

        \hline
        \\
\textbf{Vulnerability Detection Result}\\
\rowcolor{white}{\verb=The following certificate details of the remote service were collected.=}\\
\rowcolor{white}{\verb=Certificate details:=}\\
\rowcolor{white}{\verb=fingerprint (SHA-1)             | 69429852EE28F42AC047C8CF69BC994B46C15D0F=}\\
\rowcolor{white}{\verb=fingerprint (SHA-256)           | 03C38A23289F58F7BDB0626EEF393A1CC50E5C14986692=}\\
\rowcolor{white}{$\hookrightarrow$\verb=2AE072521C4F5B7A0E=}\\
\rowcolor{white}{\verb=issued by                       | L=\verb-=-\verb=Unknown,ST=\verb-=-\verb=Unknown,OU=\verb-=-\verb=Unknown,O=\verb-=-\verb=homeprinter,=}\\
\rowcolor{white}{$\hookrightarrow$\verb=CN=\verb-=-\verb=homeprinter,C=\verb-=-\verb=DE=}\\
\rowcolor{white}{\verb=public key algorithm            | RSA=}\\
\rowcolor{white}{\verb=public key size (bits)          | 2048=}\\
\rowcolor{white}{\verb=serial                          | 61DDAF12=}\\
\rowcolor{white}{\verb=signature algorithm             | sha256WithRSAEncryption=}\\
\rowcolor{white}{\verb=subject                         | L=\verb-=-\verb=Unknown,ST=\verb-=-\verb=Unknown,OU=\verb-=-\verb=Unknown,O=\verb-=-\verb=homeprinter,=}\\
\rowcolor{white}{$\hookrightarrow$\verb=CN=\verb-=-\verb=homeprinter,C=\verb-=-\verb=DE=}\\
\rowcolor{white}{\verb=subject alternative names (SAN) | homeprinter, homeprinter.local, localhost=}\\
\rowcolor{white}{\verb=valid from                      | 2022-01-11 16:23:46 UTC=}\\
\rowcolor{white}{\verb=valid until                     | 2032-01-09 16:23:46 UTC=}\\

          \hline
          \\
\textbf{Solution:}\\
\\


        \hline
        \\
\textbf{Log Method}\\
Details:
\rowcolor{white}{\verb=SSL/TLS: Collect and Report Certificate Details=}\\
OID:1.3.6.1.4.1.25623.1.0.103692\\
Version used:
\rowcolor{white}{\verb=2021-12-10T12:48:00Z=}\\
\end{longtable}

\begin{footnotesize}\hyperref[host:192.168.178.22]{[ return to 192.168.178.22 ]}\end{footnotesize}
\subsubsection{Log 445/tcp}
\label{port:192.168.178.22 445/tcp Log}

\begin{longtable}{|p{\textwidth * 1}|}
\hline
\rowcolor{gvm_log}{\color{white}{Log (CVSS: 0.0) }}\\
\rowcolor{gvm_log}{\color{white}{NVT: SMB/CIFS Server Detection}}\\
\hline
\endfirsthead
\hfill\ldots continued from previous page \ldots \\
\hline
\endhead
\hline
\ldots continues on next page \ldots \\
\endfoot
\hline
\endlastfoot
\\
\textbf{Summary}\\
This script detects whether port 445 and 139 are open and
  if they are running a CIFS/SMB server.\\

        \hline
        \\
\textbf{Vulnerability Detection Result}\\
\rowcolor{white}{\verb=A CIFS server is running on this port=}\\

          \hline
          \\
\textbf{Solution:}\\
\\


        \hline
        \\
\textbf{Log Method}\\
Details:
\rowcolor{white}{\verb=SMB/CIFS Server Detection=}\\
OID:1.3.6.1.4.1.25623.1.0.11011\\
Version used:
\rowcolor{white}{\verb=2020-11-10T15:30:28Z=}\\
\end{longtable}

\begin{longtable}{|p{\textwidth * 1}|}
\hline
\rowcolor{gvm_log}{\color{white}{Log (CVSS: 0.0) }}\\
\rowcolor{gvm_log}{\color{white}{NVT: SMB log in}}\\
\hline
\endfirsthead
\hfill\ldots continued from previous page \ldots \\
\hline
\endhead
\hline
\ldots continues on next page \ldots \\
\endfoot
\hline
\endlastfoot
\\
\textbf{Summary}\\
This script attempts to logon into the remote host using
  login/password credentials.\\

        \hline
        \\
\textbf{Vulnerability Detection Result}\\
\rowcolor{white}{\verb=It was possible to log into the remote host using the SMB protocol.=}\\

          \hline
          \\
\textbf{Solution:}\\
\\


        \hline
        \\
\textbf{Log Method}\\
Details:
\rowcolor{white}{\verb=SMB log in=}\\
OID:1.3.6.1.4.1.25623.1.0.10394\\
Version used:
\rowcolor{white}{\verb=2021-08-11T09:39:10Z=}\\
\end{longtable}

\begin{footnotesize}\hyperref[host:192.168.178.22]{[ return to 192.168.178.22 ]}\end{footnotesize}
\subsection{192.168.178.1}
\label{host:192.168.178.1}

\begin{tabular}{ll}
Host scan start&Fri Apr 22 12:21:26 2022 UTC\\
Host scan end&Fri Apr 22 13:15:03 2022 UTC\\
\end{tabular}

\begin{longtable}{|l|l|}
\hline
\rowcolor{gvm_report}Service (Port)&Threat Level\\
\hline
\endfirsthead
\multicolumn{2}{l}{\hfill\ldots (continued) \ldots}\\
\hline
\rowcolor{gvm_report}Service (Port)&Threat Level\\
\hline
\endhead
\hline
\multicolumn{2}{l}{\ldots (continues) \ldots}\\
\endfoot
\hline
\endlastfoot
\hline
\hyperref[port:192.168.178.1 5060/tcp Log]{5060/tcp}&Log\\
\hline
\hyperref[port:192.168.178.1 8181/tcp Log]{8181/tcp}&Log\\
\hline
\hyperref[port:192.168.178.1 8182/tcp Log]{8182/tcp}&Log\\
\hline
\hyperref[port:192.168.178.1 21/tcp Log]{21/tcp}&Log\\
\hline
\hyperref[port:192.168.178.1 53/tcp Log]{53/tcp}&Log\\
\hline
\hyperref[port:192.168.178.1 8183/tcp Log]{8183/tcp}&Log\\
\hline
\hyperref[port:192.168.178.1 general/tcp Log]{general/tcp}&Log\\
\hline
\hyperref[port:192.168.178.1 8184/tcp Log]{8184/tcp}&Log\\
\hline
\hyperref[port:192.168.178.1 49000/tcp Log]{49000/tcp}&Log\\
\hline
\hyperref[port:192.168.178.1 80/tcp Log]{80/tcp}&Log\\
\hline
\hyperref[port:192.168.178.1 443/tcp Log]{443/tcp}&Log\\
\hline
\end{longtable}


%\subsection*{Security Issues and Fixes -- 192.168.178.1}

\subsubsection{Log 5060/tcp}
\label{port:192.168.178.1 5060/tcp Log}

\begin{longtable}{|p{\textwidth * 1}|}
\hline
\rowcolor{gvm_log}{\color{white}{Log (CVSS: 0.0) }}\\
\rowcolor{gvm_log}{\color{white}{NVT: Service Detection with 'GET' Request}}\\
\hline
\endfirsthead
\hfill\ldots continued from previous page \ldots \\
\hline
\endhead
\hline
\ldots continues on next page \ldots \\
\endfoot
\hline
\endlastfoot
\\
\textbf{Summary}\\
This plugin performs service detection.\\
  This plugin is a complement of find\_service.nasl. It sends a 'GET' request
  to the remaining unknown services and tries to identify them.\\

        \hline
        \\
\textbf{Vulnerability Detection Result}\\
\rowcolor{white}{\verb=A service supporting the SIP protocol seems to be running on this port.=}\\

          \hline
          \\
\textbf{Solution:}\\
\\


        \hline
        \\
\textbf{Log Method}\\
Details:
\rowcolor{white}{\verb=Service Detection with 'GET' Request=}\\
OID:1.3.6.1.4.1.25623.1.0.17975\\
Version used:
\rowcolor{white}{\verb=2022-02-01T12:51:06Z=}\\
\end{longtable}

\begin{longtable}{|p{\textwidth * 1}|}
\hline
\rowcolor{gvm_log}{\color{white}{Log (CVSS: 0.0) }}\\
\rowcolor{gvm_log}{\color{white}{NVT: Detect SIP Compatible Hosts (TCP)}}\\
\hline
\endfirsthead
\hfill\ldots continued from previous page \ldots \\
\hline
\endhead
\hline
\ldots continues on next page \ldots \\
\endfoot
\hline
\endlastfoot
\\
\textbf{Summary}\\
A Voice Over IP service is listening on the remote port.\\
  The remote host is running SIP (Session Initiation Protocol), a protocol
  used for Internet conferencing and telephony.\\

        \hline
        \\
\textbf{Vulnerability Detection Result}\\
\rowcolor{white}{\verb=User-Agent: FRITZ!OS=}\\
\rowcolor{white}{\verb=Full banner output:=}\\
\rowcolor{white}{\verb=SIP/2.0 400 Illegal request line=}\\
\rowcolor{white}{\verb=From: <sip:missing>=}\\
\rowcolor{white}{\verb=To: <sip:missing>;tag=\verb-=-\verb=badrequest=}\\
\rowcolor{white}{\verb=User-Agent: FRITZ!OS=}\\
\rowcolor{white}{\verb=Content-Length: 0=}\\

          \hline
          \\
\textbf{Solution:}\\
\\


        \hline
        \\
\textbf{Log Method}\\
Details:
\rowcolor{white}{\verb=Detect SIP Compatible Hosts (TCP)=}\\
OID:1.3.6.1.4.1.25623.1.0.108020\\
Version used:
\rowcolor{white}{\verb=2020-11-10T15:30:28Z=}\\

      \hline
      \\
\textbf{References}\\
\rowcolor{white}{\verb=url: http://www.cs.columbia.edu/sip/=}\\
\end{longtable}

\begin{footnotesize}\hyperref[host:192.168.178.1]{[ return to 192.168.178.1 ]}\end{footnotesize}
\subsubsection{Log 8181/tcp}
\label{port:192.168.178.1 8181/tcp Log}

\begin{longtable}{|p{\textwidth * 1}|}
\hline
\rowcolor{gvm_log}{\color{white}{Log (CVSS: 0.0) }}\\
\rowcolor{gvm_log}{\color{white}{NVT: Services}}\\
\hline
\endfirsthead
\hfill\ldots continued from previous page \ldots \\
\hline
\endhead
\hline
\ldots continues on next page \ldots \\
\endfoot
\hline
\endlastfoot
\\
\textbf{Summary}\\
This routine attempts to guess which service is running on the
  remote ports. For instance, it searches for a web server which could listen on another port than
  80 or 443 and makes this information available for other check routines.\\

        \hline
        \\
\textbf{Vulnerability Detection Result}\\
\rowcolor{white}{\verb=A web server is running on this port=}\\

          \hline
          \\
\textbf{Solution:}\\
\\


        \hline
        \\
\textbf{Log Method}\\
Details:
\rowcolor{white}{\verb=Services=}\\
OID:1.3.6.1.4.1.25623.1.0.10330\\
Version used:
\rowcolor{white}{\verb=2021-03-15T10:42:03Z=}\\
\end{longtable}

\begin{footnotesize}\hyperref[host:192.168.178.1]{[ return to 192.168.178.1 ]}\end{footnotesize}
\subsubsection{Log 8182/tcp}
\label{port:192.168.178.1 8182/tcp Log}

\begin{longtable}{|p{\textwidth * 1}|}
\hline
\rowcolor{gvm_log}{\color{white}{Log (CVSS: 0.0) }}\\
\rowcolor{gvm_log}{\color{white}{NVT: Services}}\\
\hline
\endfirsthead
\hfill\ldots continued from previous page \ldots \\
\hline
\endhead
\hline
\ldots continues on next page \ldots \\
\endfoot
\hline
\endlastfoot
\\
\textbf{Summary}\\
This routine attempts to guess which service is running on the
  remote ports. For instance, it searches for a web server which could listen on another port than
  80 or 443 and makes this information available for other check routines.\\

        \hline
        \\
\textbf{Vulnerability Detection Result}\\
\rowcolor{white}{\verb=A web server is running on this port=}\\

          \hline
          \\
\textbf{Solution:}\\
\\


        \hline
        \\
\textbf{Log Method}\\
Details:
\rowcolor{white}{\verb=Services=}\\
OID:1.3.6.1.4.1.25623.1.0.10330\\
Version used:
\rowcolor{white}{\verb=2021-03-15T10:42:03Z=}\\
\end{longtable}

\begin{footnotesize}\hyperref[host:192.168.178.1]{[ return to 192.168.178.1 ]}\end{footnotesize}
\subsubsection{Log 21/tcp}
\label{port:192.168.178.1 21/tcp Log}

\begin{longtable}{|p{\textwidth * 1}|}
\hline
\rowcolor{gvm_log}{\color{white}{Log (CVSS: 0.0) }}\\
\rowcolor{gvm_log}{\color{white}{NVT: Services}}\\
\hline
\endfirsthead
\hfill\ldots continued from previous page \ldots \\
\hline
\endhead
\hline
\ldots continues on next page \ldots \\
\endfoot
\hline
\endlastfoot
\\
\textbf{Summary}\\
This routine attempts to guess which service is running on the
  remote ports. For instance, it searches for a web server which could listen on another port than
  80 or 443 and makes this information available for other check routines.\\

        \hline
        \\
\textbf{Vulnerability Detection Result}\\
\rowcolor{white}{\verb=An FTP server is running on this port.=}\\
\rowcolor{white}{\verb=Here is its banner : =}\\
\rowcolor{white}{\verb=220 FRITZ!Box7490 FTP server ready.=}\\

          \hline
          \\
\textbf{Solution:}\\
\\


        \hline
        \\
\textbf{Log Method}\\
Details:
\rowcolor{white}{\verb=Services=}\\
OID:1.3.6.1.4.1.25623.1.0.10330\\
Version used:
\rowcolor{white}{\verb=2021-03-15T10:42:03Z=}\\
\end{longtable}

\begin{longtable}{|p{\textwidth * 1}|}
\hline
\rowcolor{gvm_log}{\color{white}{Log (CVSS: 0.0) }}\\
\rowcolor{gvm_log}{\color{white}{NVT: FTP Banner Detection}}\\
\hline
\endfirsthead
\hfill\ldots continued from previous page \ldots \\
\hline
\endhead
\hline
\ldots continues on next page \ldots \\
\endfoot
\hline
\endlastfoot
\\
\textbf{Summary}\\
This Plugin detects and reports a FTP Server Banner.\\

        \hline
        \\
\textbf{Vulnerability Detection Result}\\
\rowcolor{white}{\verb=Remote FTP server banner:=}\\
\rowcolor{white}{\verb=220 FRITZ!Box7490 FTP server ready.=}\\
\rowcolor{white}{\verb=This is probably (a):=}\\
\rowcolor{white}{\verb=- AVM FRITZ!Box FTP=}\\
\rowcolor{white}{\verb=Server operating system information collected via "SYST" command:=}\\
\rowcolor{white}{\verb=215 UNIX Type: L8 Version: Linux 3.10.107=}\\
\rowcolor{white}{\verb=Server status information collected via "STAT" command:=}\\
\rowcolor{white}{\verb=211- FRITZ!Box7490 FTP server status:=}\\
\rowcolor{white}{\verb=     Connected to 192.168.178.2=}\\
\rowcolor{white}{\verb=     Waiting for user name=}\\
\rowcolor{white}{\verb=     TYPE: ASCII, FORM: Nonprint; STRUcture: File; transfer MODE: Stream=}\\
\rowcolor{white}{\verb=     No data connection=}\\
\rowcolor{white}{\verb=211 End of status=}\\

          \hline
          \\
\textbf{Solution:}\\
\\


        \hline
        \\
\textbf{Log Method}\\
Details:
\rowcolor{white}{\verb=FTP Banner Detection=}\\
OID:1.3.6.1.4.1.25623.1.0.10092\\
Version used:
\rowcolor{white}{\verb=2022-02-16T13:39:14Z=}\\
\end{longtable}

\begin{longtable}{|p{\textwidth * 1}|}
\hline
\rowcolor{gvm_log}{\color{white}{Log (CVSS: 0.0) }}\\
\rowcolor{gvm_log}{\color{white}{NVT: SSL/TLS: FTP 'AUTH TLS' Command Detection}}\\
\hline
\endfirsthead
\hfill\ldots continued from previous page \ldots \\
\hline
\endhead
\hline
\ldots continues on next page \ldots \\
\endfoot
\hline
\endlastfoot
\\
\textbf{Summary}\\
Checks if the remote FTP server supports SSL/TLS (FTPS) with the 'AUTH TLS' command.\\

        \hline
        \\
\textbf{Vulnerability Detection Result}\\
\rowcolor{white}{\verb=The remote FTP server supports TLS (FTPS) with the 'AUTH TLS' command.=}\\

          \hline
          \\
\textbf{Solution:}\\
\\


        \hline
        \\
\textbf{Log Method}\\
Details:
\rowcolor{white}{\verb=SSL/TLS: FTP 'AUTH TLS' Command Detection=}\\
OID:1.3.6.1.4.1.25623.1.0.105009\\
Version used:
\rowcolor{white}{\verb=2020-08-24T08:40:10Z=}\\

      \hline
      \\
\textbf{References}\\
\rowcolor{white}{\verb=url: https://tools.ietf.org/html/rfc4217=}\\
\end{longtable}

\begin{footnotesize}\hyperref[host:192.168.178.1]{[ return to 192.168.178.1 ]}\end{footnotesize}
\subsubsection{Log 53/tcp}
\label{port:192.168.178.1 53/tcp Log}

\begin{longtable}{|p{\textwidth * 1}|}
\hline
\rowcolor{gvm_log}{\color{white}{Log (CVSS: 0.0) }}\\
\rowcolor{gvm_log}{\color{white}{NVT: Services}}\\
\hline
\endfirsthead
\hfill\ldots continued from previous page \ldots \\
\hline
\endhead
\hline
\ldots continues on next page \ldots \\
\endfoot
\hline
\endlastfoot
\\
\textbf{Summary}\\
This routine attempts to guess which service is running on the
  remote ports. For instance, it searches for a web server which could listen on another port than
  80 or 443 and makes this information available for other check routines.\\

        \hline
        \\
\textbf{Vulnerability Detection Result}\\
\rowcolor{white}{\verb=The service closed the connection after 3 seconds without sending any data=}\\
\rowcolor{white}{\verb=It might be protected by some TCP wrapper=}\\

          \hline
          \\
\textbf{Solution:}\\
\\


        \hline
        \\
\textbf{Log Method}\\
Details:
\rowcolor{white}{\verb=Services=}\\
OID:1.3.6.1.4.1.25623.1.0.10330\\
Version used:
\rowcolor{white}{\verb=2021-03-15T10:42:03Z=}\\
\end{longtable}

\begin{footnotesize}\hyperref[host:192.168.178.1]{[ return to 192.168.178.1 ]}\end{footnotesize}
\subsubsection{Log 8183/tcp}
\label{port:192.168.178.1 8183/tcp Log}

\begin{longtable}{|p{\textwidth * 1}|}
\hline
\rowcolor{gvm_log}{\color{white}{Log (CVSS: 0.0) }}\\
\rowcolor{gvm_log}{\color{white}{NVT: Services}}\\
\hline
\endfirsthead
\hfill\ldots continued from previous page \ldots \\
\hline
\endhead
\hline
\ldots continues on next page \ldots \\
\endfoot
\hline
\endlastfoot
\\
\textbf{Summary}\\
This routine attempts to guess which service is running on the
  remote ports. For instance, it searches for a web server which could listen on another port than
  80 or 443 and makes this information available for other check routines.\\

        \hline
        \\
\textbf{Vulnerability Detection Result}\\
\rowcolor{white}{\verb=A web server is running on this port=}\\

          \hline
          \\
\textbf{Solution:}\\
\\


        \hline
        \\
\textbf{Log Method}\\
Details:
\rowcolor{white}{\verb=Services=}\\
OID:1.3.6.1.4.1.25623.1.0.10330\\
Version used:
\rowcolor{white}{\verb=2021-03-15T10:42:03Z=}\\
\end{longtable}

\begin{footnotesize}\hyperref[host:192.168.178.1]{[ return to 192.168.178.1 ]}\end{footnotesize}
\subsubsection{Log general/tcp}
\label{port:192.168.178.1 general/tcp Log}

\begin{longtable}{|p{\textwidth * 1}|}
\hline
\rowcolor{gvm_log}{\color{white}{Log (CVSS: 0.0) }}\\
\rowcolor{gvm_log}{\color{white}{NVT: SSL/TLS: Hostname discovery from server certificate}}\\
\hline
\endfirsthead
\hfill\ldots continued from previous page \ldots \\
\hline
\endhead
\hline
\ldots continues on next page \ldots \\
\endfoot
\hline
\endlastfoot
\\
\textbf{Summary}\\
It was possible to discover an additional hostname
  of this server from its certificate Common or Subject Alt Name.\\

        \hline
        \\
\textbf{Vulnerability Detection Result}\\
\rowcolor{white}{\verb=The following additional and resolvable hostnames were detected:=}\\
\rowcolor{white}{\verb=Breuelstrasse11=}\\
\rowcolor{white}{\verb=fritz.nas=}\\
\rowcolor{white}{\verb=myfritz.box=}\\
\rowcolor{white}{\verb=www.fritz.box=}\\
\rowcolor{white}{\verb=www.fritz.nas=}\\
\rowcolor{white}{\verb=www.myfritz.box=}\\

          \hline
          \\
\textbf{Solution:}\\
\\


        \hline
        \\
\textbf{Log Method}\\
Details:
\rowcolor{white}{\verb=SSL/TLS: Hostname discovery from server certificate=}\\
OID:1.3.6.1.4.1.25623.1.0.111010\\
Version used:
\rowcolor{white}{\verb=2021-11-22T15:32:39Z=}\\
\end{longtable}

\begin{longtable}{|p{\textwidth * 1}|}
\hline
\rowcolor{gvm_log}{\color{white}{Log (CVSS: 0.0) }}\\
\rowcolor{gvm_log}{\color{white}{NVT: AVM FRITZ!Box / FRITZ!OS Detection Consolidation}}\\
\hline
\endfirsthead
\hfill\ldots continued from previous page \ldots \\
\hline
\endhead
\hline
\ldots continues on next page \ldots \\
\endfoot
\hline
\endlastfoot
\\
\textbf{Summary}\\
Consolidation of AVM FRITZ!Box and FRITZ!OS detections.\\

        \hline
        \\
\textbf{Vulnerability Detection Result}\\
\rowcolor{white}{\verb=Detected AVM FRITZ!OS=}\\
\rowcolor{white}{\verb=Version:       unknown=}\\
\rowcolor{white}{\verb=Location:      /=}\\
\rowcolor{white}{\verb=CPE:           cpe:/o:avm:fritz%21_os=}\\
\rowcolor{white}{\verb=Detected AVM FRITZ!Box 7490=}\\
\rowcolor{white}{\verb=Location:      /=}\\
\rowcolor{white}{\verb=CPE:           cpe:/h:avm:fritzbox:7490=}\\
\rowcolor{white}{\verb=Exposed services:=}\\
\rowcolor{white}{\verb=HTTP(s) on port 80/tcp=}\\
\rowcolor{white}{\verb=HTTP(s) on port 443/tcp=}\\
\rowcolor{white}{\verb=SIP on port 5060/tcp=}\\
\rowcolor{white}{\verb=Banner: FRITZ!OS=}\\
\rowcolor{white}{\verb=FTP on port 21/ftp=}\\
\rowcolor{white}{\verb=Banner: 220 FRITZ!Box7490 FTP server ready.=}\\

          \hline
          \\
\textbf{Solution:}\\
\\


        \hline
        \\
\textbf{Log Method}\\
Details:
\rowcolor{white}{\verb=AVM FRITZ!Box / FRITZ!OS Detection Consolidation=}\\
OID:1.3.6.1.4.1.25623.1.0.103910\\
Version used:
\rowcolor{white}{\verb=2022-03-28T10:48:38Z=}\\
\end{longtable}

\begin{longtable}{|p{\textwidth * 1}|}
\hline
\rowcolor{gvm_log}{\color{white}{Log (CVSS: 0.0) }}\\
\rowcolor{gvm_log}{\color{white}{NVT: OS Detection Consolidation and Reporting}}\\
\hline
\endfirsthead
\hfill\ldots continued from previous page \ldots \\
\hline
\endhead
\hline
\ldots continues on next page \ldots \\
\endfoot
\hline
\endlastfoot
\\
\textbf{Summary}\\
This script consolidates the OS information detected by several
  VTs and tries to find the best matching OS.\\
  Furthermore it reports all previously collected information leading to this best matching OS. It
  also reports possible additional information which might help to improve the OS detection.\\
  If any of this information is wrong or could be improved please consider to report these to the
  referenced community portal.\\

        \hline
        \\
\textbf{Vulnerability Detection Result}\\
\rowcolor{white}{\verb=Best matching OS:=}\\
\rowcolor{white}{\verb=OS:           AVM FRITZ!OS=}\\
\rowcolor{white}{\verb=CPE:          cpe:/o:avm:fritz%21_os=}\\
\rowcolor{white}{\verb=Found by NVT: 1.3.6.1.4.1.25623.1.0.103910 (AVM FRITZ!Box / FRITZ!OS Detection C=}\\
\rowcolor{white}{$\hookrightarrow$\verb=onsolidation)=}\\
\rowcolor{white}{\verb=Setting key "Host/runs_unixoide" based on this information=}\\
\rowcolor{white}{\verb=Other OS detections (in order of reliability):=}\\
\rowcolor{white}{\verb=OS:           AVM FRITZ!OS=}\\
\rowcolor{white}{\verb=CPE:          cpe:/o:avm:fritz%21_os=}\\
\rowcolor{white}{\verb=Found by NVT: 1.3.6.1.4.1.25623.1.0.108201 (Operating System (OS) Detection (SIP=}\\
\rowcolor{white}{$\hookrightarrow$\verb=))=}\\
\rowcolor{white}{\verb=Concluded from SIP server banner on port 5060/tcp: User-Agent Banner: FRITZ!OS=}\\

          \hline
          \\
\textbf{Solution:}\\
\\


        \hline
        \\
\textbf{Log Method}\\
Details:
\rowcolor{white}{\verb=OS Detection Consolidation and Reporting=}\\
OID:1.3.6.1.4.1.25623.1.0.105937\\
Version used:
\rowcolor{white}{\verb=2022-04-05T09:27:51Z=}\\

      \hline
      \\
\textbf{References}\\
\rowcolor{white}{\verb=url: https://community.greenbone.net/c/vulnerability-tests=}\\
\end{longtable}

\begin{longtable}{|p{\textwidth * 1}|}
\hline
\rowcolor{gvm_log}{\color{white}{Log (CVSS: 0.0) }}\\
\rowcolor{gvm_log}{\color{white}{NVT: Hostname Determination Reporting}}\\
\hline
\endfirsthead
\hfill\ldots continued from previous page \ldots \\
\hline
\endhead
\hline
\ldots continues on next page \ldots \\
\endfoot
\hline
\endlastfoot
\\
\textbf{Summary}\\
The script reports information on how the hostname
  of the target was determined.\\

        \hline
        \\
\textbf{Vulnerability Detection Result}\\
\rowcolor{white}{\verb=Hostname determination for IP 192.168.178.1:=}\\
\rowcolor{white}{\verb=Hostname|Source=}\\
\rowcolor{white}{\verb=breuelstrasse11|SSL/TLS server certificate=}\\
\rowcolor{white}{\verb=fritz.box|Reverse-DNS=}\\
\rowcolor{white}{\verb=fritz.nas|SSL/TLS server certificate=}\\
\rowcolor{white}{\verb=myfritz.box|SSL/TLS server certificate=}\\
\rowcolor{white}{\verb=www.fritz.box|SSL/TLS server certificate=}\\
\rowcolor{white}{\verb=www.fritz.nas|SSL/TLS server certificate=}\\
\rowcolor{white}{\verb=www.myfritz.box|SSL/TLS server certificate=}\\

          \hline
          \\
\textbf{Solution:}\\
\\


        \hline
        \\
\textbf{Log Method}\\
Details:
\rowcolor{white}{\verb=Hostname Determination Reporting=}\\
OID:1.3.6.1.4.1.25623.1.0.108449\\
Version used:
\rowcolor{white}{\verb=2018-11-19T11:11:31Z=}\\
\end{longtable}

\begin{footnotesize}\hyperref[host:192.168.178.1]{[ return to 192.168.178.1 ]}\end{footnotesize}
\subsubsection{Log 8184/tcp}
\label{port:192.168.178.1 8184/tcp Log}

\begin{longtable}{|p{\textwidth * 1}|}
\hline
\rowcolor{gvm_log}{\color{white}{Log (CVSS: 0.0) }}\\
\rowcolor{gvm_log}{\color{white}{NVT: Services}}\\
\hline
\endfirsthead
\hfill\ldots continued from previous page \ldots \\
\hline
\endhead
\hline
\ldots continues on next page \ldots \\
\endfoot
\hline
\endlastfoot
\\
\textbf{Summary}\\
This routine attempts to guess which service is running on the
  remote ports. For instance, it searches for a web server which could listen on another port than
  80 or 443 and makes this information available for other check routines.\\

        \hline
        \\
\textbf{Vulnerability Detection Result}\\
\rowcolor{white}{\verb=A web server is running on this port=}\\

          \hline
          \\
\textbf{Solution:}\\
\\


        \hline
        \\
\textbf{Log Method}\\
Details:
\rowcolor{white}{\verb=Services=}\\
OID:1.3.6.1.4.1.25623.1.0.10330\\
Version used:
\rowcolor{white}{\verb=2021-03-15T10:42:03Z=}\\
\end{longtable}

\begin{footnotesize}\hyperref[host:192.168.178.1]{[ return to 192.168.178.1 ]}\end{footnotesize}
\subsubsection{Log 49000/tcp}
\label{port:192.168.178.1 49000/tcp Log}

\begin{longtable}{|p{\textwidth * 1}|}
\hline
\rowcolor{gvm_log}{\color{white}{Log (CVSS: 0.0) }}\\
\rowcolor{gvm_log}{\color{white}{NVT: Services}}\\
\hline
\endfirsthead
\hfill\ldots continued from previous page \ldots \\
\hline
\endhead
\hline
\ldots continues on next page \ldots \\
\endfoot
\hline
\endlastfoot
\\
\textbf{Summary}\\
This routine attempts to guess which service is running on the
  remote ports. For instance, it searches for a web server which could listen on another port than
  80 or 443 and makes this information available for other check routines.\\

        \hline
        \\
\textbf{Vulnerability Detection Result}\\
\rowcolor{white}{\verb=A web server is running on this port=}\\

          \hline
          \\
\textbf{Solution:}\\
\\


        \hline
        \\
\textbf{Log Method}\\
Details:
\rowcolor{white}{\verb=Services=}\\
OID:1.3.6.1.4.1.25623.1.0.10330\\
Version used:
\rowcolor{white}{\verb=2021-03-15T10:42:03Z=}\\
\end{longtable}

\begin{footnotesize}\hyperref[host:192.168.178.1]{[ return to 192.168.178.1 ]}\end{footnotesize}
\subsubsection{Log 80/tcp}
\label{port:192.168.178.1 80/tcp Log}

\begin{longtable}{|p{\textwidth * 1}|}
\hline
\rowcolor{gvm_log}{\color{white}{Log (CVSS: 0.0) }}\\
\rowcolor{gvm_log}{\color{white}{NVT: Services}}\\
\hline
\endfirsthead
\hfill\ldots continued from previous page \ldots \\
\hline
\endhead
\hline
\ldots continues on next page \ldots \\
\endfoot
\hline
\endlastfoot
\\
\textbf{Summary}\\
This routine attempts to guess which service is running on the
  remote ports. For instance, it searches for a web server which could listen on another port than
  80 or 443 and makes this information available for other check routines.\\

        \hline
        \\
\textbf{Vulnerability Detection Result}\\
\rowcolor{white}{\verb=A web server is running on this port=}\\

          \hline
          \\
\textbf{Solution:}\\
\\


        \hline
        \\
\textbf{Log Method}\\
Details:
\rowcolor{white}{\verb=Services=}\\
OID:1.3.6.1.4.1.25623.1.0.10330\\
Version used:
\rowcolor{white}{\verb=2021-03-15T10:42:03Z=}\\
\end{longtable}

\begin{longtable}{|p{\textwidth * 1}|}
\hline
\rowcolor{gvm_log}{\color{white}{Log (CVSS: 0.0) }}\\
\rowcolor{gvm_log}{\color{white}{NVT: Response Time / No 404 Error Code Check}}\\
\hline
\endfirsthead
\hfill\ldots continued from previous page \ldots \\
\hline
\endhead
\hline
\ldots continues on next page \ldots \\
\endfoot
\hline
\endlastfoot
\\
\textbf{Summary}\\
This VT tests if the remote web server does not reply with a 404
  error code and checks if it is replying to the scanners requests in a reasonable amount of time.\\

        \hline
        \\
\textbf{Vulnerability Detection Result}\\
\rowcolor{white}{\verb=The service is responding with a 200 HTTP status code to non-existent files/urls=}\\
\rowcolor{white}{$\hookrightarrow$\verb=. The following pattern is used to work around possible false detections:=}\\
\rowcolor{white}{\verb=-----=}\\
\rowcolor{white}{\verb=MyFRITZ!=}\\
\rowcolor{white}{\verb=-----=}\\

          \hline
          \\
\textbf{Solution:}\\
\\

          \hline
          \\
\textbf{Vulnerability Insight}\\
This web server might show the following issues:\\
  - it is [mis]configured in that it does not return '404 Not Found' error codes when a non-existent
  file is requested, perhaps returning a site map, search page, authentication page or redirect instead.\\
  The Scanner might enabled some counter measures for that, however they might be insufficient. If a
  great number of security issues are reported for this port, they might not all be accurate.\\
  - it doesn't response in a reasonable amount of time to various HTTP requests sent by this VT.\\
  In order to keep the scan total time to a reasonable amount, the remote web server might not be
  tested. If the remote server should be tested it has to be fixed to have it reply to the scanners
  requests in a reasonable amount of time.\\
  Alternatively the 'Maximum response time (in seconds)' preference could be raised to a higher
  value if longer scan times are accepted.\\


        \hline
        \\
\textbf{Log Method}\\
Details:
\rowcolor{white}{\verb=Response Time / No 404 Error Code Check=}\\
OID:1.3.6.1.4.1.25623.1.0.10386\\
Version used:
\rowcolor{white}{\verb=2020-11-27T13:32:50Z=}\\
\end{longtable}

\begin{footnotesize}\hyperref[host:192.168.178.1]{[ return to 192.168.178.1 ]}\end{footnotesize}
\subsubsection{Log 443/tcp}
\label{port:192.168.178.1 443/tcp Log}

\begin{longtable}{|p{\textwidth * 1}|}
\hline
\rowcolor{gvm_log}{\color{white}{Log (CVSS: 0.0) }}\\
\rowcolor{gvm_log}{\color{white}{NVT: Services}}\\
\hline
\endfirsthead
\hfill\ldots continued from previous page \ldots \\
\hline
\endhead
\hline
\ldots continues on next page \ldots \\
\endfoot
\hline
\endlastfoot
\\
\textbf{Summary}\\
This routine attempts to guess which service is running on the
  remote ports. For instance, it searches for a web server which could listen on another port than
  80 or 443 and makes this information available for other check routines.\\

        \hline
        \\
\textbf{Vulnerability Detection Result}\\
\rowcolor{white}{\verb=A TLScustom server answered on this port=}\\

          \hline
          \\
\textbf{Solution:}\\
\\


        \hline
        \\
\textbf{Log Method}\\
Details:
\rowcolor{white}{\verb=Services=}\\
OID:1.3.6.1.4.1.25623.1.0.10330\\
Version used:
\rowcolor{white}{\verb=2021-03-15T10:42:03Z=}\\
\end{longtable}

\begin{longtable}{|p{\textwidth * 1}|}
\hline
\rowcolor{gvm_log}{\color{white}{Log (CVSS: 0.0) }}\\
\rowcolor{gvm_log}{\color{white}{NVT: Services}}\\
\hline
\endfirsthead
\hfill\ldots continued from previous page \ldots \\
\hline
\endhead
\hline
\ldots continues on next page \ldots \\
\endfoot
\hline
\endlastfoot
\\
\textbf{Summary}\\
This routine attempts to guess which service is running on the
  remote ports. For instance, it searches for a web server which could listen on another port than
  80 or 443 and makes this information available for other check routines.\\

        \hline
        \\
\textbf{Vulnerability Detection Result}\\
\rowcolor{white}{\verb=A web server is running on this port through SSL=}\\

          \hline
          \\
\textbf{Solution:}\\
\\


        \hline
        \\
\textbf{Log Method}\\
Details:
\rowcolor{white}{\verb=Services=}\\
OID:1.3.6.1.4.1.25623.1.0.10330\\
Version used:
\rowcolor{white}{\verb=2021-03-15T10:42:03Z=}\\
\end{longtable}

\begin{longtable}{|p{\textwidth * 1}|}
\hline
\rowcolor{gvm_log}{\color{white}{Log (CVSS: 0.0) }}\\
\rowcolor{gvm_log}{\color{white}{NVT: SSL/TLS: Version Detection}}\\
\hline
\endfirsthead
\hfill\ldots continued from previous page \ldots \\
\hline
\endhead
\hline
\ldots continues on next page \ldots \\
\endfoot
\hline
\endlastfoot
\\
\textbf{Summary}\\
Enumeration and reporting of SSL/TLS protocol versions supported
  by a remote service.\\

        \hline
        \\
\textbf{Vulnerability Detection Result}\\
\rowcolor{white}{\verb=The remote SSL/TLS service supports the following SSL/TLS protocol version(s):=}\\
\rowcolor{white}{\verb=TLSv1.2=}\\
\rowcolor{white}{\verb=TLSv1.3=}\\

          \hline
          \\
\textbf{Solution:}\\
\\


        \hline
        \\
\textbf{Log Method}\\
Sends multiple connection requests to the remote service and
  attempts to determine the SSL/TLS protocol versions supported by the service from the replies.\\
  Note: The supported SSL/TLS protocol versions included in the report of this VT are reported
  independently from the allowed / supported SSL/TLS ciphers.\\
Details:
\rowcolor{white}{\verb=SSL/TLS: Version Detection=}\\
OID:1.3.6.1.4.1.25623.1.0.105782\\
Version used:
\rowcolor{white}{\verb=2021-12-06T15:42:24Z=}\\
\end{longtable}

\begin{longtable}{|p{\textwidth * 1}|}
\hline
\rowcolor{gvm_log}{\color{white}{Log (CVSS: 0.0) }}\\
\rowcolor{gvm_log}{\color{white}{NVT: SSL/TLS: Collect and Report Certificate Details}}\\
\hline
\endfirsthead
\hfill\ldots continued from previous page \ldots \\
\hline
\endhead
\hline
\ldots continues on next page \ldots \\
\endfoot
\hline
\endlastfoot
\\
\textbf{Summary}\\
This script collects and reports the details of all SSL/TLS
  certificates.\\
  This data will be used by other tests to verify server certificates.\\

        \hline
        \\
\textbf{Vulnerability Detection Result}\\
\rowcolor{white}{\verb=The following certificate details of the remote service were collected.=}\\
\rowcolor{white}{\verb=Certificate details:=}\\
\rowcolor{white}{\verb=fingerprint (SHA-1)             | CFB7FD5D27141810D8F9CE8D1BD74B12E6E26780=}\\
\rowcolor{white}{\verb=fingerprint (SHA-256)           | 5ECA6F620C8BC526BEA6FA3E37282C1FD1E98BDC85CAB6=}\\
\rowcolor{white}{$\hookrightarrow$\verb=F55C8C4DEDDEEEA99C=}\\
\rowcolor{white}{\verb=issued by                       | CN=\verb-=-\verb=92.192.143.206=}\\
\rowcolor{white}{\verb=public key algorithm            | RSA=}\\
\rowcolor{white}{\verb=public key size (bits)          | 2048=}\\
\rowcolor{white}{\verb=serial                          | 58C6BD4C62AC35CA3C7D9EA7569675FA4263ECDF=}\\
\rowcolor{white}{\verb=signature algorithm             | sha256WithRSAEncryption=}\\
\rowcolor{white}{\verb=subject                         | CN=\verb-=-\verb=92.192.143.206=}\\
\rowcolor{white}{\verb=subject alternative names (SAN) | Breuelstrasse11, fritz.nas, www.fritz.nas, fri=}\\
\rowcolor{white}{$\hookrightarrow$\verb=tz.box, www.fritz.box, myfritz.box, www.myfritz.box=}\\
\rowcolor{white}{\verb=valid from                      | 2022-04-22 05:53:04 UTC=}\\
\rowcolor{white}{\verb=valid until                     | 2038-01-15 05:53:04 UTC=}\\

          \hline
          \\
\textbf{Solution:}\\
\\


        \hline
        \\
\textbf{Log Method}\\
Details:
\rowcolor{white}{\verb=SSL/TLS: Collect and Report Certificate Details=}\\
OID:1.3.6.1.4.1.25623.1.0.103692\\
Version used:
\rowcolor{white}{\verb=2021-12-10T12:48:00Z=}\\
\end{longtable}

\begin{longtable}{|p{\textwidth * 1}|}
\hline
\rowcolor{gvm_log}{\color{white}{Log (CVSS: 0.0) }}\\
\rowcolor{gvm_log}{\color{white}{NVT: Response Time / No 404 Error Code Check}}\\
\hline
\endfirsthead
\hfill\ldots continued from previous page \ldots \\
\hline
\endhead
\hline
\ldots continues on next page \ldots \\
\endfoot
\hline
\endlastfoot
\\
\textbf{Summary}\\
This VT tests if the remote web server does not reply with a 404
  error code and checks if it is replying to the scanners requests in a reasonable amount of time.\\

        \hline
        \\
\textbf{Vulnerability Detection Result}\\
\rowcolor{white}{\verb=The remote web server is very slow - it took 83 seconds (Maximum response time c=}\\
\rowcolor{white}{$\hookrightarrow$\verb=onfigured in 'Response Time / No 404 Error Code Check' (OID: 1.3.6.1.4.1.25623=}\\
\rowcolor{white}{$\hookrightarrow$\verb=.1.0.10386) preferences: 60 seconds) to execute the plugin no404.nasl (it usua=}\\
\rowcolor{white}{$\hookrightarrow$\verb=lly only takes a few seconds).=}\\
\rowcolor{white}{\verb=In order to keep the scan total time to a reasonable amount, the remote web serv=}\\
\rowcolor{white}{$\hookrightarrow$\verb=er has not been tested.=}\\
\rowcolor{white}{\verb=If the remote server should be tested it has to be fixed to have it reply to the=}\\
\rowcolor{white}{$\hookrightarrow$\verb= scanners requests in a reasonable amount of time. Alternatively the 'Maximum =}\\
\rowcolor{white}{$\hookrightarrow$\verb=response time (in seconds)' preference could be raised to a higher value if lo=}\\
\rowcolor{white}{$\hookrightarrow$\verb=nger scan times are accepted.=}\\

          \hline
          \\
\textbf{Solution:}\\
\\

          \hline
          \\
\textbf{Vulnerability Insight}\\
This web server might show the following issues:\\
  - it is [mis]configured in that it does not return '404 Not Found' error codes when a non-existent
  file is requested, perhaps returning a site map, search page, authentication page or redirect instead.\\
  The Scanner might enabled some counter measures for that, however they might be insufficient. If a
  great number of security issues are reported for this port, they might not all be accurate.\\
  - it doesn't response in a reasonable amount of time to various HTTP requests sent by this VT.\\
  In order to keep the scan total time to a reasonable amount, the remote web server might not be
  tested. If the remote server should be tested it has to be fixed to have it reply to the scanners
  requests in a reasonable amount of time.\\
  Alternatively the 'Maximum response time (in seconds)' preference could be raised to a higher
  value if longer scan times are accepted.\\


        \hline
        \\
\textbf{Log Method}\\
Details:
\rowcolor{white}{\verb=Response Time / No 404 Error Code Check=}\\
OID:1.3.6.1.4.1.25623.1.0.10386\\
Version used:
\rowcolor{white}{\verb=2020-11-27T13:32:50Z=}\\
\end{longtable}

\begin{footnotesize}\hyperref[host:192.168.178.1]{[ return to 192.168.178.1 ]}\end{footnotesize}
\subsection{192.168.178.24}
\label{host:192.168.178.24}

\begin{tabular}{ll}
Host scan start&Fri Apr 22 12:21:26 2022 UTC\\
Host scan end&Fri Apr 22 12:22:03 2022 UTC\\
\end{tabular}

\begin{longtable}{|l|l|}
\hline
\rowcolor{gvm_report}Service (Port)&Threat Level\\
\hline
\endfirsthead
\multicolumn{2}{l}{\hfill\ldots (continued) \ldots}\\
\hline
\rowcolor{gvm_report}Service (Port)&Threat Level\\
\hline
\endhead
\hline
\multicolumn{2}{l}{\ldots (continues) \ldots}\\
\endfoot
\hline
\endlastfoot
\hline
\hyperref[port:192.168.178.24 general/tcp Log]{general/tcp}&Log\\
\hline
\end{longtable}


%\subsection*{Security Issues and Fixes -- 192.168.178.24}

\subsubsection{Log general/tcp}
\label{port:192.168.178.24 general/tcp Log}

\begin{longtable}{|p{\textwidth * 1}|}
\hline
\rowcolor{gvm_log}{\color{white}{Log (CVSS: 0.0) }}\\
\rowcolor{gvm_log}{\color{white}{NVT: Hostname Determination Reporting}}\\
\hline
\endfirsthead
\hfill\ldots continued from previous page \ldots \\
\hline
\endhead
\hline
\ldots continues on next page \ldots \\
\endfoot
\hline
\endlastfoot
\\
\textbf{Summary}\\
The script reports information on how the hostname
  of the target was determined.\\

        \hline
        \\
\textbf{Vulnerability Detection Result}\\
\rowcolor{white}{\verb=Hostname determination for IP 192.168.178.24:=}\\
\rowcolor{white}{\verb=Hostname|Source=}\\
\rowcolor{white}{\verb=192.168.178.24|IP-address=}\\

          \hline
          \\
\textbf{Solution:}\\
\\


        \hline
        \\
\textbf{Log Method}\\
Details:
\rowcolor{white}{\verb=Hostname Determination Reporting=}\\
OID:1.3.6.1.4.1.25623.1.0.108449\\
Version used:
\rowcolor{white}{\verb=2018-11-19T11:11:31Z=}\\
\end{longtable}

\begin{longtable}{|p{\textwidth * 1}|}
\hline
\rowcolor{gvm_log}{\color{white}{Log (CVSS: 0.0) }}\\
\rowcolor{gvm_log}{\color{white}{NVT: OS Detection Consolidation and Reporting}}\\
\hline
\endfirsthead
\hfill\ldots continued from previous page \ldots \\
\hline
\endhead
\hline
\ldots continues on next page \ldots \\
\endfoot
\hline
\endlastfoot
\\
\textbf{Summary}\\
This script consolidates the OS information detected by several
  VTs and tries to find the best matching OS.\\
  Furthermore it reports all previously collected information leading to this best matching OS. It
  also reports possible additional information which might help to improve the OS detection.\\
  If any of this information is wrong or could be improved please consider to report these to the
  referenced community portal.\\

        \hline
        \\
\textbf{Vulnerability Detection Result}\\
\rowcolor{white}{\verb=Best matching OS:=}\\
\rowcolor{white}{\verb=OS:           OpenBSD=}\\
\rowcolor{white}{\verb=CPE:          cpe:/o:openbsd:openbsd=}\\
\rowcolor{white}{\verb=Found by NVT: 1.3.6.1.4.1.25623.1.0.102002 (Operating System (OS) Detection (ICM=}\\
\rowcolor{white}{$\hookrightarrow$\verb=P))=}\\
\rowcolor{white}{\verb=Concluded from ICMP based OS fingerprint=}\\
\rowcolor{white}{\verb=Setting key "Host/runs_unixoide" based on this information=}\\
\rowcolor{white}{\verb=Other OS detections (in order of reliability):=}\\
\rowcolor{white}{\verb=OS:           HP UX=}\\
\rowcolor{white}{\verb=CPE:          cpe:/o:hp:hp-ux=}\\
\rowcolor{white}{\verb=Found by NVT: 1.3.6.1.4.1.25623.1.0.102002 (Operating System (OS) Detection (ICM=}\\
\rowcolor{white}{$\hookrightarrow$\verb=P))=}\\
\rowcolor{white}{\verb=Concluded from ICMP based OS fingerprint=}\\
\rowcolor{white}{\verb=OS:           Cisco IOS=}\\
\rowcolor{white}{\verb=CPE:          cpe:/o:cisco:ios=}\\
\rowcolor{white}{\verb=Found by NVT: 1.3.6.1.4.1.25623.1.0.102002 (Operating System (OS) Detection (ICM=}\\
\rowcolor{white}{$\hookrightarrow$\verb=P))=}\\
\rowcolor{white}{\verb=Concluded from ICMP based OS fingerprint=}\\
\rowcolor{white}{\verb=OS:           NetBSD=}\\
\rowcolor{white}{\verb=CPE:          cpe:/o:netbsd:netbsd=}\\
\rowcolor{white}{\verb=Found by NVT: 1.3.6.1.4.1.25623.1.0.102002 (Operating System (OS) Detection (ICM=}\\
\rowcolor{white}{$\hookrightarrow$\verb=P))=}\\
\rowcolor{white}{\verb=Concluded from ICMP based OS fingerprint=}\\

          \hline
          \\
\textbf{Solution:}\\
\\


        \hline
        \\
\textbf{Log Method}\\
Details:
\rowcolor{white}{\verb=OS Detection Consolidation and Reporting=}\\
OID:1.3.6.1.4.1.25623.1.0.105937\\
Version used:
\rowcolor{white}{\verb=2022-04-05T09:27:51Z=}\\

      \hline
      \\
\textbf{References}\\
\rowcolor{white}{\verb=url: https://community.greenbone.net/c/vulnerability-tests=}\\
\end{longtable}

\begin{footnotesize}\hyperref[host:192.168.178.24]{[ return to 192.168.178.24 ]}\end{footnotesize}
\subsection{192.168.178.23}
\label{host:192.168.178.23}

\begin{tabular}{ll}
Host scan start&Fri Apr 22 12:21:26 2022 UTC\\
Host scan end&Fri Apr 22 12:25:25 2022 UTC\\
\end{tabular}

\begin{longtable}{|l|l|}
\hline
\rowcolor{gvm_report}Service (Port)&Threat Level\\
\hline
\endfirsthead
\multicolumn{2}{l}{\hfill\ldots (continued) \ldots}\\
\hline
\rowcolor{gvm_report}Service (Port)&Threat Level\\
\hline
\endhead
\hline
\multicolumn{2}{l}{\ldots (continues) \ldots}\\
\endfoot
\hline
\endlastfoot
\hline
\hyperref[port:192.168.178.23 general/tcp Log]{general/tcp}&Log\\
\hline
\hyperref[port:192.168.178.23 8009/tcp Log]{8009/tcp}&Log\\
\hline
\end{longtable}


%\subsection*{Security Issues and Fixes -- 192.168.178.23}

\subsubsection{Log general/tcp}
\label{port:192.168.178.23 general/tcp Log}

\begin{longtable}{|p{\textwidth * 1}|}
\hline
\rowcolor{gvm_log}{\color{white}{Log (CVSS: 0.0) }}\\
\rowcolor{gvm_log}{\color{white}{NVT: OS Detection Consolidation and Reporting}}\\
\hline
\endfirsthead
\hfill\ldots continued from previous page \ldots \\
\hline
\endhead
\hline
\ldots continues on next page \ldots \\
\endfoot
\hline
\endlastfoot
\\
\textbf{Summary}\\
This script consolidates the OS information detected by several
  VTs and tries to find the best matching OS.\\
  Furthermore it reports all previously collected information leading to this best matching OS. It
  also reports possible additional information which might help to improve the OS detection.\\
  If any of this information is wrong or could be improved please consider to report these to the
  referenced community portal.\\

        \hline
        \\
\textbf{Vulnerability Detection Result}\\
\rowcolor{white}{\verb=Best matching OS:=}\\
\rowcolor{white}{\verb=OS:           Linux Kernel=}\\
\rowcolor{white}{\verb=CPE:          cpe:/o:linux:kernel=}\\
\rowcolor{white}{\verb=Found by NVT: 1.3.6.1.4.1.25623.1.0.102002 (Operating System (OS) Detection (ICM=}\\
\rowcolor{white}{$\hookrightarrow$\verb=P))=}\\
\rowcolor{white}{\verb=Concluded from ICMP based OS fingerprint=}\\
\rowcolor{white}{\verb=Setting key "Host/runs_unixoide" based on this information=}\\

          \hline
          \\
\textbf{Solution:}\\
\\


        \hline
        \\
\textbf{Log Method}\\
Details:
\rowcolor{white}{\verb=OS Detection Consolidation and Reporting=}\\
OID:1.3.6.1.4.1.25623.1.0.105937\\
Version used:
\rowcolor{white}{\verb=2022-04-05T09:27:51Z=}\\

      \hline
      \\
\textbf{References}\\
\rowcolor{white}{\verb=url: https://community.greenbone.net/c/vulnerability-tests=}\\
\end{longtable}

\begin{longtable}{|p{\textwidth * 1}|}
\hline
\rowcolor{gvm_log}{\color{white}{Log (CVSS: 0.0) }}\\
\rowcolor{gvm_log}{\color{white}{NVT: Hostname Determination Reporting}}\\
\hline
\endfirsthead
\hfill\ldots continued from previous page \ldots \\
\hline
\endhead
\hline
\ldots continues on next page \ldots \\
\endfoot
\hline
\endlastfoot
\\
\textbf{Summary}\\
The script reports information on how the hostname
  of the target was determined.\\

        \hline
        \\
\textbf{Vulnerability Detection Result}\\
\rowcolor{white}{\verb=Hostname determination for IP 192.168.178.23:=}\\
\rowcolor{white}{\verb=Hostname|Source=}\\
\rowcolor{white}{\verb=amazon-f0c31363c.fritz.box|Reverse-DNS=}\\

          \hline
          \\
\textbf{Solution:}\\
\\


        \hline
        \\
\textbf{Log Method}\\
Details:
\rowcolor{white}{\verb=Hostname Determination Reporting=}\\
OID:1.3.6.1.4.1.25623.1.0.108449\\
Version used:
\rowcolor{white}{\verb=2018-11-19T11:11:31Z=}\\
\end{longtable}

\begin{footnotesize}\hyperref[host:192.168.178.23]{[ return to 192.168.178.23 ]}\end{footnotesize}
\subsubsection{Log 8009/tcp}
\label{port:192.168.178.23 8009/tcp Log}

\begin{longtable}{|p{\textwidth * 1}|}
\hline
\rowcolor{gvm_log}{\color{white}{Log (CVSS: 0.0) }}\\
\rowcolor{gvm_log}{\color{white}{NVT: Services}}\\
\hline
\endfirsthead
\hfill\ldots continued from previous page \ldots \\
\hline
\endhead
\hline
\ldots continues on next page \ldots \\
\endfoot
\hline
\endlastfoot
\\
\textbf{Summary}\\
This routine attempts to guess which service is running on the
  remote ports. For instance, it searches for a web server which could listen on another port than
  80 or 443 and makes this information available for other check routines.\\

        \hline
        \\
\textbf{Vulnerability Detection Result}\\
\rowcolor{white}{\verb=A web server is running on this port=}\\

          \hline
          \\
\textbf{Solution:}\\
\\


        \hline
        \\
\textbf{Log Method}\\
Details:
\rowcolor{white}{\verb=Services=}\\
OID:1.3.6.1.4.1.25623.1.0.10330\\
Version used:
\rowcolor{white}{\verb=2021-03-15T10:42:03Z=}\\
\end{longtable}

\begin{footnotesize}\hyperref[host:192.168.178.23]{[ return to 192.168.178.23 ]}\end{footnotesize}
\subsection{192.168.178.29}
\label{host:192.168.178.29}

\begin{tabular}{ll}
Host scan start&Fri Apr 22 12:21:26 2022 UTC\\
Host scan end&Fri Apr 22 12:34:53 2022 UTC\\
\end{tabular}

\begin{longtable}{|l|l|}
\hline
\rowcolor{gvm_report}Service (Port)&Threat Level\\
\hline
\endfirsthead
\multicolumn{2}{l}{\hfill\ldots (continued) \ldots}\\
\hline
\rowcolor{gvm_report}Service (Port)&Threat Level\\
\hline
\endhead
\hline
\multicolumn{2}{l}{\ldots (continues) \ldots}\\
\endfoot
\hline
\endlastfoot
\hline
\hyperref[port:192.168.178.29 general/tcp Log]{general/tcp}&Log\\
\hline
\hyperref[port:192.168.178.29 80/tcp Log]{80/tcp}&Log\\
\hline
\end{longtable}


%\subsection*{Security Issues and Fixes -- 192.168.178.29}

\subsubsection{Log general/tcp}
\label{port:192.168.178.29 general/tcp Log}

\begin{longtable}{|p{\textwidth * 1}|}
\hline
\rowcolor{gvm_log}{\color{white}{Log (CVSS: 0.0) }}\\
\rowcolor{gvm_log}{\color{white}{NVT: OS Detection Consolidation and Reporting}}\\
\hline
\endfirsthead
\hfill\ldots continued from previous page \ldots \\
\hline
\endhead
\hline
\ldots continues on next page \ldots \\
\endfoot
\hline
\endlastfoot
\\
\textbf{Summary}\\
This script consolidates the OS information detected by several
  VTs and tries to find the best matching OS.\\
  Furthermore it reports all previously collected information leading to this best matching OS. It
  also reports possible additional information which might help to improve the OS detection.\\
  If any of this information is wrong or could be improved please consider to report these to the
  referenced community portal.\\

        \hline
        \\
\textbf{Vulnerability Detection Result}\\
\rowcolor{white}{\verb=Best matching OS:=}\\
\rowcolor{white}{\verb=OS:           HP JetDirect=}\\
\rowcolor{white}{\verb=CPE:          cpe:/h:hp:jetdirect=}\\
\rowcolor{white}{\verb=Found by NVT: 1.3.6.1.4.1.25623.1.0.102002 (Operating System (OS) Detection (ICM=}\\
\rowcolor{white}{$\hookrightarrow$\verb=P))=}\\
\rowcolor{white}{\verb=Concluded from ICMP based OS fingerprint=}\\
\rowcolor{white}{\verb=Setting key "Host/runs_unixoide" based on this information=}\\

          \hline
          \\
\textbf{Solution:}\\
\\


        \hline
        \\
\textbf{Log Method}\\
Details:
\rowcolor{white}{\verb=OS Detection Consolidation and Reporting=}\\
OID:1.3.6.1.4.1.25623.1.0.105937\\
Version used:
\rowcolor{white}{\verb=2022-04-05T09:27:51Z=}\\

      \hline
      \\
\textbf{References}\\
\rowcolor{white}{\verb=url: https://community.greenbone.net/c/vulnerability-tests=}\\
\end{longtable}

\begin{longtable}{|p{\textwidth * 1}|}
\hline
\rowcolor{gvm_log}{\color{white}{Log (CVSS: 0.0) }}\\
\rowcolor{gvm_log}{\color{white}{NVT: Hostname Determination Reporting}}\\
\hline
\endfirsthead
\hfill\ldots continued from previous page \ldots \\
\hline
\endhead
\hline
\ldots continues on next page \ldots \\
\endfoot
\hline
\endlastfoot
\\
\textbf{Summary}\\
The script reports information on how the hostname
  of the target was determined.\\

        \hline
        \\
\textbf{Vulnerability Detection Result}\\
\rowcolor{white}{\verb=Hostname determination for IP 192.168.178.29:=}\\
\rowcolor{white}{\verb=Hostname|Source=}\\
\rowcolor{white}{\verb=wohn-tl-sg108e.fritz.box|Reverse-DNS=}\\

          \hline
          \\
\textbf{Solution:}\\
\\


        \hline
        \\
\textbf{Log Method}\\
Details:
\rowcolor{white}{\verb=Hostname Determination Reporting=}\\
OID:1.3.6.1.4.1.25623.1.0.108449\\
Version used:
\rowcolor{white}{\verb=2018-11-19T11:11:31Z=}\\
\end{longtable}

\begin{footnotesize}\hyperref[host:192.168.178.29]{[ return to 192.168.178.29 ]}\end{footnotesize}
\subsubsection{Log 80/tcp}
\label{port:192.168.178.29 80/tcp Log}

\begin{longtable}{|p{\textwidth * 1}|}
\hline
\rowcolor{gvm_log}{\color{white}{Log (CVSS: 0.0) }}\\
\rowcolor{gvm_log}{\color{white}{NVT: Services}}\\
\hline
\endfirsthead
\hfill\ldots continued from previous page \ldots \\
\hline
\endhead
\hline
\ldots continues on next page \ldots \\
\endfoot
\hline
\endlastfoot
\\
\textbf{Summary}\\
This routine attempts to guess which service is running on the
  remote ports. For instance, it searches for a web server which could listen on another port than
  80 or 443 and makes this information available for other check routines.\\

        \hline
        \\
\textbf{Vulnerability Detection Result}\\
\rowcolor{white}{\verb=A web server is running on this port=}\\

          \hline
          \\
\textbf{Solution:}\\
\\


        \hline
        \\
\textbf{Log Method}\\
Details:
\rowcolor{white}{\verb=Services=}\\
OID:1.3.6.1.4.1.25623.1.0.10330\\
Version used:
\rowcolor{white}{\verb=2021-03-15T10:42:03Z=}\\
\end{longtable}

\begin{longtable}{|p{\textwidth * 1}|}
\hline
\rowcolor{gvm_log}{\color{white}{Log (CVSS: 0.0) }}\\
\rowcolor{gvm_log}{\color{white}{NVT: Response Time / No 404 Error Code Check}}\\
\hline
\endfirsthead
\hfill\ldots continued from previous page \ldots \\
\hline
\endhead
\hline
\ldots continues on next page \ldots \\
\endfoot
\hline
\endlastfoot
\\
\textbf{Summary}\\
This VT tests if the remote web server does not reply with a 404
  error code and checks if it is replying to the scanners requests in a reasonable amount of time.\\

        \hline
        \\
\textbf{Vulnerability Detection Result}\\
\rowcolor{white}{\verb=The service is responding with a 200 HTTP status code to non-existent files/urls=}\\
\rowcolor{white}{$\hookrightarrow$\verb=. The following pattern is used to work around possible false detections:=}\\
\rowcolor{white}{\verb=-----=}\\
\rowcolor{white}{\verb=TYPE=\verb-=-\verb="password"=}\\
\rowcolor{white}{\verb=-----=}\\

          \hline
          \\
\textbf{Solution:}\\
\\

          \hline
          \\
\textbf{Vulnerability Insight}\\
This web server might show the following issues:\\
  - it is [mis]configured in that it does not return '404 Not Found' error codes when a non-existent
  file is requested, perhaps returning a site map, search page, authentication page or redirect instead.\\
  The Scanner might enabled some counter measures for that, however they might be insufficient. If a
  great number of security issues are reported for this port, they might not all be accurate.\\
  - it doesn't response in a reasonable amount of time to various HTTP requests sent by this VT.\\
  In order to keep the scan total time to a reasonable amount, the remote web server might not be
  tested. If the remote server should be tested it has to be fixed to have it reply to the scanners
  requests in a reasonable amount of time.\\
  Alternatively the 'Maximum response time (in seconds)' preference could be raised to a higher
  value if longer scan times are accepted.\\


        \hline
        \\
\textbf{Log Method}\\
Details:
\rowcolor{white}{\verb=Response Time / No 404 Error Code Check=}\\
OID:1.3.6.1.4.1.25623.1.0.10386\\
Version used:
\rowcolor{white}{\verb=2020-11-27T13:32:50Z=}\\
\end{longtable}

\begin{footnotesize}\hyperref[host:192.168.178.29]{[ return to 192.168.178.29 ]}\end{footnotesize}
\subsection{192.168.178.27}
\label{host:192.168.178.27}

\begin{tabular}{ll}
Host scan start&Fri Apr 22 12:21:26 2022 UTC\\
Host scan end&Fri Apr 22 12:22:28 2022 UTC\\
\end{tabular}

\begin{longtable}{|l|l|}
\hline
\rowcolor{gvm_report}Service (Port)&Threat Level\\
\hline
\endfirsthead
\multicolumn{2}{l}{\hfill\ldots (continued) \ldots}\\
\hline
\rowcolor{gvm_report}Service (Port)&Threat Level\\
\hline
\endhead
\hline
\multicolumn{2}{l}{\ldots (continues) \ldots}\\
\endfoot
\hline
\endlastfoot
\hline
\hyperref[port:192.168.178.27 general/tcp Log]{general/tcp}&Log\\
\hline
\end{longtable}


%\subsection*{Security Issues and Fixes -- 192.168.178.27}

\subsubsection{Log general/tcp}
\label{port:192.168.178.27 general/tcp Log}

\begin{longtable}{|p{\textwidth * 1}|}
\hline
\rowcolor{gvm_log}{\color{white}{Log (CVSS: 0.0) }}\\
\rowcolor{gvm_log}{\color{white}{NVT: OS Detection Consolidation and Reporting}}\\
\hline
\endfirsthead
\hfill\ldots continued from previous page \ldots \\
\hline
\endhead
\hline
\ldots continues on next page \ldots \\
\endfoot
\hline
\endlastfoot
\\
\textbf{Summary}\\
This script consolidates the OS information detected by several
  VTs and tries to find the best matching OS.\\
  Furthermore it reports all previously collected information leading to this best matching OS. It
  also reports possible additional information which might help to improve the OS detection.\\
  If any of this information is wrong or could be improved please consider to report these to the
  referenced community portal.\\

        \hline
        \\
\textbf{Vulnerability Detection Result}\\
\rowcolor{white}{\verb=Best matching OS:=}\\
\rowcolor{white}{\verb=OS:           Linux Kernel=}\\
\rowcolor{white}{\verb=CPE:          cpe:/o:linux:kernel=}\\
\rowcolor{white}{\verb=Found by NVT: 1.3.6.1.4.1.25623.1.0.102002 (Operating System (OS) Detection (ICM=}\\
\rowcolor{white}{$\hookrightarrow$\verb=P))=}\\
\rowcolor{white}{\verb=Concluded from ICMP based OS fingerprint=}\\
\rowcolor{white}{\verb=Setting key "Host/runs_unixoide" based on this information=}\\
\rowcolor{white}{\verb=Other OS detections (in order of reliability):=}\\
\rowcolor{white}{\verb=OS:           FreeBSD=}\\
\rowcolor{white}{\verb=CPE:          cpe:/o:freebsd:freebsd=}\\
\rowcolor{white}{\verb=Found by NVT: 1.3.6.1.4.1.25623.1.0.102002 (Operating System (OS) Detection (ICM=}\\
\rowcolor{white}{$\hookrightarrow$\verb=P))=}\\
\rowcolor{white}{\verb=Concluded from ICMP based OS fingerprint=}\\

          \hline
          \\
\textbf{Solution:}\\
\\


        \hline
        \\
\textbf{Log Method}\\
Details:
\rowcolor{white}{\verb=OS Detection Consolidation and Reporting=}\\
OID:1.3.6.1.4.1.25623.1.0.105937\\
Version used:
\rowcolor{white}{\verb=2022-04-05T09:27:51Z=}\\

      \hline
      \\
\textbf{References}\\
\rowcolor{white}{\verb=url: https://community.greenbone.net/c/vulnerability-tests=}\\
\end{longtable}

\begin{longtable}{|p{\textwidth * 1}|}
\hline
\rowcolor{gvm_log}{\color{white}{Log (CVSS: 0.0) }}\\
\rowcolor{gvm_log}{\color{white}{NVT: Hostname Determination Reporting}}\\
\hline
\endfirsthead
\hfill\ldots continued from previous page \ldots \\
\hline
\endhead
\hline
\ldots continues on next page \ldots \\
\endfoot
\hline
\endlastfoot
\\
\textbf{Summary}\\
The script reports information on how the hostname
  of the target was determined.\\

        \hline
        \\
\textbf{Vulnerability Detection Result}\\
\rowcolor{white}{\verb=Hostname determination for IP 192.168.178.27:=}\\
\rowcolor{white}{\verb=Hostname|Source=}\\
\rowcolor{white}{\verb=192.168.178.27|IP-address=}\\

          \hline
          \\
\textbf{Solution:}\\
\\


        \hline
        \\
\textbf{Log Method}\\
Details:
\rowcolor{white}{\verb=Hostname Determination Reporting=}\\
OID:1.3.6.1.4.1.25623.1.0.108449\\
Version used:
\rowcolor{white}{\verb=2018-11-19T11:11:31Z=}\\
\end{longtable}

\begin{footnotesize}\hyperref[host:192.168.178.27]{[ return to 192.168.178.27 ]}\end{footnotesize}
\subsection{192.168.178.25}
\label{host:192.168.178.25}

\begin{tabular}{ll}
Host scan start&Fri Apr 22 12:21:26 2022 UTC\\
Host scan end&Fri Apr 22 12:22:28 2022 UTC\\
\end{tabular}

\begin{longtable}{|l|l|}
\hline
\rowcolor{gvm_report}Service (Port)&Threat Level\\
\hline
\endfirsthead
\multicolumn{2}{l}{\hfill\ldots (continued) \ldots}\\
\hline
\rowcolor{gvm_report}Service (Port)&Threat Level\\
\hline
\endhead
\hline
\multicolumn{2}{l}{\ldots (continues) \ldots}\\
\endfoot
\hline
\endlastfoot
\hline
\hyperref[port:192.168.178.25 general/tcp Log]{general/tcp}&Log\\
\hline
\end{longtable}


%\subsection*{Security Issues and Fixes -- 192.168.178.25}

\subsubsection{Log general/tcp}
\label{port:192.168.178.25 general/tcp Log}

\begin{longtable}{|p{\textwidth * 1}|}
\hline
\rowcolor{gvm_log}{\color{white}{Log (CVSS: 0.0) }}\\
\rowcolor{gvm_log}{\color{white}{NVT: OS Detection Consolidation and Reporting}}\\
\hline
\endfirsthead
\hfill\ldots continued from previous page \ldots \\
\hline
\endhead
\hline
\ldots continues on next page \ldots \\
\endfoot
\hline
\endlastfoot
\\
\textbf{Summary}\\
This script consolidates the OS information detected by several
  VTs and tries to find the best matching OS.\\
  Furthermore it reports all previously collected information leading to this best matching OS. It
  also reports possible additional information which might help to improve the OS detection.\\
  If any of this information is wrong or could be improved please consider to report these to the
  referenced community portal.\\

        \hline
        \\
\textbf{Vulnerability Detection Result}\\
\rowcolor{white}{\verb=Best matching OS:=}\\
\rowcolor{white}{\verb=OS:           Linux Kernel=}\\
\rowcolor{white}{\verb=CPE:          cpe:/o:linux:kernel=}\\
\rowcolor{white}{\verb=Found by NVT: 1.3.6.1.4.1.25623.1.0.102002 (Operating System (OS) Detection (ICM=}\\
\rowcolor{white}{$\hookrightarrow$\verb=P))=}\\
\rowcolor{white}{\verb=Concluded from ICMP based OS fingerprint=}\\
\rowcolor{white}{\verb=Setting key "Host/runs_unixoide" based on this information=}\\
\rowcolor{white}{\verb=Other OS detections (in order of reliability):=}\\
\rowcolor{white}{\verb=OS:           FreeBSD=}\\
\rowcolor{white}{\verb=CPE:          cpe:/o:freebsd:freebsd=}\\
\rowcolor{white}{\verb=Found by NVT: 1.3.6.1.4.1.25623.1.0.102002 (Operating System (OS) Detection (ICM=}\\
\rowcolor{white}{$\hookrightarrow$\verb=P))=}\\
\rowcolor{white}{\verb=Concluded from ICMP based OS fingerprint=}\\

          \hline
          \\
\textbf{Solution:}\\
\\


        \hline
        \\
\textbf{Log Method}\\
Details:
\rowcolor{white}{\verb=OS Detection Consolidation and Reporting=}\\
OID:1.3.6.1.4.1.25623.1.0.105937\\
Version used:
\rowcolor{white}{\verb=2022-04-05T09:27:51Z=}\\

      \hline
      \\
\textbf{References}\\
\rowcolor{white}{\verb=url: https://community.greenbone.net/c/vulnerability-tests=}\\
\end{longtable}

\begin{longtable}{|p{\textwidth * 1}|}
\hline
\rowcolor{gvm_log}{\color{white}{Log (CVSS: 0.0) }}\\
\rowcolor{gvm_log}{\color{white}{NVT: Hostname Determination Reporting}}\\
\hline
\endfirsthead
\hfill\ldots continued from previous page \ldots \\
\hline
\endhead
\hline
\ldots continues on next page \ldots \\
\endfoot
\hline
\endlastfoot
\\
\textbf{Summary}\\
The script reports information on how the hostname
  of the target was determined.\\

        \hline
        \\
\textbf{Vulnerability Detection Result}\\
\rowcolor{white}{\verb=Hostname determination for IP 192.168.178.25:=}\\
\rowcolor{white}{\verb=Hostname|Source=}\\
\rowcolor{white}{\verb=192.168.178.25|IP-address=}\\

          \hline
          \\
\textbf{Solution:}\\
\\


        \hline
        \\
\textbf{Log Method}\\
Details:
\rowcolor{white}{\verb=Hostname Determination Reporting=}\\
OID:1.3.6.1.4.1.25623.1.0.108449\\
Version used:
\rowcolor{white}{\verb=2018-11-19T11:11:31Z=}\\
\end{longtable}

\begin{footnotesize}\hyperref[host:192.168.178.25]{[ return to 192.168.178.25 ]}\end{footnotesize}
\subsection{192.168.178.36}
\label{host:192.168.178.36}

\begin{tabular}{ll}
Host scan start&Fri Apr 22 12:21:26 2022 UTC\\
Host scan end&Fri Apr 22 12:23:47 2022 UTC\\
\end{tabular}

\begin{longtable}{|l|l|}
\hline
\rowcolor{gvm_report}Service (Port)&Threat Level\\
\hline
\endfirsthead
\multicolumn{2}{l}{\hfill\ldots (continued) \ldots}\\
\hline
\rowcolor{gvm_report}Service (Port)&Threat Level\\
\hline
\endhead
\hline
\multicolumn{2}{l}{\ldots (continues) \ldots}\\
\endfoot
\hline
\endlastfoot
\hline
\hyperref[port:192.168.178.36 general/tcp Log]{general/tcp}&Log\\
\hline
\end{longtable}


%\subsection*{Security Issues and Fixes -- 192.168.178.36}

\subsubsection{Log general/tcp}
\label{port:192.168.178.36 general/tcp Log}

\begin{longtable}{|p{\textwidth * 1}|}
\hline
\rowcolor{gvm_log}{\color{white}{Log (CVSS: 0.0) }}\\
\rowcolor{gvm_log}{\color{white}{NVT: OS Detection Consolidation and Reporting}}\\
\hline
\endfirsthead
\hfill\ldots continued from previous page \ldots \\
\hline
\endhead
\hline
\ldots continues on next page \ldots \\
\endfoot
\hline
\endlastfoot
\\
\textbf{Summary}\\
This script consolidates the OS information detected by several
  VTs and tries to find the best matching OS.\\
  Furthermore it reports all previously collected information leading to this best matching OS. It
  also reports possible additional information which might help to improve the OS detection.\\
  If any of this information is wrong or could be improved please consider to report these to the
  referenced community portal.\\

        \hline
        \\
\textbf{Vulnerability Detection Result}\\
\rowcolor{white}{\verb=No Best matching OS identified. Please see the VT 'Unknown OS and Service Banner=}\\
\rowcolor{white}{$\hookrightarrow$\verb= Reporting' (OID: 1.3.6.1.4.1.25623.1.0.108441) for possible ways to identify =}\\
\rowcolor{white}{$\hookrightarrow$\verb=this OS.=}\\

          \hline
          \\
\textbf{Solution:}\\
\\


        \hline
        \\
\textbf{Log Method}\\
Details:
\rowcolor{white}{\verb=OS Detection Consolidation and Reporting=}\\
OID:1.3.6.1.4.1.25623.1.0.105937\\
Version used:
\rowcolor{white}{\verb=2022-04-05T09:27:51Z=}\\

      \hline
      \\
\textbf{References}\\
\rowcolor{white}{\verb=url: https://community.greenbone.net/c/vulnerability-tests=}\\
\end{longtable}

\begin{longtable}{|p{\textwidth * 1}|}
\hline
\rowcolor{gvm_log}{\color{white}{Log (CVSS: 0.0) }}\\
\rowcolor{gvm_log}{\color{white}{NVT: Hostname Determination Reporting}}\\
\hline
\endfirsthead
\hfill\ldots continued from previous page \ldots \\
\hline
\endhead
\hline
\ldots continues on next page \ldots \\
\endfoot
\hline
\endlastfoot
\\
\textbf{Summary}\\
The script reports information on how the hostname
  of the target was determined.\\

        \hline
        \\
\textbf{Vulnerability Detection Result}\\
\rowcolor{white}{\verb=Hostname determination for IP 192.168.178.36:=}\\
\rowcolor{white}{\verb=Hostname|Source=}\\
\rowcolor{white}{\verb=iphone-fj.fritz.box|Reverse-DNS=}\\

          \hline
          \\
\textbf{Solution:}\\
\\


        \hline
        \\
\textbf{Log Method}\\
Details:
\rowcolor{white}{\verb=Hostname Determination Reporting=}\\
OID:1.3.6.1.4.1.25623.1.0.108449\\
Version used:
\rowcolor{white}{\verb=2018-11-19T11:11:31Z=}\\
\end{longtable}

\begin{footnotesize}\hyperref[host:192.168.178.36]{[ return to 192.168.178.36 ]}\end{footnotesize}
\subsection{192.168.178.28}
\label{host:192.168.178.28}

\begin{tabular}{ll}
Host scan start&Fri Apr 22 12:21:27 2022 UTC\\
Host scan end&Fri Apr 22 12:25:01 2022 UTC\\
\end{tabular}

\begin{longtable}{|l|l|}
\hline
\rowcolor{gvm_report}Service (Port)&Threat Level\\
\hline
\endfirsthead
\multicolumn{2}{l}{\hfill\ldots (continued) \ldots}\\
\hline
\rowcolor{gvm_report}Service (Port)&Threat Level\\
\hline
\endhead
\hline
\multicolumn{2}{l}{\ldots (continues) \ldots}\\
\endfoot
\hline
\endlastfoot
\hline
\hyperref[port:192.168.178.28 5094/tcp Log]{5094/tcp}&Log\\
\hline
\hyperref[port:192.168.178.28 general/tcp Log]{general/tcp}&Log\\
\hline
\end{longtable}


%\subsection*{Security Issues and Fixes -- 192.168.178.28}

\subsubsection{Log 5094/tcp}
\label{port:192.168.178.28 5094/tcp Log}

\begin{longtable}{|p{\textwidth * 1}|}
\hline
\rowcolor{gvm_log}{\color{white}{Log (CVSS: 0.0) }}\\
\rowcolor{gvm_log}{\color{white}{NVT: Service Detection with 'HELP' Request'}}\\
\hline
\endfirsthead
\hfill\ldots continued from previous page \ldots \\
\hline
\endhead
\hline
\ldots continues on next page \ldots \\
\endfoot
\hline
\endlastfoot
\\
\textbf{Summary}\\
This plugin performs service detection.\\
  This plugin is a complement of find\_service.nasl. It sends a 'HELP'
  request to the remaining unknown services and tries to identify them.\\

        \hline
        \\
\textbf{Vulnerability Detection Result}\\
\rowcolor{white}{\verb=A service responding with an SSL/TLS alert seems to be running on this port.=}\\

          \hline
          \\
\textbf{Solution:}\\
\\


        \hline
        \\
\textbf{Log Method}\\
Details:
\rowcolor{white}{\verb=Service Detection with 'HELP' Request'=}\\
OID:1.3.6.1.4.1.25623.1.0.11153\\
Version used:
\rowcolor{white}{\verb=2022-01-31T16:15:30Z=}\\
\end{longtable}

\begin{longtable}{|p{\textwidth * 1}|}
\hline
\rowcolor{gvm_log}{\color{white}{Log (CVSS: 0.0) }}\\
\rowcolor{gvm_log}{\color{white}{NVT: SSL/TLS: Version Detection}}\\
\hline
\endfirsthead
\hfill\ldots continued from previous page \ldots \\
\hline
\endhead
\hline
\ldots continues on next page \ldots \\
\endfoot
\hline
\endlastfoot
\\
\textbf{Summary}\\
Enumeration and reporting of SSL/TLS protocol versions supported
  by a remote service.\\

        \hline
        \\
\textbf{Vulnerability Detection Result}\\
\rowcolor{white}{\verb=The remote SSL/TLS service supports the following SSL/TLS protocol version(s):=}\\
\rowcolor{white}{\verb=TLSv1.3=}\\

          \hline
          \\
\textbf{Solution:}\\
\\


        \hline
        \\
\textbf{Log Method}\\
Sends multiple connection requests to the remote service and
  attempts to determine the SSL/TLS protocol versions supported by the service from the replies.\\
  Note: The supported SSL/TLS protocol versions included in the report of this VT are reported
  independently from the allowed / supported SSL/TLS ciphers.\\
Details:
\rowcolor{white}{\verb=SSL/TLS: Version Detection=}\\
OID:1.3.6.1.4.1.25623.1.0.105782\\
Version used:
\rowcolor{white}{\verb=2021-12-06T15:42:24Z=}\\
\end{longtable}

\begin{footnotesize}\hyperref[host:192.168.178.28]{[ return to 192.168.178.28 ]}\end{footnotesize}
\subsubsection{Log general/tcp}
\label{port:192.168.178.28 general/tcp Log}

\begin{longtable}{|p{\textwidth * 1}|}
\hline
\rowcolor{gvm_log}{\color{white}{Log (CVSS: 0.0) }}\\
\rowcolor{gvm_log}{\color{white}{NVT: Hostname Determination Reporting}}\\
\hline
\endfirsthead
\hfill\ldots continued from previous page \ldots \\
\hline
\endhead
\hline
\ldots continues on next page \ldots \\
\endfoot
\hline
\endlastfoot
\\
\textbf{Summary}\\
The script reports information on how the hostname
  of the target was determined.\\

        \hline
        \\
\textbf{Vulnerability Detection Result}\\
\rowcolor{white}{\verb=Hostname determination for IP 192.168.178.28:=}\\
\rowcolor{white}{\verb=Hostname|Source=}\\
\rowcolor{white}{\verb=s20-fe-von-carina.fritz.box|Reverse-DNS=}\\

          \hline
          \\
\textbf{Solution:}\\
\\


        \hline
        \\
\textbf{Log Method}\\
Details:
\rowcolor{white}{\verb=Hostname Determination Reporting=}\\
OID:1.3.6.1.4.1.25623.1.0.108449\\
Version used:
\rowcolor{white}{\verb=2018-11-19T11:11:31Z=}\\
\end{longtable}

\begin{longtable}{|p{\textwidth * 1}|}
\hline
\rowcolor{gvm_log}{\color{white}{Log (CVSS: 0.0) }}\\
\rowcolor{gvm_log}{\color{white}{NVT: OS Detection Consolidation and Reporting}}\\
\hline
\endfirsthead
\hfill\ldots continued from previous page \ldots \\
\hline
\endhead
\hline
\ldots continues on next page \ldots \\
\endfoot
\hline
\endlastfoot
\\
\textbf{Summary}\\
This script consolidates the OS information detected by several
  VTs and tries to find the best matching OS.\\
  Furthermore it reports all previously collected information leading to this best matching OS. It
  also reports possible additional information which might help to improve the OS detection.\\
  If any of this information is wrong or could be improved please consider to report these to the
  referenced community portal.\\

        \hline
        \\
\textbf{Vulnerability Detection Result}\\
\rowcolor{white}{\verb=Best matching OS:=}\\
\rowcolor{white}{\verb=OS:           Linux Kernel=}\\
\rowcolor{white}{\verb=CPE:          cpe:/o:linux:kernel=}\\
\rowcolor{white}{\verb=Found by NVT: 1.3.6.1.4.1.25623.1.0.102002 (Operating System (OS) Detection (ICM=}\\
\rowcolor{white}{$\hookrightarrow$\verb=P))=}\\
\rowcolor{white}{\verb=Concluded from ICMP based OS fingerprint=}\\
\rowcolor{white}{\verb=Setting key "Host/runs_unixoide" based on this information=}\\

          \hline
          \\
\textbf{Solution:}\\
\\


        \hline
        \\
\textbf{Log Method}\\
Details:
\rowcolor{white}{\verb=OS Detection Consolidation and Reporting=}\\
OID:1.3.6.1.4.1.25623.1.0.105937\\
Version used:
\rowcolor{white}{\verb=2022-04-05T09:27:51Z=}\\

      \hline
      \\
\textbf{References}\\
\rowcolor{white}{\verb=url: https://community.greenbone.net/c/vulnerability-tests=}\\
\end{longtable}

\begin{footnotesize}\hyperref[host:192.168.178.28]{[ return to 192.168.178.28 ]}\end{footnotesize}
\subsection{192.168.178.42}
\label{host:192.168.178.42}

\begin{tabular}{ll}
Host scan start&Fri Apr 22 12:21:26 2022 UTC\\
Host scan end&Fri Apr 22 12:31:43 2022 UTC\\
\end{tabular}

\begin{longtable}{|l|l|}
\hline
\rowcolor{gvm_report}Service (Port)&Threat Level\\
\hline
\endfirsthead
\multicolumn{2}{l}{\hfill\ldots (continued) \ldots}\\
\hline
\rowcolor{gvm_report}Service (Port)&Threat Level\\
\hline
\endhead
\hline
\multicolumn{2}{l}{\ldots (continues) \ldots}\\
\endfoot
\hline
\endlastfoot
\hline
\hyperref[port:192.168.178.42 general/tcp Log]{general/tcp}&Log\\
\hline
\end{longtable}


%\subsection*{Security Issues and Fixes -- 192.168.178.42}

\subsubsection{Log general/tcp}
\label{port:192.168.178.42 general/tcp Log}

\begin{longtable}{|p{\textwidth * 1}|}
\hline
\rowcolor{gvm_log}{\color{white}{Log (CVSS: 0.0) }}\\
\rowcolor{gvm_log}{\color{white}{NVT: OS Detection Consolidation and Reporting}}\\
\hline
\endfirsthead
\hfill\ldots continued from previous page \ldots \\
\hline
\endhead
\hline
\ldots continues on next page \ldots \\
\endfoot
\hline
\endlastfoot
\\
\textbf{Summary}\\
This script consolidates the OS information detected by several
  VTs and tries to find the best matching OS.\\
  Furthermore it reports all previously collected information leading to this best matching OS. It
  also reports possible additional information which might help to improve the OS detection.\\
  If any of this information is wrong or could be improved please consider to report these to the
  referenced community portal.\\

        \hline
        \\
\textbf{Vulnerability Detection Result}\\
\rowcolor{white}{\verb=No Best matching OS identified. Please see the VT 'Unknown OS and Service Banner=}\\
\rowcolor{white}{$\hookrightarrow$\verb= Reporting' (OID: 1.3.6.1.4.1.25623.1.0.108441) for possible ways to identify =}\\
\rowcolor{white}{$\hookrightarrow$\verb=this OS.=}\\

          \hline
          \\
\textbf{Solution:}\\
\\


        \hline
        \\
\textbf{Log Method}\\
Details:
\rowcolor{white}{\verb=OS Detection Consolidation and Reporting=}\\
OID:1.3.6.1.4.1.25623.1.0.105937\\
Version used:
\rowcolor{white}{\verb=2022-04-05T09:27:51Z=}\\

      \hline
      \\
\textbf{References}\\
\rowcolor{white}{\verb=url: https://community.greenbone.net/c/vulnerability-tests=}\\
\end{longtable}

\begin{longtable}{|p{\textwidth * 1}|}
\hline
\rowcolor{gvm_log}{\color{white}{Log (CVSS: 0.0) }}\\
\rowcolor{gvm_log}{\color{white}{NVT: Hostname Determination Reporting}}\\
\hline
\endfirsthead
\hfill\ldots continued from previous page \ldots \\
\hline
\endhead
\hline
\ldots continues on next page \ldots \\
\endfoot
\hline
\endlastfoot
\\
\textbf{Summary}\\
The script reports information on how the hostname
  of the target was determined.\\

        \hline
        \\
\textbf{Vulnerability Detection Result}\\
\rowcolor{white}{\verb=Hostname determination for IP 192.168.178.42:=}\\
\rowcolor{white}{\verb=Hostname|Source=}\\
\rowcolor{white}{\verb=nasale.fritz.box|Reverse-DNS=}\\

          \hline
          \\
\textbf{Solution:}\\
\\


        \hline
        \\
\textbf{Log Method}\\
Details:
\rowcolor{white}{\verb=Hostname Determination Reporting=}\\
OID:1.3.6.1.4.1.25623.1.0.108449\\
Version used:
\rowcolor{white}{\verb=2018-11-19T11:11:31Z=}\\
\end{longtable}

\begin{footnotesize}\hyperref[host:192.168.178.42]{[ return to 192.168.178.42 ]}\end{footnotesize}
\subsection{192.168.178.41}
\label{host:192.168.178.41}

\begin{tabular}{ll}
Host scan start&Fri Apr 22 12:21:26 2022 UTC\\
Host scan end&Fri Apr 22 12:31:44 2022 UTC\\
\end{tabular}

\begin{longtable}{|l|l|}
\hline
\rowcolor{gvm_report}Service (Port)&Threat Level\\
\hline
\endfirsthead
\multicolumn{2}{l}{\hfill\ldots (continued) \ldots}\\
\hline
\rowcolor{gvm_report}Service (Port)&Threat Level\\
\hline
\endhead
\hline
\multicolumn{2}{l}{\ldots (continues) \ldots}\\
\endfoot
\hline
\endlastfoot
\hline
\hyperref[port:192.168.178.41 general/tcp Log]{general/tcp}&Log\\
\hline
\end{longtable}


%\subsection*{Security Issues and Fixes -- 192.168.178.41}

\subsubsection{Log general/tcp}
\label{port:192.168.178.41 general/tcp Log}

\begin{longtable}{|p{\textwidth * 1}|}
\hline
\rowcolor{gvm_log}{\color{white}{Log (CVSS: 0.0) }}\\
\rowcolor{gvm_log}{\color{white}{NVT: OS Detection Consolidation and Reporting}}\\
\hline
\endfirsthead
\hfill\ldots continued from previous page \ldots \\
\hline
\endhead
\hline
\ldots continues on next page \ldots \\
\endfoot
\hline
\endlastfoot
\\
\textbf{Summary}\\
This script consolidates the OS information detected by several
  VTs and tries to find the best matching OS.\\
  Furthermore it reports all previously collected information leading to this best matching OS. It
  also reports possible additional information which might help to improve the OS detection.\\
  If any of this information is wrong or could be improved please consider to report these to the
  referenced community portal.\\

        \hline
        \\
\textbf{Vulnerability Detection Result}\\
\rowcolor{white}{\verb=No Best matching OS identified. Please see the VT 'Unknown OS and Service Banner=}\\
\rowcolor{white}{$\hookrightarrow$\verb= Reporting' (OID: 1.3.6.1.4.1.25623.1.0.108441) for possible ways to identify =}\\
\rowcolor{white}{$\hookrightarrow$\verb=this OS.=}\\

          \hline
          \\
\textbf{Solution:}\\
\\


        \hline
        \\
\textbf{Log Method}\\
Details:
\rowcolor{white}{\verb=OS Detection Consolidation and Reporting=}\\
OID:1.3.6.1.4.1.25623.1.0.105937\\
Version used:
\rowcolor{white}{\verb=2022-04-05T09:27:51Z=}\\

      \hline
      \\
\textbf{References}\\
\rowcolor{white}{\verb=url: https://community.greenbone.net/c/vulnerability-tests=}\\
\end{longtable}

\begin{longtable}{|p{\textwidth * 1}|}
\hline
\rowcolor{gvm_log}{\color{white}{Log (CVSS: 0.0) }}\\
\rowcolor{gvm_log}{\color{white}{NVT: Hostname Determination Reporting}}\\
\hline
\endfirsthead
\hfill\ldots continued from previous page \ldots \\
\hline
\endhead
\hline
\ldots continues on next page \ldots \\
\endfoot
\hline
\endlastfoot
\\
\textbf{Summary}\\
The script reports information on how the hostname
  of the target was determined.\\

        \hline
        \\
\textbf{Vulnerability Detection Result}\\
\rowcolor{white}{\verb=Hostname determination for IP 192.168.178.41:=}\\
\rowcolor{white}{\verb=Hostname|Source=}\\
\rowcolor{white}{\verb=nasale.fritz.box|Reverse-DNS=}\\

          \hline
          \\
\textbf{Solution:}\\
\\


        \hline
        \\
\textbf{Log Method}\\
Details:
\rowcolor{white}{\verb=Hostname Determination Reporting=}\\
OID:1.3.6.1.4.1.25623.1.0.108449\\
Version used:
\rowcolor{white}{\verb=2018-11-19T11:11:31Z=}\\
\end{longtable}

\begin{footnotesize}\hyperref[host:192.168.178.41]{[ return to 192.168.178.41 ]}\end{footnotesize}

\begin{center}
\medskip
\rule{\textwidth}{0.1pt}

This file was automatically generated.
\end{center}

\end{document}
